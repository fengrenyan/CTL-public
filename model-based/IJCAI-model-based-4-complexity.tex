%%%% ijcai20-multiauthor.tex

\typeout{IJCAI--PRICAI--20 Multiple authors example}

% These are the instructions for authors for IJCAI-20.

\documentclass{article}
\pdfpagewidth=8.5in
\pdfpageheight=11in
% The file ijcai20.sty is NOT the same than previous years'
\usepackage{ijcai20}

% Use the postscript times font!
\usepackage{times}
\renewcommand*\ttdefault{txtt}
\usepackage{soul}
\usepackage{url}
\usepackage[hidelinks]{hyperref}
\usepackage[utf8]{inputenc}
\usepackage[small]{caption}
\usepackage{graphicx}
\usepackage{amsmath}
\usepackage{booktabs}
\urlstyle{same}



\usepackage{setspace}
\usepackage{times}  %Required
\usepackage{helvet}  %Required
\usepackage{courier}  %Required
\usepackage{url}  %Required
%\usepackage{graphicx}  %Required

\usepackage{enumerate}


%\usepackage{algorithm}
%\usepackage{algorithmic}

\usepackage{amssymb}
\usepackage{enumerate}

\usepackage{subfigure}

\usepackage[linesnumbered,boxed,ruled,commentsnumbered]{algorithm2e}


% the following package is optional:
%\usepackage{latexsym}

% Following comment is from ijcai97-submit.tex:
% The preparation of these files was supported by Schlumberger Palo Alto
% Research, AT\&T Bell Laboratories, and Morgan Kaufmann Publishers.
% Shirley Jowell, of Morgan Kaufmann Publishers, and Peter F.
% Patel-Schneider, of AT\&T Bell Laboratories collaborated on their
% preparation.

% These instructions can be modified and used in other conferences as long
% as credit to the authors and supporting agencies is retained, this notice
% is not changed, and further modification or reuse is not restricted.
% Neither Shirley Jowell nor Peter F. Patel-Schneider can be listed as
% contacts for providing assistance without their prior permission.

% To use for other conferences, change references to files and the
% conference appropriate and use other authors, contacts, publishers, and
% organizations.
% Also change the deadline and address for returning papers and the length and
% page charge instructions.
% Put where the files are available in the appropriate places.

\title{Forgetting in CTL to Compute Necessary and Sufficient Conditions}

%\author{
%Renyan Feng$^1$\and
%Yisong Wang$^2$\footnote{Contact Author}\and
%Stefan Schlobach$^{3}$\footnote{Contact Author}\And
%Erman Acar$^4$\\
%\affiliations
%$^{1,2}$GuiZhou University\\
%$^{3,4}$Vrije Universiteit Amsterdam\\
%\emails
%fengrenyan@gmail.com,
%yswang@gzu.edu.cn,
%\{k.s.schlobach, erman.acar\}@vu.nl
%}

\begin{document}


\newcommand{\tuple}[1]{{\langle{#1}\rangle}}
\newcommand{\Mod}{\textit{Mod}}
\newcommand\ie{{\it i.e. }}
\newcommand\eg{{\it e.g.}}
%\newcommand\st{{\it s.t. }}
\newtheorem{definition}{Definition}
\newtheorem{examp}{Example}
\newenvironment{example}{\begin{examp}\rm}{\end{examp}}
\newtheorem{lemma}{Lemma}
\newtheorem{proposition}{Proposition}
\newtheorem{theorem}{Theorem}
\newtheorem{corollary}[theorem]{Corollary}
\newenvironment{proof}{{\bf Proof:}}{\hfill\rule{2mm}{2mm}\\ }
\newcommand{\rto}{\rightarrow}
\newcommand{\lto}{\leftarrow}
\newcommand{\lrto}{\leftrightarrow}
\newcommand{\Rto}{\Rightarrow}
\newcommand{\Lto}{\Leftarrow}
\newcommand{\LRto}{\Leftrightarrow}
\newcommand{\Var}{\textit{Var}}
\newcommand{\Forget}{\textit{Forget}}
\newcommand{\KForget}{\textit{KForget}}
\newcommand{\TForget}{\textit{TForget}}
%\newcommand{\forget}{\textit{forget}}
\newcommand{\Fst}{\textit{Fst}}
\newcommand{\dep}{\textit{dep}}
\newcommand{\term}{\textit{term}}
\newcommand{\literal}{\textit{literal}}

\newcommand{\Atom}{\mathcal{A}}
\newcommand{\SFive}{\textbf{S5}}
\newcommand{\MPK}{\textsc{k}}
\newcommand{\MPB}{\textsc{b}}
\newcommand{\MPT}{\textsc{t}}
\newcommand{\MPA}{\forall}
\newcommand{\MPE}{\exists}

\newcommand{\DNF}{\textit{DNF}}
\newcommand{\CNF}{\textit{CNF}}

\newcommand{\degree}{\textit{degree}}
\newcommand{\sunfold}{\textit{sunfold}}

\newcommand{\Pos}{\textit{Pos}}
\newcommand{\Neg}{\textit{Neg}}
\newcommand\wrt{{\it w.r.t.}}
\newcommand{\Hm} {{\cal M}}
\newcommand{\Hw} {{\cal W}}
\newcommand{\Hr} {{\cal R}}
\newcommand{\Hb} {{\cal B}}
\newcommand{\Ha} {{\cal A}}

\newcommand{\Dsj}{\triangledown}

\newcommand{\wnext}{\widetilde{\bigcirc}}
\newcommand{\nex}{\bigcirc}
\newcommand{\ness}{\square}
\newcommand{\qness}{\boxminus}
\newcommand{\wqnext}{\widetilde{\circleddash}}
\newcommand{\qnext}{\circleddash}
\newcommand{\may}{\lozenge}
\newcommand{\qmay}{\blacklozenge}
\newcommand{\unt} {{\cal U}}
\newcommand{\since} {{\cal S}}
\newcommand{\SNF} {\textit{SNF$_C$}}
\newcommand{\start}{\textbf{start}}
\newcommand{\Elm}{\textit{Elm}}
\newcommand{\simp}{\textbf{simp}}
\newcommand{\nnf}{\textbf{nnf}}

\newcommand{\CTL}{\textrm{CTL}}
\newcommand{\Ind}{\textrm{Ind}}
\newcommand{\Tran}{\textrm{Tran}}
\newcommand{\Sub}{\textrm{Sub}}
\newcommand{\forget}{{\textsc{f}_\CTL}}
\newcommand{\ALL}{\textsc{a}}
\newcommand{\EXIST}{\textsc{e}}
\newcommand{\NEXT}{\textsc{x}}
\newcommand{\FUTURE}{\textsc{f}}
\newcommand{\UNTIL}{\textsc{u}}
\newcommand{\GLOBAL}{\textsc{g}}
\newcommand{\UNLESS}{\textsc{w}}
\newcommand{\Def}{\textrm{def}}
\newcommand{\IR}{\textrm{IR}}
\newcommand{\Tr}{\textrm{Tr}}
\newcommand{\dis}{\textrm{dis}}
\def\PP{\ensuremath{\textbf{PP}}}
\def\NgP{\ensuremath{\textbf{NP}}}
\def\W{\ensuremath{\textbf{W}}}
\newcommand{\Pre}{\textrm{Pre}}
\newcommand{\Post}{\textrm{Post}}


\newcommand{\CTLsnf}{{\textsc{SNF}_{\textsc{ctl}}^g}}
\newcommand{\ResC}{{\textsc{R}_{\textsc{ctl}}^{\succ, S}}}
\newcommand{\CTLforget}{{\textsc{F}_{\textsc{ctl}}}}
\newcommand{\Refine}{\textsc{Refine}}
\newcommand{\cf}{\textrm{cf.}}
\newcommand{\NEXP}{\textmd{\rm NEXP}}
\newcommand{\EXP}{\textmd{\rm EXP}}
\newcommand{\coNEXP}{\textmd{\rm co-NEXP}}
\newcommand{\NP}{\textmd{\rm NP}}
\newcommand{\coNP}{\textmd{\rm co-NP}}
\newcommand{\Pol}{\textmd{\rm P}}
\newcommand{\BH}[1]{\textmd{\rm BH}_{#1}}
\newcommand{\coBH}[1]{\textmd{\rm co-BH}_{#1}}
\newcommand{\Empty}{\varnothing}
\newcommand{\NLOG}{\textmd{\rm NLOG}}
\newcommand{\DeltaP}[1]{\Delta_{#1}^{p}}
\newcommand{\PIP}[1]{\Pi_{#1}^{p}}
\newcommand{\SigmaP}[1]{\Sigma_{#1}^{p}}



\maketitle

\begin{abstract}
%This article describes a method to compute the SNC (WSC) of a given property (a \CTL\ formula) under a given transition system (expressed as a Kripke structure) by proposing a semantic forgetting for \CTL.
%%Model checking is an automatic, model-based, property-verification approach. It is intended to be used for concurrent, reactive systems and originated as a post-development methodology.
%%Specification, which prescribes what the system has to do and what not, is used to product the properties that a system should satisfy.
%%Computation Tree Logic (\CTL) is one of the main logical formalisms for program specification and verification.
%%In order to compute the SNC (WSC) of a given property (a \CTL\ formula) under a given transition system, we study forgetting in CTL from the sematic forgetting point of view.
%We show that the CTL system is close under our definition of forgetting, and this definition satisfies those four postulates of forgetting at first.
%By changing a transition system $\Hm$ into its characteristic formula, we can compute the WSC (SNC) of a property under the transition system.
%We also investigate some properties and algorithm for the forgetting.
%%And then we will show the SNC and WSC can be computed by using the technology of forgetting.
%%Besides, an algorithm model-based has been put forward to compute forgetting in \CTL.
%%Keywords: Forgetting, \CTL, Bisimulation, Specification, Resolution.

%Computation Tree Logic  (\CTL) is
%one of the central formalisms in formal verification.  As a specification language, it is used to express a property that the system at hand is expected to satisfy. From both the verification and the system design points of view, some information content of such property might become irrelevant for the system due to various reasons e.g., it might become obsolete by time, or perhaps infeasible due to practical difficulties such that a further abstraction might be more desirable.  In such regard,  the \emph{strongest necessary condition} (SNC) and the \emph{weakest sufficient condition}  (WSC)  of a given specification are very informative; hence, might be of crucial importance.  In such scenario, how to subtract such piece of information without altering the relevant system behaviour or violating the existing specifications rises as a natural question.  To address such scenario in a principled way, we develop a formal framework based on the notion of \emph{forgetting} (a line of research in knowledge representation) for \CTL, and investigate its theoretical properties. In particular,  we show that our framework satisfies essential postulates of forgetting, and can be used to compute SNC and WSC of a \CTL\ specification under a given model. Furthermore, we outline the computational complexity of our work, including various results for the relevant fragment $\CTL_{\ALL\FUTURE}$.

Computation Tree Logic  (\CTL) is
one of the central formalisms in formal verification.  As a specification language, it is used to express a property that the system at hand is expected to satisfy. From both the verification and the system design points of view, some information content of such property might become irrelevant for the system due to various reasons e.g., it might become obsolete by time, or perhaps infeasible due to practical difficulties. Then, the problems arises on how to subtract such piece of information without altering the relevant system behaviour or violating the existing specifications.  Moreover, in such a scenario, two crucial notions are informative: the \emph{strongest necessary condition} (SNC)  and the \emph{weakest sufficient condition}  (WSC)   of a given property.

To address such a scenario in a principled way,  we introduce a \emph{forgetting}-based approach in \CTL\ and show that it can be used to compute SNC and WSC of a property under a given model.  We study its theoretical properties and also show that our notion of forgetting satisfies existing essential postulates. Furthermore, we analyse the computational complexity of basic tasks, including various results for the relevant fragment $\CTL_{\ALL\FUTURE}$.

\end{abstract}

\section{Introduction}
\emph{Weakest precondition}, we also call \emph{weakest sufficient condition} (WSC), is introduced by Dijkstra in~\cite{dijkstra1978guarded}.
\emph{Strongest postcondition} (we also call \emph{strongest necessary condition} (SNC)), a dual concept, was introduced subsequently.
WSC was widely used in program \emph{verification}, especially in generating counterexamples~\cite{dailler2018instrumenting} and refinement of  system~\cite{woodcock1990refinement}.
%In program correctness methods, SNC and WSC meet their toughest challenge when
%they deal with iterative constructs~\cite{mraihi2011computing}.
{\em Computation Tree Logic} (\CTL)~\cite{clarke1981design} Model modification, which has been developed in~\cite{ramdani2019r,martinez2016ctl,ding2006ctl}, is an extension of refinement of system.
This paper explore a method to compute the WSC of a property (a \CTL\ formula) under a given model system that may be modified for guiding \CTL\ Model modification.
%It is known that the computing of WSC for code fragment $S$ with respect to assertion $Q$  requires $S$  must terminate~\cite{tremblay1996logic}.
%test
It is known that the computing of WSC for code fragment $S$ with respect to assertion $Q$  requires $S$  must terminate~\cite{tremblay1996logic} due to it just concerns relation among input values and output values.
However, in the case of model checking, it concerns properties about execution runs, which may not be terminate.
%For example, the concurrent system with there is at least one successor state for each state $s$ in this system, the termination of this system is impossible.
%However, in \emph{model checking} of concurrent system with there is at least one successor state for each state $s$ in this system, the termination of this system is impossible.
%Informally, given a transition system $\Hm$ with $s_0$ as an initial state and a specification $\varphi$ (in this article we suppose it is a {\em Computation Tree Logic}
%(\CTL\ in short)~\cite{clarke1981design} formula), we should decide whether $(\Hm, s_0) \models \varphi$. It is a good thing if $(\Hm, s_0) \models \varphi$ indeed. However, if $(\Hm, s_0) \nvDash \varphi$, how can we find the WSC $\psi$ under a given set of atoms such that $(\Hm, s_0) \models \psi \supset \varphi$.
It can be shown by the following example.

\begin{example}\label{exmp:1}
 We give a simplified variant of an example from~\cite{Baier:PMC:2008}. A Beverage Vending Machine %, which has been established as standard in the field of process calculi,
 can be described as a Kripke structure $\Hm=(S, R, L, s_0)$ in Figure~\ref{BVM} on $V_a=\{select, pay, beer, soda\}$.
 % with $S=\{s_0, s_1, s_2\}$, $R=\{(s_0, s_1), (s_0, s_2), (s_1, s_0), (s_2,s_0)\}$, $L(s_0)=\{select\}$, $L(s_1)=\{pay, soda\}$, $L(s_2)=\{pay, beer\}$ and $s_0$ is an initial state.
This means that  if we are in $s_0$, (s)elect (so)da and pay for it then we are in $s_1$, else if we (s)elect (b)eer and (p)ay for it then we  are in $s_2$, after taking out the drink we are back to $s_0$ again.
For convenience, we use $s$ for $select$, $p$ for $pay$, $b$ for $beer$, $so$ for $soda$ and $r$ for orange juice.
%It follows that $s_0$ is an initial state.
Let $\varphi=\ALL\GLOBAL \ALL \FUTURE(p\wedge r)$, which means $p\wedge r$ will be satisfied infinite times in the structure, be a \CTL\ formula.
\begin{figure}
  \centering
  % Requires \usepackage{graphicx}
  \includegraphics[width=5cm]{BVM.png}\\
  \caption{A Beverage Vending Machine}\label{BVM}
\end{figure}
\end{example}

We can decide $(\Hm, s_0)\nvDash \varphi$ easily due to this structure do not contain the atom $r$.
In order for $(\Hm, s_0)$ satisfy $\varphi$, we should find a condition $\psi$ such that $(\Hm, s_0) \models \psi \supset \varphi$.
 As we know that if this condition exists, there are many conditions that satisfy the need.
In this case, if we are clever enough to judge in advance the set of possible atomic propositions that make up the condition, then we can find this condition in the set only, and the smaller the set, the easier it is to work out the condition.
In this paper, we always assume that the condition is a property defined on the specified atomic proposition set $V$, for our example $V=\{p,r\}$, and find the weakest property (that is, the weakest sufficient condition) satisfying the condition on the set. Finding this property is called discovering theorem by Lin in~\cite{DBLP:journals/aim/Lin18}.
Inspired by the forgetting-based method to compute SNC (WSC)~\cite{DBLP:journals/ai/Lin01}, in this paper, we tackle this problem by proposing a semantic forgetting for \CTL.

However, as we have said that $\Hm$ is a Kripke structure, which needs to be converted into a logical formula (theory), that is the characteristic formula, which is a \CTL\ formula proposed in~\cite{DBLP:journals/tcs/BrowneCG88}.
Thanks to we find the WSC in a set $V$ of atoms, hence a set-based bisimulation between two \MPK-structures (a Kripke structure with a state in it), $V$-bisimulation, and characteristic formula on $V$ will be proposed in this paper.
%Although the relation of bisimulation between transition systems has been wildly researched during the past several decades, it is different from our concept.
Our $V$-bisimulation is a more general bisimulation relation than others.
On the one hand, the above set-based bisimulation is an extension of the
bisimulation-equivalence of Definition~7.1 in \cite{Baier:PMC:2008} in the
sense that if $V=\cal A$ then our bisimulation is almost same to the
latter.
%\footnote{The latter has a given set of initial states,
%while there is only one initial state in our case.}.
On the other hand, the above set-based bisimulation notion is similar to
the state equivalence in \cite{DBLP:journals/tcs/BrowneCG88}. But it is
different in the sense that ours is defined on \MPK-structures,
while it is defined on states in \cite{DBLP:journals/tcs/BrowneCG88}.
What's more, the set-based bisimulation notion is also different
from  the state-based bisimulation notion of Definition~7.7 in \cite{Baier:PMC:2008},
which is defined for states of a given \MPK-structure.




As a logical notion, \emph{forgetting} was first formally defined
in propostional and first order logics by Lin and Reiter~\cite{lin1994forget}.
Over the last twenty years, researchers have developed forgetting notions and theories not only in classical logic but also in other non-classical logic systems~\cite{eiter2019brief}, such as forgetting in logic programs under answer set/stable model semantics~\cite{DBLP:Zhang:AIJ2006,Eiter2008Semantic,Wong:PhD:Thesis,Yisong:KR:2012,Yisong:IJCAI:2013}, forgetting in description logic~\cite{Wang:AMAI:2010,Lutz:IJCAI:2011,zhao2017role} and knowledge forgetting in modal logic~\cite{Yan:AIJ:2009,Kaile:JAIR:2009,Yongmei:IJCAI:2011,fang2019forgetting}. In application, forgetting has been used in planning~\cite{lin2003compiling},  conflict solving \cite{Lang2010Reasoning,Zhang2005Solving},
%knowledge compilation \cite{Zhang2009Knowledge,Bienvenu2010Knowledge},
createing restricted views of ontologies~\cite{zhao2017role},
%{ZhaoSchmidt18a},
strongest and weakest definitions \cite{Lang2008On}, SNC (WSC) \cite{DBLP:journals/ai/Lin01} and so on.


Though forgetting has been extensively investigated from various aspects of different logical systems.
However, the existing forgetting method in propositional
logic, answer set programming, description logic and modal logic are not directly applicable in \CTL.
For instance, in propositional forgetting theory, forgetting atom $q$ from $\varphi$ is equivalent to a formula $\varphi[q/\top] \vee \varphi[q/\perp]$, where $\varphi[q/X]$ is a formula obtained from $\varphi$ by replacing each $q$ with $X$ ($X\in \{\top, \perp\}$).
However, this method cannot be extended to a \CTL\ formula. Consider a \CTL\ formula $\psi=\ALL\GLOBAL p \wedge \neg \ALL\GLOBAL q \wedge \neg \ALL\GLOBAL \neg q$. If we want to forget
atom $q$ from $\psi$ by using the above method, we would have $\psi[q/\top] \vee \psi[q/\perp] \equiv \perp$. This is obviously not correct because after forgetting $q$ this specification should
not become inconsistent.
Similar with that in~\cite{Yan:AIJ:2009}, we research forgetting in \CTL\ from the semantic forgetting point of view.
And it is shown that our definition of forgetting satisfies those four postulates of forgetting.

The rest of the paper is organised as follows. Section 2 introduces the related notions for forgetting in \CTL, including the syntax and semantics of \CTL, the language we aimed for.
A formal definition of concept forgetting and its properties for \CTL\ follows in Section 3.
Section 4 explores the relation between forgetting and SNC (WSC).
From the point of view of model, we propose an algorithm for computing forgetting on \CTL\ in Section 5.
Finally, we conclude this paper.



 \section{Preliminaries}
 We start with some technical and notational preliminaries. Throughout this paper, we fix a finite set $\Ha$ of propositional variables (or atoms), and use $V$, $V'$ for subsets of $\Ha$. In this part, we will introduce the structure we will use for \CTL\ and syntax and semantic of \CTL.
\subsection{Model structure in \CTL}
In general, a transition system
%\footnote{According to \cite{Baier:PMC:2008},
%a {\em transition system} TS is a tuple $(S, Act,\rto,I, AP, L)$ where
%(1) $S$ is a set of states,
%(2) $\textrm{Act}$ is a set of actions,
%(3) $\rto\subseteq S\times \textrm{Act}\times S$ is a transition relation,
%(4) $I\subseteq S$ is a set of initial states,
%(5) $\textrm{AP}$ is a set of atomic propositions, and
%(6) $L:S\rto 2^{\textrm{AP}}$ is a labeling function.}
 can be described by a \emph{model\ structure} (or \emph{Kripke \ structure}) (see~\cite{Baier:PMC:2008} for details). A model structure is a triple $\Hm=(S,R,L)$, where
\begin{itemize}
  \item $S$ is a finite nonempty set of states,
  \item $R\subseteq S\times S$ and, for each $s\in S$, there
  is $s'\in S$ such that $(s,s')\in R$,
  \item $L$ is a labeling function $S\rto 2^{\cal A}$.
\end{itemize}
We call a model structure $\Hm$ on a set $V$ of atoms if $L: S \rto 2^V$, \ie, the labeling function $L$ map every state to $V$ (not the $\Ha$).  A \emph{path} $\pi_{s_i}$ start from $s_i$ of $\Hm$ is an infinite sequence of states $\pi_{s_i}=(s_i, s_{i+1} s_{i+2},\dots)$, where for each $j$ ($0\leq i\leq j$), $(s_j, s_{j+1}) \in R$. By $s'\in \pi_{s_i}$ we mean that $s'$ is a state in the path $\pi_{s_i}$.
A sate $s\in S$ is {\em initial} if for any state $s'\in S$, there is a path $\pi_s$ s.t $s'\in \pi_s$.
We denote this model structure as $(S,R,L,s_0)$, where $s_0$ is initial.

For a given model structure $(S,R,L,s_0)$ and $s\in S$,
the {\em computation tree}
$\Tr_n^{\cal M}(s)$ of $\cal M$(or simply $\Tr_n(s)$), that has depth $n$ and is rooted at $s$, is recursively defined as~\cite{DBLP:journals/tcs/BrowneCG88}, for $n\ge 0$,
\begin{itemize}
  \item $\Tr_0(s)$ consists of a single node $s$ with label $s$.
  \item $\Tr_{n+1}(s)$ has as its root a node $m$ with label  $s$, and
  if $(s,s')\in R$ then the node $m$ has a subtree $\Tr_n(s')$.
 % \footnote{Though
%  some nodes of the tree may have the same label, they are different nodes in the tree.}.
\end{itemize}
By $s_n$ we mean a $n$th level node of tree $\Tr_m(s)$ $(m \geq n)$.

A {\em \MPK-structure} (or {\em \MPK-interpretation}) is a model structure
${\cal M}=(S, R, L, s_0)$ associating
with a state $s\in S$, which is written as $({\cal M},s)$ for convenience in the following.
In the case $s=s_0$ is an initial state of $\cal M$, the \MPK-structure is {\em initial}.



\subsection{Syntax and semantics of \CTL}
In the following we briefly review the basic syntax and semantics
of the \CTL~\cite{DBLP:journals/toplas/ClarkeES86}.
The {\em signature} of the language $\cal L$ of \CTL\ includes:
\begin{itemize}
  \item a finite set of Boolean variables, called {\em atoms} of $\cal L$: $\cal A$;
  \item constant symbols: $\bot$ and $\top$;
  \item the classical connectives: $\lor$ and $\neg$;
  %\item the propositional constants: $\bot$;
  \item the path quantifiers: $\ALL$ and $\EXIST$;
  \item the temporal operators: \NEXT, \FUTURE, \GLOBAL\, \UNTIL\ and \UNLESS, that
  means `neXt state', `some Future state', `all future states (Globally)', `Until' and `Unless', respectively;
  \item parentheses: ( and ).
\end{itemize}

The {\em (existential normal form or ENF in short) formulas} of
$\cal L$ are inductively defined via a Backus Naur form:
\begin{equation}\label{def:CTL:formulas}
  \phi ::=  \bot \mid \top \mid p \mid\neg\phi \mid \phi\lor\phi \mid
    \EXIST \NEXT \phi \mid
    %\EXIST \FUTURE \phi \mid
    \EXIST \GLOBAL \phi \mid
    \EXIST [\phi\ \UNTIL\ \phi]%.% \mid
    %\ALL \NEXT \phi \mid
%    \ALL \FUTURE \phi \mid
%    \ALL \GLOBAL \phi \mid
%    \ALL [\phi\ \UNTIL\ \phi]
\end{equation}
where $p\in\cal A$. The formulas $\phi\land\psi$ and $\phi\rto\psi$
are defined in a standard manner of propositional logic.
The other form formulas of $\cal L$ are abbreviated
using the forms of (\ref{def:CTL:formulas}).
%Notice that, according to the
%above definition for formulas of \CTL,
%each of the \CTL\ {\em temporal connectives} has the form $XY$
%where $X\in \{\ALL,\EXIST\}$ and  $Y\in\{\NEXT, \FUTURE, \GLOBAL, \UNTIL\}$.
%The priorities for the \CTL\ connectives are assumed to be (from the highest to the lowest):
%\begin{equation*}
 % \neg, \EXIST\NEXT, \EXIST\FUTURE, \EXIST\GLOBAL, \ALL\NEXT, \ALL\FUTURE, \ALL\GLOBAL
 % \prec \land \prec \lor \prec \EXIST\UNTIL, \ALL\UNTIL, \EXIST \UNLESS, \ALL \UNLESS, \rto.
%\end{equation*}

We are now in the position to define the semantics of $\cal L$.
Let ${\cal M}=(S,R,L,s_0)$ be a model structure, $s\in S$ and $\phi$ a formula of $\cal L$.
The {\em satisfiability} relationship between $({\cal M},s)$ and $\phi$,
written $({\cal M},s)\models\phi$, is inductively defined on the structure of $\phi$ as follows:

\begin{itemize}
  \item $({\cal M},s)\not\models\bot$ \text{ and }  $({\cal M},s)\models\top$;
  \item $({\cal M},s)\models p$ iff $p\in L(s)$;
  \item $({\cal M},s)\models \phi_1\lor\phi_2$ iff
    $({\cal M},s)\models \phi_1$ or $({\cal M},s)\models \phi_2$;
  \item $({\cal M},s)\models \neg\phi$ iff  $({\cal M},s)\not\models\phi$;
  \item $({\cal M},s)\models \EXIST\NEXT\phi$ iff
    $({\cal M},s_1)\models\phi$ for some $s_1\in S$ and $(s,s_1)\in R$;
  \item $({\cal M},s)\models \EXIST\GLOBAL\phi$ iff
    $\cal M$ has a path $(s_1=s,s_2,\ldots)$ such that
    $({\cal M},s_i)\models\phi$ for each $i\ge 1$;
  \item $({\cal M},s)\models \EXIST[\phi_1\UNTIL\phi_2]$ iff
    $\cal M$ has a path $(s_1=s,s_2,\ldots)$ such that, for some $i\ge 1$,
    $({\cal M},s_i)\models\phi_2$ and
    $({\cal M},s_j)\models\phi_1$ for each $1\leq j<i$.
\end{itemize}

Similar to the work in \cite{DBLP:journals/tcs/BrowneCG88,Bolotov:1999:JETAI},
only initial \MPK-structures are considered to be candidate models
in the following, unless explicitly stated. Formally,
an initial \MPK-structure $\cal K$ is a {\em model} of a formula $\phi$
whenever ${\cal K}\models\phi$.
%Let $\Pi$ be a set of formulae, ${\cal K} \models \Pi$ if for each $\phi\in \Pi$ there is $\cal K \models \phi$.
We denote $\Mod(\phi)$ the set of models of $\phi$.
The formula $\phi$  is {\em satisfiable}
if $\Mod(\phi)\neq\emptyset$.
Since the states in model structure is finite, $\Mod(\phi)$
is finite for any formula $\phi$.

Let $\phi_1$ and $\phi_2$ be two formulas.
By $\phi_1\models\phi_2$ we denote $\Mod(\phi_1)\subseteq\Mod(\phi_2)$.
By $\phi_1\equiv\phi_2$ we mean $\phi_1\models\phi_2$ and $\phi_2\models\phi_1$.
In this case $\phi_1$ is {\em equivalent} to $\phi_2$.
By $\Var(\phi_1)$ we mean the set of atoms occurring in $\phi_1$.
 $\phi_1$ is $V$-{\em irrelevant}, written $\IR(\phi_1,V)$,
if there is a formula $\psi$ with
$\Var(\psi)\cap V=\emptyset$ such that $\phi_1\equiv\psi$.





\section{Forgetting in \CTL}
In this section, we will define the forgetting in \CTL\ by $V$-bisimulation, set-based bisimulations.
Besides, some properties of forgetting are also explored.
For convenience, let $\Hm=(S, R, L, s_0)$, $\Hm'=(S',R',L',s_0')$ and ${\cal K}_i=(\Hm_i, s_i)$ with $\Hm_i=(S_i, R_i,L_i, s_0^i)$, $s_i \in S_i$ and $i$ is an integer.
\subsection{Set-based bisimulation}
To present a formal definition of forgetting, we need the concept of $V$-bisimulation.
Inspired by the notion of bisimulation in~\cite{DBLP:journals/tcs/BrowneCG88}, we define the relations $\Hb_0,\Hb_1,\ldots$
between \MPK-structures on $V$ as follows: let
${\cal K}_i=({\cal M}_i,s_i)$ with $i\in\{1,2\}$,
\begin{itemize}
  \item $({\cal K}_1,{\cal K}_2)\in\Hb_0$ if $L_1(s_1)- V=L_2(s_2)- V$;  % and ${\cal K}'=(\tuple{S', R',L'},s')$;
  \item for $n\ge 0$, $({\cal K}_1,{\cal K}_2)\in\Hb_{n+1}$ if
  \begin{itemize}
    \item $({\cal K}_1,{\cal K}_2)\in\Hb_0$,
    \item for every $(s_1,s_1')\in R_1$, there is $(s_2,s_2')\in R_2$
    such that $({\cal K}_1',{\cal K}_2')\in \Hb_n$, and
    \item for every $(s_2,s_2')\in R_2$, there is $(s_1,s_1')\in R_1$
    such that $({\cal K}_1',{\cal K}_2')\in \Hb_n$,
  \end{itemize}
  where ${\cal K}_i'=({\cal M}_i,s_i')$ with $i\in\{1,2\}$.
\end{itemize}

Now, we define the notion of $V$-bisimulation between \MPK-structures:
\begin{definition}[$V$-bisimulation]
  \label{def:V-bisimulation}
   Let $V\subseteq\cal A$. Given   two \MPK-structures ${\cal K}_1$ and ${\cal K}_2$ are $V$-{\em bisimilar},  denoted ${\cal K}_1 \lrto_V {\cal K}_2$
 if and only if $ ({\cal K}_1,{\cal K}_2)\in {\Hb_i}\mbox{ for all }i\ge 0.$ Moreover, two paths $\pi_i=(s_{i,1},s_{i,2},\ldots)$ of $\Hm_i$ with $i\in \{1,2\}$
 are $V$-{\em bisimilar} if
$ {\cal K}_{1,j} \lrto_V {\cal K}_{2,j}\mbox { for every $j\ge 1$ }$
 where ${\cal K}_{i,j}=(\Hm_i,s_{i,j})$.

\end{definition}

%\begin{proposition}\label{Vbi:Equ}
%Let $V\subseteq\cal A$
%%${\cal M}_i=(S_i,R_i,L_i,s_0^i)~(i=1,2)$ be model structures
%and ${\cal K}_i=({\cal M}_i,s_i)~(i=1,2)$ be \MPK-structures.
%Then $({\cal K}_1,{\cal K}_2)\in\cal B$ if and only if
%  \begin{enumerate}[(i)]
%    \item $L_1(s_1)- V = L_2(s_2)- V$,
%    \item for every $(s_1,s_1')\in R_1$, there is $(s_2,s_2')\in R_2$
%    such that $({\cal K}_1',{\cal K}_2')\in \Hb$, and
%    \item for every $(s_2,s_2')\in R_2$, there is $(s_1,s_1')\in R_1$
%    such that $({\cal K}_1',{\cal K}_2')\in \Hb$,
%   \end{enumerate}
% where ${\cal K}_i'=({\cal M}_i,s_i')$ with $i\in\{1,2\}$.
%\end{proposition}


It's apparent that $\lrto_V$ is a binary relation.
 In the sequel, we abbreviate ${\cal K}_1 \lrto_V {\cal K}_2$
 by $s_1 \lrto_V s_2 $
 whenever the underlying model structures of states $s_1$ and $s_2$ are clear from the context.% when it is clear
  %from its context.
 % The next lemma easily follows from the above definition,
\begin{lemma}\label{lem:equive}
  The relation $\lrto_V$ is an equivalence relation.
\end{lemma}

Besides, we have the following properties:
\begin{proposition}\label{div}
Let $i\in \{1,2\}$, $V_1,V_2\subseteq\cal A$, $s_i'$s be two states and
  $\pi_i'$s be two pathes,
and ${\cal K}_i=({\cal M}_i,s_i)~(i=1,2,3)$ be \MPK-structures
 such that
${\cal K}_1\lrto_{V_1}{\cal K}_2$ and ${\cal K}_2\lrto_{V_2}{\cal K}_3$.
 Then:
 \begin{enumerate}[(i)]
   \item $s_1'\lrto_{V_i}s_2'~(i=1,2)$ implies $s_1'\lrto_{V_1\cup V_2}s_2'$;
   \item $\pi_1'\lrto_{V_i}\pi_2'~(i=1,2)$ implies $\pi_1'\lrto_{V_1\cup V_2}\pi_2'$;
   \item for each path $\pi_{s_1}$ of $\Hm_1$ there is a path $\pi_{s_2}$  of $\Hm_2$ such that $\pi_{s_1} \lrto_{V_1} \pi_{s_2}$, and vice versa;
   \item ${\cal K}_1\lrto_{V_1\cup V_2}{\cal K}_3$;
   \item If $V_1 \subseteq V_2$ then ${\cal K}_1 \lrto_{V_2} {\cal K}_2$.
 \end{enumerate}
\end{proposition}








Intuitively, if two \MPK-structures are $V$-bisimilar, then they satisfy the same formula $\varphi$ that dose not contain any atoms in $V$, \ie $\IR(\varphi, V)$.
\begin{theorem}\label{thm:V-bisimulation:EQ}
  Let $V\subseteq\cal A$, ${\cal K}_i~(i=1,2)$ be two \MPK-structures such that
  ${\cal K}_1\lrto_V{\cal K}_2$ and $\phi$ a formula with $\IR(\phi,V)$. Then
  ${\cal K}_1\models\phi$ if and only if ${\cal K}_2\models\phi$.
\end{theorem}


Let $V\subseteq\cal A$, ${\cal M}_i~(i=1,2)$ be  model structures.
A computation tree $\Tr_n(s_1)$ of ${\cal M}_1$ is $V$-{\em bisimilar}
to a computation tree $\Tr_n(s_2)$ of ${\cal M}_2$, written
$({\cal M}_1,\Tr_n(s_1))\lrto_V({\cal M}_2,\Tr_n(s_2))$ (or simply
$\Tr_n(s_1)\lrto_V\Tr_n(s_2)$), if % $({\cal M}_1,s_1)\lrto_V({\cal M}_2,s_2)$.
\begin{itemize}
  \item $L_1(s_1)- V=L_2(s_2)- V$,
  \item for every subtree $\Tr_{n-1}(s_1')$ of $\Tr_n(s_1)$,
  $\Tr_n(s_2)$ has a subtree $\Tr_{n-1}(s_2')$ such that
  $\Tr_{n-1}(s_1')\lrto_V\Tr_{n-1}(s_2')$, and vice verse.
  %\item for every subtree $\Tr_{n-1}(s_2')$ of $\Tr_n(s_2)$,
%  $\Tr_n(s_1)$ has a subtree $\Tr_{n-1}(s_1')$ such that
%  $\Tr_{n-1}(s_1')\lrto_V\Tr_{n-1}(s_2')$.
\end{itemize}
Note that the last condition in the above definition
hold trivially for $n=0$.

\begin{proposition}\label{B_to_T}
  Let $V\subseteq\cal A$ and $({\cal M}_i,s_i)~(i=1,2)$ be two \MPK-structures.
  Then
  \[(s_1,s_2)\in{\cal B}_n\mbox{ iff }
  \Tr_j(s_1)\lrto_V\Tr_j(s_2)\mbox{ for every $0\le j\le n$}.\]
\end{proposition}
This means that $\Tr_j(s_1) \lrto_V \Tr_j(s_2)$ for all $j \geq 0$ if $s_1 \lrto_V s_2$, otherwise there is some $k$ such that $\Tr_k(s_1)$ and $\Tr_k(s_2)$ are not $V$-bisimilar.

\begin{proposition}\label{pro:k}
  Let $V\subseteq \Ha$, $\Hm$ be a model structure and $s,s'\in S$
  such that $s\not\lrto_V s'$.
  There exists a least  $k$ such that
  $\Tr_k(s)$ and $\Tr_k(s')$ are not $V$-bisimilar.
\end{proposition}
In this case the  model structure ${\cal M}$ is called $V$-{\em distinguishable} (by
states $s$ and $s'$ at the least depth $k$), which is denoted by $\dis_V({\cal M},s,s',k)$.
It is evident that
$\dis_V({\cal M},s,s',k)$ implies $\dis_V({\cal M},s,s',k')$ whenever $k'\ge k$.
The $V$-{\em characterization number}
of ${\cal M}$, written $ch({\cal M},V)$, is defined as
\[ch({\cal M},V)=
\left\{
  \begin{array}{ll}
    \max\{k\mid s,s'\in S\ \&\ \dis_V({\cal M},s,s',k)\},\\
         \ \ \qquad \qquad \qquad \hbox{${\cal M}$ is $V$-distinguishable;} \\
    \min\{k\mid {\cal B}_{k}={\cal B}_{k+1}\}, \ \ \ \quad \qquad \hbox{otherwise.}
  \end{array}
\right.
\]



%Intuitively, forgetting an atom results in a weaker theory which entails the same set of formulae that are irrelevant to the atom.
%To present the representation property of forgetting in \CTL\ and compute WSC (SNC) under an initial \MPK-structure, we will give the characterizing formula of an initial \MPK-structure on $V$ in the next subsection.
%Before define the concept of forgetting in \CTL\, we will give the characterizing formula of an initial \MPK-structure on $V$ in the next subsection.

\subsection{Characterization of Initial \MPK-structure}

In order to introduce our notion of forgetting, and to compute strongest necessary and weakest sufficient conditions, we need a formula that captures the  initial \MPK-structure on $V$ syntactically. We call such formula as characterizing formula.
In the following, we present such characterization.

Given a set $V\subseteq\Ha$, we define a formula $\varphi$ of $V$ (that is $\Var(\varphi) \subseteq V$) in \CTL\ that describes a computation tree.
\begin{definition}\label{def:V:char:formula}
Let $V\subseteq \Ha$, $\Hm =(S,R,L,s_0)$ be a model structure and $s\in S$.
The {\em characterizing formula} of the computation tree $\Tr_n(s)$ on $V$,
written ${\cal F}_V(\Tr_n(s))$, is defined recursively as:
\begin{align*}
   {\cal F}_V(\Tr_0(s)) &=  \bigwedge_{p \in V\cap L(s)}p
     \wedge \bigwedge_{q\in V-L(s)} \neg q,\\
   {\cal F}_V(\Tr_{k+1}(s))& = \bigwedge_{(s,s')\in R}
    \EXIST \NEXT {\cal F}_V(\Tr_k(s'))\\
  &\wedge
    \ALL \NEXT \left( \bigvee_{(s,s')\in R} {\cal F}_V(\Tr_k(s')) \right) \wedge {\cal F}_V(\Tr_0(s))
\end{align*}
for $k\ge 0$.
\end{definition}
The characterizing formula of a computation tree formally exhibit the context of each node on $V$ (atoms are true at this node if they are in V, else false) and the temporal relation between states recursively. In this way, we know:

\begin{lemma}\label{lem:Vb:TrFormula:Equ}
Let $V\subseteq \Ha$, $\Hm$ and $\Hm'$ be two model structures,
$s\in S$, $s'\in S'$ and $n\ge 0$. If $\Tr_n(s) \lrto_{\overline V} \Tr_n(s')$, then ${\cal F}_V(\Tr_n(s)) \equiv {\cal F}_V(\Tr_n(s'))$.
\end{lemma}
Let $s'=s$, it shows that for any formula $\varphi$ of $V$, if $\varphi$ is a characterizing formula of $\Tr_n(s)$ then $\varphi \equiv {\cal F}_V(\Tr_n(s))$.



Let $V\subseteq\cal A$,
%, ${\cal M}=(S,R,L,s_0)$
 ${\cal K}=({\cal M},s_0)$ be an initial \MPK-structure and $T(s') = {\cal F}_V(\Tr_c(s'))$.
The {\em characterizing formula} of $\cal K$ on $V$, written ${\cal F}_V(\Hm,s_0)$ (or ${\cal F}_V({\cal K})$), is
defined as the conjunction of the following formulas:
\begin{align*}
  &{\cal F}_V(\Tr_c(s_0)), \mbox{ and }\\
  & \bigwedge_{s\in S}\ALL \GLOBAL\left(
    {\cal F}_V(\Tr_c(s)) \rto
    \bigwedge_{(s,s')\in R}
        \EXIST \NEXT T(s')
        \wedge
        \ALL \NEXT \bigvee_{(s,s')\in R}T(s')
    \right)
\end{align*}
%\begin{equation*}
%\resizebox{.91\linewidth}{!}{$
%    \displaystyle
%   \ALL \GLOBAL\left(
%    {\cal F}_V(\Tr_c(s)) \rto
%    \bigwedge_{(s,s')\in R}
%        \EXIST \NEXT T(s')
%        \wedge
%        \ALL \NEXT \bigvee_{(s,s')\in R}T(s')
%    \right)
%$}
%\end{equation*}
where $c=ch({\cal M},V)$. It is apparent that $\IR({\cal F}_V(\Hm, s_0), \overline V)$.

The following example show how to compute characterizing formula:
\begin{example}
%Let ${\cal K} = (\Hm, s_0)$ with $\Hm=(S, R, L,s_0)$ be a initial \MPK-structure, % (in Fig.~\ref{Kripke_1}),
% in which $S=\{s_0, s_1, s_2\}$, $R=\{(s_0, s_1), (s_0, s_2), (s_1, s_0), (s_2, s_0)\}$, $L(s_0)= \{a\}$, $L(s_1) =\{a,c\}$ and $L(s_2) = \{b,c\}$. Let $V=\{a, b\}$, compute the characterizing formula of ${\cal K}$ on $V$.
Let ${\cal K} = (\Hm, s_0)$ in Figure~\ref{Kripke_1} be an initial \MPK-structure and $V=\{a, b\}$, then compute the characterizing formula of ${\cal K}$ on $V$.

It is apparent that $\Tr_0(s_0) \lrto_{\overline V} \Tr_0(s_1)$ due to $L(s_0) - \overline V = L(s_1) - \overline V$, $\Tr_1(s_0) \not\lrto_{\overline V} \Tr_1(s_1)$ due to there is $(s_0, s_2)\in R$ such that for any $(s_1, s') \in R$ (there is only one immediate successor $s'=s_0$) there is $L(s_2)- \overline V \neq L(s') - \overline V$. Hence, we have that $\Hm$ is $\overline V$-distinguished by state $s_0$ and $s_1$ at the least depth 1, \ie $\dis_{\overline V}(\Hm, s_0, s_1, 1)$. Similarly, we have $\dis_{\overline V}(\Hm, s_0, s_2, 0)$ and $\dis_{\overline V}(\Hm, s_1, s_2, 0)$. Therefore, $ch(\Hm, \overline V) =  \max\{k\mid s,s'\in S\ \&\ \dis_{\overline V}({\cal M},s,s',k)\} = 1$.
%It is apparent that $\Hb_0=\{(s_0, s_1)\}$ and $\Hb_1={\O}$, then $ch(\Hm, s_0) = 1$.
Then we have:
\begin{align*}
  & {\cal F}_V(\Tr_0(s_0)) = a \wedge \neg b, \\
  & {\cal F}_V(\Tr_0(s_1)) = a \wedge \neg b, \\
  & {\cal F}_V(\Tr_0(s_2)) = b \wedge \neg a, \\
  & {\cal F}_V(\Tr_1(s_0)) = \EXIST\NEXT(a \wedge \neg b)  \wedge \EXIST\NEXT(b \wedge \neg a) \wedge \ALL\NEXT((a \wedge \neg b) \vee\\
  & \qquad \qquad  \qquad (b \wedge \neg a)) \wedge (a \wedge \neg b), \\
  & {\cal F}_V(\Tr_1(s_1)) = \EXIST\NEXT(a \wedge \neg b)  \wedge \ALL\NEXT(a \wedge \neg b) \wedge (a \wedge \neg b), \\
  & {\cal F}_V(\Tr_1(s_2)) = \EXIST\NEXT(a \wedge \neg b)  \wedge \ALL\NEXT(a \wedge \neg b) \wedge (b \wedge \neg a).\\
  & \mbox{ Then it is easy to obtain\ } {\cal F}_V(\Hm, s_0).
  %&{\cal F}_V(\Hm, s_0)= {\cal F}_V(\Tr_1(s_0)) \wedge  \bigwedge_{s\in S} \ALL \GLOBAL \left(
%  {\cal F}_V(\Tr_1(s)) \rto
%  \bigwedge_{(s,s')\in R}
%        \EXIST \NEXT {\cal F}_V(\Tr_1(s'))
%        \wedge
%        \ALL \NEXT \bigvee_{(s,s')\in R}{\cal F}_V(\Tr_1(s'))
%  \right)
\end{align*}

\begin{figure}
  \centering
  % Requires \usepackage{graphicx}
  \includegraphics[width=5cm]{k1.png}\\
  \caption{A simple Kripke structure}\label{Kripke_1}
\end{figure}
\end{example}

By the following theorem we also have that given a set $V\subseteq \Ha$, the characterizing formula of an initial \MPK-structure is equivalent uniquely describe this initial \MPK-structure on $V$.
\begin{theorem}\label{CF}
Given $V\subseteq \Ha$, let $\Hm=(S,R,L,s_0)$  and $\Hm'=(S',R', L',s_0')$ be two model structures. Then,
\begin{enumerate}[(i)]
 \item $(\Hm',s_0') \models {\cal F}_V({\cal M},s_0)
\text{ iff }
({\cal M},s_0) \lrto_{\overline V} ({\cal M}',s_0')$;

\item $s_0 \lrto_{\overline V} s_0'$ implies  ${\cal F}_V(\Hm, s_0) \equiv {\cal F}_V(\Hm', s_0')$.
\end{enumerate}

\end{theorem}

Moreover, we know that any initial \MPK-structure can be described as a \CTL\ formula from the definition of characterizing formula. Then,
\begin{lemma}\label{lem:models:formula}
  Let $\varphi$ be a formula. We have
  \begin{equation}
    \varphi\equiv \bigvee_{(\Hm, s_0)\in\Mod(\varphi)}{\cal F}_{\cal A}(\Hm, s_0).
\end{equation}
\end{lemma}
It follows that any \CTL\ formula can be described by the disjunction of the characterizing formulas of all the models of itself due to the number of models of a \CTL\ formula is finite.




%\begin{proof}
%This is following Lemma~\ref{lem:Vb:TrFormula:Equ} and the definition of the characterizing formula of initial \MPK-structure ${\cal K}$ on $V$.
%\end{proof}



\subsection{Semantic Properties of Forgetting in \CTL}
In this subsection we will give the definition of forgetting in \CTL\ and study it's semantic properties.
 We will first show via a representation theorem that our definition of forgetting correspond to the readily existing notion of forgetting which is characterised by several desirable properties (also called postulates) suggested in~\cite{Yan:AIJ:2009}. Next, we discuss various additional semantic properties of forgetting.

Now, we give the formal definition of forgetting in \CTL\ from the semantic point view.
\begin{definition}[Forgetting]\label{def:V:forgetting}
  Let $V\subseteq\cal A$ and $\phi$ a formula.
A formula $\psi$ with $\Var(\psi)\cap V=\emptyset$
is a {\em result of forgetting $V$ from} $\phi$, if
\begin{equation}
\resizebox{.91\linewidth}{!}{$
\displaystyle
  \Mod(\psi)=\{{\cal K}\mbox{ is initial}\mid \exists {\cal K}'\in\Mod(\phi)\ \&\ {\cal K}'\lrto_V{\cal K}\}.
  $}
\end{equation}
\end{definition}
Note that if both $\psi$ and $\psi'$ are results of forgetting $V$ from $\phi$ then
$\Mod(\psi)=\Mod(\psi')$, \ie, $\psi$ and $\psi'$ have the same models. In the sense
of equivalence the forgetting result is unique (up to equivalence).
By Lemma~\ref{lem:models:formula},such a formula always exists, which
is equivalent to
\begin{equation*}
  \bigvee_{{\cal K}\in  \{{\cal K}'\mid \exists {\cal K}''\in\Mod(\phi)\ \text{ and }\ {\cal K}''\lrto_V{\cal K}'\}} {\cal F}_{\overline V}({\cal K}).
\end{equation*}
For this reason, the forgetting result is denoted by $\CTLforget(\phi,V)$.

%By the definition of forgetting, we have

%\begin{proposition}\label{pro:IR_V:forget}
%Let $\varphi$ be a CTL formula and $V$ a set of atoms. If $V \cap \Var(\varphi) = {\O}$, then
%\[
%\CTLforget(\varphi,V) \equiv \varphi.
%\]
%\end{proposition}

%In the case $\psi$ is a result of forgetting $V$ from $\phi$, there are usually some
%expected properties (called {\em postulates}: (\W), (\PP), (\NgP) and (\textbf{IR})) for it~\cite{Yan:AIJ:2009}.
Assume you are given a formula $\varphi$, and $\varphi'$ is the formula after forgetting $V$, then we have the following desired properties, also called {\em postulates} of forgetting~\cite{Yan:AIJ:2009}.
\begin{itemize}
  \item Weakening (\W): $\varphi \models \varphi'$;
  \item Positive Persistence (\PP):
  Given $\eta \in \CTL$ if $\IR(\eta, V)$ and $\varphi \models \eta$, then $\varphi' \models \eta$;
  \item Negative Persistence (\NgP):   Given $\eta \in \CTL$  if $\IR(\eta, V)$ and $\varphi \nvDash \eta$, then $\varphi' \nvDash \eta$;
  \item Irrelevance (\textbf{IR}): $\IR(\varphi', V)$.
\end{itemize}

%lets explain them here

Intuitive enough, the postulate (\W) says, forgetting weakens the original formula.  (\PP)  and  (\NgP) correspond to the fact that so long as forgotten atoms $V$ are irrelevant to the remaining positive and the negative information, respectively, they do not affect them. (\textbf{\IR}) states that forgotten atoms $V$ are not relevant for the final formula anymore (i.e., $\varphi'$ is $V$-irrelevant).



\begin{theorem}[Representation Theorem]\label{thm:close}
Let $\varphi$, $\varphi'$ and $\phi$ be \CTL\ formulas and $V \subseteq \Ha$.
Then the following statements are equivalent:
\begin{enumerate}[(i)]
  \item $\varphi' \equiv \CTLforget(\varphi, V)$,
  \item $\varphi'\equiv \{\phi | \varphi \models \phi \text{ and } \IR(\phi, V)\}$,
  \item Postulates (\W), (\PP), (\NgP) and (\textbf{IR}) hold.
\end{enumerate}
\end{theorem}
The above theorem says that \CTL\ is closed under our definition of forgetting, \ie, for any CTL formula the result of forgetting is also a CTL formula,  and captures and entailed by the four postulates that forgetting should satisfy.

\begin{lemma}\label{lem:KF:eq}
	Let $\varphi$ and $\alpha$ be two \CTL\ formulae and $q\in
		\overline{\Var(\varphi) \cup \Var(\alpha)}$. Then
	$\forget(\varphi \cup\{q\lrto\alpha\}, q)\equiv \varphi$.
\end{lemma}


\begin{proposition}[Commutativity]\label{disTF}
Let $\varphi$ be a formula, $V$ a set of atoms and $p$ an atom such that $p \notin V$. Then,
\[
\CTLforget(\varphi, \{p\} \cup V) \equiv \CTLforget(\CTLforget(\varphi, p), V).
\]
\end{proposition}
This means that the result of forgetting $V$ from $\varphi$ can be obtained by forgetting atoms in $V$ one by one.
Moreover, the order of atoms does not matter (commutativity), which follows from  Proposition~\ref{disTF}.

\begin{corollary}\label{disTFV}
Let $\varphi$ be a formula and $V_i\subseteq{\cal A}~(i=1,2)$. Then:
\[
\CTLforget(\varphi, V_1 \cup V_2) \equiv \CTLforget(\CTLforget(\varphi, V_1), V_2).
\]
\end{corollary}


The following results, which are satisfied in both classical proposition logic and modal logic \SFive~\cite{Yan:AIJ:2009}, further illustrate other essential semantic properties of forgetting.
\begin{proposition}\label{pro:ctl:forget:1}
Let $\varphi$, $\varphi_i$, $\psi_i$ ($i=1,2$) be formulas and $V\subseteq \Ha$. We have
\begin{enumerate}[(i)]
  \item $\CTLforget(\varphi, V)$ is satisfiable iff $\varphi$ is;
  \item If $\varphi_1 \equiv \varphi_2$, then $\CTLforget(\varphi_1, V) \equiv \CTLforget(\varphi_2, V)$;
  \item If $\varphi_1 \models \varphi_2$, then $\CTLforget(\varphi_1, V) \models \CTLforget(\varphi_2, V)$;
  \item $\CTLforget(\psi_1 \vee \psi_2, V) \equiv \CTLforget(\psi_1, V) \vee \CTLforget(\psi_2, V)$;
  \item $\CTLforget(\psi_1 \wedge \psi_2, V) \models \CTLforget(\psi_1, V) \wedge \CTLforget(\psi_2, V)$;
 % \item If $\IR(\psi_1, V)$, then $\CTLforget(\varphi \wedge \psi_1, V) \equiv \CTLforget(\varphi, V) \wedge \psi_1$.
\end{enumerate}
\end{proposition}


Another interesting result is that the forgetting of $P T \varphi$ ($P\in \{\EXIST, \ALL\}$, $T \in \{\FUTURE, \NEXT\}$) on $V\subseteq \Ha$ can be computed by $PT \CTLforget(\varphi, V)$. This gives us a convenient method to compute forgetting since we can push the forgetting operator to a subformula without affecting the semantics.
\begin{proposition}[Linearity]\label{pro:ctl:forget:2}
  Let $V\subseteq\cal A$ and $\phi \in \CTL$,% and $Q\in \{\EXIST, \ALL\}$.
  \begin{enumerate}[(i)]
    \item $\CTLforget(\ALL\NEXT\phi,V)\equiv \ALL\NEXT \CTLforget(\phi,V)$.
    \item $\CTLforget(\EXIST\NEXT\phi,V)\equiv\EXIST\NEXT \CTLforget(\phi,V)$.
    \item $\CTLforget(\ALL \FUTURE\phi,V)\equiv \ALL \FUTURE \CTLforget(\phi,V)$.
    \item $\CTLforget(\EXIST\FUTURE\phi,V)\equiv\EXIST\FUTURE \CTLforget(\phi,V)$.
  \end{enumerate}
\end{proposition}



\subsection{Complexity Results}
In the following, we outline the computational complexity of the various tasks regarding the forgetting in \CTL\ and its popular fragment $\CTL_{\ALL\FUTURE}$.  It turns out that the model-checking on forgetting without any restriction is NP-complete.
%In this part we talk about the main complexity of entailment complexity of forgetting in \CTL.
\begin{proposition}[Model Checking on Forgetting]\label{modelChecking}
Let $(\Hm,s_0)$ be an initial \MPK-structure, $\varphi$ be a \CTL\ formula and $V$ a set of atoms. Deciding whether $(\Hm,s_0)$ is a model of $\forget(\varphi, V )$ is NP-complete.
\end{proposition}
%\begin{proof}
%The problem can be determined by the following two things: (1) guessing
%an I-structure $\Hm',s_0'$ satisfying $\varphi$; and
%(2) checking if  $(\Hm, s_0) \leftrightarrow_V (\Hm', s_0')$. Both two steps can be done in polynomial time.
% Hence, the problem is in NP.
%The hardness follows that the model checking for propositional variable
%forgetting is NP-hard.
%\end{proof}

The fragment of \CTL, in which each formula contains only $\ALL \FUTURE$ temporal connective correspond to specification descriptions for properties that is desired to hold in all branches eventually. Such properties are of special interest in concurrent systems e.g., mutual exclusion and  waiting events~\cite{Baier:PMC:2008}.
In the following, we report various complexity results concerning forgetting and the logical entailment in this fragment.

\begin{theorem}[Entailment on Forgetting]\label{thm:comF}
Let $\varphi$ and $\psi$ be two $\CTL_{\ALL \FUTURE}$ formulas and $V$ a set of atoms. Then,
results:
\begin{enumerate}[(i)]
  \item deciding  $\forget(\varphi, V ) \models^? \psi$ is co-NP-complete,
  \item deciding  $\psi \models^? \forget(\varphi, V)$ is $\Pi_2^P$-complete,
  \item deciding $\forget(\varphi, V) \models^? \forget(\psi, V)$ is $\Pi_2^P$-complete.
\end{enumerate}
%Where $X\in\{\ALL, \EXIST\}$.
\end{theorem}
%\begin{proof}
%(1) It is proved that deciding whether $\psi$ is satisfiable is NP-Complete~\cite{DBLP:journals/ijfcs/MeierTVM15}. The hardness is easy to see by setting $\forget(\varphi, \Var(\varphi))$, \ie deciding whether $\psi$ is valid.
%For membership, from Theorem
%3, we have $\forget(\varphi, V ) \models \psi$ iff $\varphi \models \psi$ and $IR(\psi, V )$.
%Clearly, in CTL($\ALL \FUTURE$), deciding $\varphi\models \psi$ is in co-NP. We show that deciding whether $IR(\psi, V )$ is also
%in co-NP. Without loss of generality, we assume that $\psi$ is satisfiable.
% %Then $\psi$ has a model in the polynomial size of $\psi$.
% We consider the complement of the problem: deciding whether $\psi$ is not irrelevant to $V$. It is easy to see that $\psi$ is
%not irrelevant to $V$ iff there exist a model $\Hm, s_0$ of $\psi$ and an
%initial \MPK-structure $\Hm',s_0'$  such that
%$\Hm, s_0 \leftrightarrow_V \Hm',s_0'$ and $\Hm',s_0'\nvDash \psi$. So checking whether $\psi$ is not irrelevant to $V$ can be achieved in the following steps: (1) guess two initial \MPK-structures $\Hm,s_0$ and $\Hm',s_0'$, (2) check if $\Hm,s_0 \models \psi$ and $\Hm',s_0'\nvDash \psi$, and (3) check
%$\Hm, s_0 \leftrightarrow_V \Hm',s_0'$. Obviously (1) can be done in polynomial time
%with a non-deterministic Turing machine while (2) and (3) can be done in polynomial time.
%
%(2) Membership. We consider the complement of the
%problem. We may guess an initial \MPK-structure $\Hm, s_0$ and check whether $\Hm,s_0 \models \psi$ and $\Hm,s_0$ $\nvDash \forget($ $\varphi$, $V)$. From Proposition~\ref{modelChecking}, we know that this is in $\Sigma_2^P$. So the original problem is in $\Pi_2^P$. Hardness. Let $\psi \equiv \top$. Then the problem is reduced to decide $\forget(\varphi, V )$'s validity. Since a propositional variable forgetting is a special case temporal forgetting, the hardness is directly followed from the proof of Proposition 24 in~\cite{DBLP:journals/jair/LangLM03}.
%
%(3) Membership. If $\forget(\varphi, V) \nvDash \forget(\psi, V)$ then there exist an initial \MPK-structure $\Hm, s$ such that $\Hm, s\models \forget(\varphi, V)$ but $\Hm, s \nvDash \forget(\psi, V)$, \ie, there is $\Hm_1, s_1 \lrto_V \Hm, s$ with $\Hm_1, s_1 \models \varphi$ but $\Hm_2, s_2 \nvDash \psi$ for every $\Hm_2, s_2$ with $\Hm, s \lrto_V \Hm_2, s_2$. It is evident that guessing such $\Hm, s$, $\Hm_1, s_1$ with $\Hm_1, s_1 \lrto_V \Hm, s$ and checking $\Hm_1, s_1\models \varphi$ are feasible while checking $\Hm_2, s_2 \nvDash \psi$ for every $\Hm, s \lrto_V \Hm_2, s_2$ can be done in polynomial time by call a nondeterministic Turing machine. Thus the problem is in $\Pi_2^P$.
%
%Hardness. It follows from (2) due to the fact that $\forget(\varphi, V) \models \forget(\psi, V)$ iff $\varphi \models \forget(\psi, V)$ thanks to $IR(\forget(\psi, V), V)$.
%
%\end{proof}

The following results follow from Theorem~\ref{thm:comF} and extends them to semantic equivalence.
\begin{corollary}
Let $\varphi$ and $\psi$ be two $\CTL_{\ALL \FUTURE}$ formulas and $V$ a set of atoms. Then
\begin{enumerate}[(i)]
  \item deciding $\psi \equiv^? \forget(\varphi, V)$ is $\Pi_2^P$-complete,
  \item deciding $\forget(\varphi, V) \equiv^? \varphi$ is co-NP-complete,
  \item deciding $\forget(\varphi, V) \equiv^? \forget(\psi, V)$ is $\Pi_2^P$-complete.
\end{enumerate}
\end{corollary}

\section{Strongest Necessary and Weakest Sufficient Conditions}
In this section, we will give the definition of SNC (WSC) and show that the SNC (WSC) of a specification (a \CTL\ formula) under a given initial \MPK-structure and set $V$ of atoms can be obtained from forgetting in \CTL.
The SNC (WSC) of a proposition will be given at first:
\begin{definition}[sufficient and necessary condition]\label{def:NC:SC}
Let $\phi$ be a formula or an initial \MPK-structure, $\psi$ be a formula, $V \subseteq \Var(\phi)$, $q\in\Var(\phi)- V$
and $\Var(\psi)\subseteq V$.
\begin{itemize}
  \item $\psi$  is a {\em necessary condition} (NC in short) of $q$ on $V$ under $\phi$
    if $\phi \models q \rto \psi$.
  \item $\psi$  is a {\em sufficient condition} (SC in short) of $q$ on $V$ under $\phi$
    if $\phi \models \psi\rto q$.
  \item $\psi$  is a {\em strongest necessary condition} (SNC in short)
  of $q$ on $V$ under $\phi$
    if it is a NC of $q$ on $V$ under $\phi$ and $\phi\models\psi\rto\psi'$
    for any NC $\psi'$ of $q$ on $V$ under $\phi$.

    \item $\psi$  is a {\em weakest sufficient condition} (SNC in short)
  of $q$ on $V$ under $\phi$
    if it is a SC of $q$ on $V$ under $\phi$ and $\phi\models\psi'\rto\psi$
    for any SC $\psi'$ of $q$ on $V$ under $\phi$.
\end{itemize}
\end{definition}
Note that if both $\psi$ and $\psi'$ are SNC (WSC) of $q$ on $V$ under $\phi$ then
$\Mod(\psi)=\Mod(\psi')$, \ie $\psi$ and $\psi'$ have the same models. In the sense
of equivalence the SNC (WSC) is unique (up to equivalence).



\begin{proposition}\label{dual}
(\textbf{dual})
 Let $V,q,\varphi$ and $\psi$ are the ones in Definition~\ref{def:NC:SC}.
 The $\psi$ is a SNC (WSC) of $q$ on $V$ under $\varphi$ iff $\neg \psi$ is a WSC (SNC)
    of $\neg q$ on $V$ under $\varphi$.
% \begin{enumerate}[(i)]
%   \item $\psi$ is a SNC of $q$ on $V$ under $\varphi$ iff $\neg \psi$ is a WSC
%    of $\neg q$ on $V$ under $\varphi$.
%   \item $\psi$ is a WSC of $q$ on $V$ under $\varphi$ iff $\neg \psi$ is a  SNC
%    of $\neg q$ on $V$ under $\varphi$.
% \end{enumerate}
\end{proposition}
This show that the SNC and WSC are in fact dual conditions. Under the dual property, we can consider the SNC party only in sometimes, while
the WSC part can be talked similarly.
%\begin{proof}
%
%\end{proof}

In order to generalise Definition~\ref{def:NC:SC} to arbitrary formulas, one can replace $q$ (in the definition)  by any formula $\alpha$, and redefine  $V$ as a subset of $\Var(\alpha) \cup \Var(\phi)$.

%  For the case of formula, we have that the SCN (WSC) of any formula can be defined as follows:
%\begin{definition}\label{formulaNS}
   % Let $\Gamma$ be a formula or an initial \MPK-structure, $\alpha$ be a formula and $P\subseteq (\Var(\Gamma) \cup \Var(\alpha))$. A formula $\varphi$ of $P$ is  said to be a NC (SC) of $\alpha$ on $P$ under $\Gamma$ iff $\Gamma \models \alpha \rto \varphi$. It is said to be a SNC (WSC) if it is a NC (SC), and for any other NC (SC) $\varphi'$, we have that $\Gamma \models \varphi \rto \varphi'$ ($\Gamma \models \varphi' \rto \varphi$).
   % \end{definition}


  It is seems that the SNC and WSC of any formula can be reduced to that of a proposition.
\begin{proposition}\label{formulaNS_to_p}
     Let $\Gamma$ and $\alpha$ be two formulas, $V \subseteq \Var(\alpha) \cup \Var(\phi)$  and $q$ is a new proposition not in $\Gamma$ and $\alpha$.
 Then, a formula $\varphi$ of $V$ is the SNC (WSC) of $\alpha$ on $V$ under  $\Gamma$ iff it is the SNC (WSC) of $q$ on $V$ under $\Gamma' = \Gamma \cup \{q \equiv \alpha\}$.   \end{proposition}

We propose the theorem of computing the SNC (WSC) of an atom due to the SNC (WSC) of a formula can be changed to the SNC (WSC) of an atom by Proposition~\ref{formulaNS_to_p}.
\begin{theorem}\label{thm:SNC:WSC:forget}
 Let $\varphi$ be a formula, $V\subseteq\Var(\varphi)$ and $q\in\Var(\varphi)- V$.
 \begin{enumerate}[(i)]
   \item $\CTLforget (\varphi \land q$, $(\Var(\varphi) \cup \{q\})- V)$
   is a SNC of $q$ on $V$ under $\varphi$.
   \item  $\neg\CTLforget (\varphi \land \neg q$, $(\Var(\varphi) \cup \{q\})- V)$
   is a WSC of $q$ on $V$ under $\varphi$.
 \end{enumerate}
 \end{theorem}

 As we have said before that any initial $\MPK$-structure can be characterized by a \CTL\ formula, we can obtain the SNC (WSC) of an initial $\MPK$-structure for satisfy some needed property (formula) by forgetting.
\begin{theorem}\label{thm:inK:SNC}
Let ${\cal K}= (\Hm, s)$ be an initial \MPK-structure with $\Hm=(S,R,L,s_0)$ on the finite set $\Ha$ of atoms, $V \subseteq \Ha$ and $q\in V'$ ($V' = \Ha - V$). Then:
 \begin{enumerate}[(i)]
   \item the SNC of $q$ on $V$ under ${\cal K}$ is $\CTLforget({\cal F}_{\Ha}({\cal K}) \wedge q, V')$.
   \item the WSC of $q$ on $V$ under ${\cal K}$ is $\neg \CTLforget({\cal F}_{\Ha}({\cal K}) \wedge \neg q, V')$.
 \end{enumerate}
\end{theorem}
%\label{thm:inK:SNC}\begin{proof}
%(i)
%As we know that any initial \MPK-structure ${\cal K}$ can be described as a characterizing formula ${\cal F}_{\Ha}({\cal K})$, then the SNC of $q$ on $V$ under ${\cal F}_{\Ha}({\cal K})$ is $\CTLforget({\cal F}_{\Ha}({\cal K}) \wedge q, \Ha - V)$. We will prove that $\CTLforget({\cal F}_{V \cup \{q\}}({\cal K}_{|V \cup \{q\}}) \wedge q, q)  \equiv  \CTLforget({\cal F}_{\Ha}({\cal K}) \wedge q, \Ha - V)$.
%
%($\Rto$) $\forall {\cal K}_1 \in \Mod(\CTLforget({\cal F}_{V \cup \{q\}}({\cal K}_{|V \cup \{q\}}) \wedge q, q))$\\
%$\Rto$ there is an initial \MPK-structure ${\cal K}'$ such that ${\cal K}' \models {\cal F}_{V \cup \{q\}}({\cal K}_{|V \cup \{q\}}) \wedge q$ and ${\cal K}_1 \lrto_{\{q\}} {\cal K}'$\\
%$\Rto$ ${\cal K}' \lrto_{\Ha-(V\cup \{q\})} {\cal K}_{|V \cup \{q\}}$  \hfill (Theorem~\ref{CF})\\
%$\Rto$ ${\cal K}_1 \lrto_{\Ha-V} {\cal K}_{|V \cup \{q\}}$   \hfill (Proposition~\ref{div})\\
%$\Rto$ ${\cal K}_{|V \cup \{q\}} \lrto_{\Ha-(V \cup \{q\})} {\cal K}$   \hfill  (Proposition~\ref{pro:VQ})\\
%$\Rto$ ${\cal K}' \lrto_{\Ha-(V\cup \{q\})} {\cal K}$  \hfill (Proposition~\ref{div})\\
%$\Rto$ ${\cal K} \models {\cal F}_{\Ha}({\cal K}) \wedge q$\\
%$\Rto$ ${\cal K}_1 \lrto_{\Ha -V} {\cal K}$\\
%$\Rto$ ${\cal K}_1 \models \CTLforget({\cal F}_{\Ha}({\cal K}) \wedge q, \Ha - V)$
%
%$(\Lto)$ $\forall {\cal K}_1 \in \Mod(\CTLforget({\cal F}_{\Ha}({\cal K}) \wedge q, \Ha - V))$ \\
%$\Rto$ there an initial \MPK-structure ${\cal K}_2$ \st\ ${\cal K}_2 \models {\cal F}_{\Ha}({\cal K}) \wedge q$ and ${\cal K}_1 \lrto_{\Ha - V} {\cal K}_2$\\
%$\Rto$ ${\cal K}_2 \lrto_{{\O}} {\cal K}$   \hfill (Theorem~\ref{CF})\\
%$\Rto$ ${\cal K}_2 \lrto_{\Ha-(V\cup \{q\})} {\cal K}_{|V \cup \{q\}}$ due to ${\cal K}_{|V \cup \{q\}} \lrto_{\Ha-(V \cup \{q\})} {\cal K}$   \hfill  (Proposition~\ref{pro:VQ})\\
%$\Rto$ ${\cal K}_2 \models {\cal F}_{V \cup \{q\}}({\cal K}_{|V \cup \{q\}}) \wedge q$  \\
%$\Rto$ ${\cal K}_1 \models \CTLforget({\cal F}_{V \cup \{q\}}({\cal K}_{|V \cup \{q\}}) \wedge q, q)$.
%
%(ii) This is proved by the dual property.
%\end{proof}

\begin{example}
For the Example~\ref{exmp:1}, the WSC of $\varphi$ on $V$ under ${\cal K} =(\Hm, s_0)$ is $\neg \CTLforget({\cal F}_{\Ha}({\cal K}) \wedge (q \equiv \varphi) \wedge \neg q, \Ha - V)$.
\end{example}


\section{Algorithm to Compute Forgetting}
To compute the forgetting in \CTL, we propose a model-based method in this part.
Literally speaking, the model-based method means that we can obtain the result of forgetting in \CTL\ by obtain all the possible finite models of this result.
By the definition of forgetting in \CTL, the set of models of the result of forgetting is also a finite set of initial \MPK-structures.
%To compute the forgetting in \CTL, we propose a model-based method in this part.
%Literally speaking, the model-based method means that we can obtain the result of forgetting in \CTL\ by obtain all the possible finite models of this result.
%% How can we obtain all the \MPK-models is what we will solved in this part.
%%The resolution-based method obtain the result of forgetting by obtaining all the possible resolutions which can be implied by the original formula.
%
%As we have said that the set of models of any formula $\varphi$ is finite, hence if we can obtain all the models of $\varphi$ then we can express this formula by the disjunction of those characteristic formulas of those models.
%By the definition of forgetting in \CTL, the set of models of the result of forgetting is also a finite set of initial \MPK-structures.
%Then the model-based method is generated for forgetting in \CTL.
%
%Though the set of models of the result of forgetting is finite, while how many models is there?
%%That's right we should given the bound of the number of the models.
%As it is said in Theorem~\ref{thm:VBChFEQ} that if two initial \MPK-structures are $V$-bisimulation, then their characteristic formulas is equal.

%Then let $\varphi$ be a CTL formula, the $|\Var(\varphi)|=m$ is a positive integer, we have the following theorem:
%\begin{theorem}
%Let $\varphi$ be a CTL formula, $V=\Var(\varphi)$, $|V|=m$, and ${\cal K}=(\Hm, s_0)$ with $\Hm=(S, R,L,s_0)$ be a initial \MPK-structure. If $(\Hm, s_0) \models \varphi$, then there is an initial \MPK-structure ${\cal K}'=(\Hm', s_0')$ with $\Hm'=(S', R',L',s_0')$ that satisfy:
%\begin{enumerate}[(i)]
%  \item $|S'|$ is at most $2^m$,
%  \item $|R'|$ is at most $2^m * 2^m$,
%  \item $|L'|$ is at most $2^m * 2^m$,
%  \item ${\cal K} \lrto_{\Ha - V} {\cal K}'$ and $(\Hm', s_0') \models \varphi$.
%\end{enumerate}
%\end{theorem}
%\begin{proof}
%If $|S| \leq 2^m$, this result clearly holds. If $|S| > 2^m$, let ${\cal K}' = {\cal K}_{|V}$, then it is apparent that ${\cal K} \lrto_{\Ha - V} {\cal K}'$ by Proposition~\ref{pro:VQ} and $(\Hm', s_0') \models \varphi$. In the worst case, the number of states in $S'$ is $2^m$ by the definition of ${\cal K}_{|V}$. Then the theorem is proved.
%\end{proof}

%By this theorem, we can see that any initial \MPK-structures ${\cal K}$ that satisfy $\varphi$ can be transformed to an initial \MPK-structure ${\cal K}'$ such that ${\cal K} \lrto_{\Ha - V} {\cal K}'$ and ${\cal K} \models \varphi$ iff ${\cal K}' \models \varphi$ due to $\IR(\varphi, \Ha- V)$, in which $V= \Var(\varphi)$. Therefore, the size of the model of $\varphi$ is at most $2^m$ by Theorem~\ref{thm:VBChFEQ}(we only consider the number of states of this model).

Then we have the following model-based Algorithm~\ref{alg:compute:forgetting:by:VB} to compute the forgetting under CTL.
By Lemma~\ref{lem:models:formula} and Theorem~\ref{CF} we can prove the correctness of this algorithm.


\begin{algorithm}[tb]
\caption{Model-based: Computing forgetting}
\label{alg:compute:forgetting:by:VB}
\KwIn{A CTL formula $\varphi$ and a set $V$ of atoms}
\KwOut{$\CTLforget(\varphi, V)$}% ????
$T={\O}$ // the set of models of $\varphi$ \;
$T' = {\O}$ // the set of possible initial \MPK-structures \;
%$m=|\Var(\varphi)|$\;
$n=|\Ha|$\;

\For {$i=1, ..., 2^n$}{
       % Enumerating all possible initial \MPK-structures $(\Hm, s_0)$ with $\Hm=(S, R, L,s_0)$ and $|S|=i$\;
        \For {$s_j\in \{s_1, \dots, s_i\}$}{
            Let $s_j$ be an initial state, construct $\Hm=(S, R, L,s_j)$ by the definition of model structure with $S=\{s_1, \dots, s_i\}$\;
            \For {$\cal K\in T'$} {
             \If {$(\Hm, s_j) \nleftrightarrow_{\overline {\Var(\varphi)}} \cal K$}{
                 Let $T' \leftarrow T' \cup \{(\Hm, s_j)\}$\;
             }
             }
        }
        \For {$(\Hm, s_0) \in T'$ }{
            \If {$(\Hm,s_0) \models \varphi$}{
                $T \leftarrow T \cup \{(\Hm,s_0)\}$\;
            }
        }
       % For all initial \MPK-structures $(\Hm, s_0)$ \If {$(\Hm,s_0) \models \varphi$}{
%            $T= T \cup \{(\Hm,s_0)\}$\;
%        }
}
%\For {${\cal K} =(\Hm, s_0) \in T$} {
%    Let $T' = T' \cup \{{\cal K}_{|V} \}$\;
%}
%
%\For {i=1, ..., $2^{m-n}$}{
%        Enumerating all possible initial \MPK-structures $(\Hm', s_0')$ with $\Hm'=(S', R', L',s_0')$ and $|S'|=i$\;
%        For all initial \MPK-structures $(\Hm', s_0')$ \If {$\exists (\Hm,s_0)\in T$ s.t. $(\Hm,s_0) \lrto_V (\Hm', s_0')$}{
%            $T'= T' \cup \{(\Hm',s_0')\}$\;
%        }
%}
\Return $\bigvee_{(\Hm', s_0')\in T} {\cal F}_{\overline V}(\Hm', s_0')$.
\end{algorithm}

\begin{example}
Let $\varphi=\ALL\GLOBAL \ALL\FUTURE (p \wedge r)$, $\Ha=\{p,r\}$ and $V=\{r\}$. For convenience, we use the label of a state to express the state and then remove the label function in a model structure.
Let $\Hm_1=(\{\{p,r\}\}, \{(\{p,r\}, \{p,r\})\}, \{p,r\})$ and $\Hm_2=(\{\emptyset,\{p,r\}\}, \{(\emptyset, \{p,r\})$, $(\{p,r\}$, $\{p,r\})\}, \emptyset)$.
%\Hm_3=(\{\emptyset,\{p\}$, $\{p,r\}\}$, $\{(\emptyset$, $\{p\})$, $(\{p\}$, $\{p, r\})$, $(\{p, r\}$, $\emptyset)\}$, $\emptyset)$,
The set of models of$\varphi$ is $\Mod(\varphi)=\{(\Hm_1, \{p\})$, $(\Hm_2$, $\emptyset), \dots\}$.

Let $\Hm_1'=(\{\{p\}\}, \{(\{p\}, \{p\})\}, \{p\})$ and $\Hm_2'=(\{\emptyset,\{p\}\}, \{(\emptyset, \{p\})$, $(\{p\}$, $\{p\})\}, \emptyset)$
  Then we can obtain all the possible initial \MPK-structure that is a model of $\CTLforget(\varphi, V)$, \ie $\Mod(\CTLforget(\varphi, V)) =\{{\cal K}_1=(\Hm_1', \{p\}), {\cal K}_2=(\Hm_2', \emptyset), \dots\}$.

Let $V'=\{p\}$, then
 ${\cal F}_{V'}({\cal K}_1)= p \wedge \ALL\GLOBAL (p\supset \EXIST \NEXT p \wedge \ALL \NEXT p)$,
 and
 ${\cal F}_{V'}({\cal K}_2)=\neg p\wedge \ALL\GLOBAL (p\supset \EXIST \NEXT \neg p \wedge \ALL \NEXT \neg p) \wedge \ALL\GLOBAL (\neg p\supset \EXIST \NEXT p \wedge \ALL \NEXT p)$. Similarly, we can obtain the characteristic formula of other models and then the $\CTLforget(\varphi, V)$.% is the disjunction of all the characteristic formulae of those models.
\end{example}

\begin{proposition}\label{pro:time:alg1}
 Let $\varphi$ be a CTL formula and $V\subseteq \Ha$. The time and space complexity of
Algorithm~\ref{alg:compute:forgetting:by:VB} are $O(2^{m*2^m})$.%and the space complexity is $O(2^{2m})$, where $|\Ha| = m$.
\end{proposition}

\section{Concluding Remark}
Based on the proposed $V$-bisimulation between \MPK-structures,  forgetting in \CTL\ and characteristic formula on $V$ on an initial \MPK-structure $\cal K$,
a method compute the WSC (SNC) of a property $\varphi$ (a \CTL\ formula) on $\cal K$ and $V$ has been introduced by computing forgetting in \CTL.
Besides, we have shown that the \CTL\ system is close under
our definition of forgetting, and this definition satisfies those four postulates of forgetting.
As we have said the complexity of Algorithm~\ref{alg:compute:forgetting:by:VB} is $O(2^{m*2^m})$ (very inefficient), a future
work is to find an efficient algorithm to compute forgetting in \CTL\ and then WSC (SNC).



\section{Proof}
%\textbf{Proposition}~\ref{Vbi:Equ}
%Let $V\subseteq\cal A$
%%${\cal M}_i=(S_i,R_i,L_i,s_0^i)~(i=1,2)$ be model structures
%and ${\cal K}_i=({\cal M}_i,s_i)~(i=1,2)$ be \MPK-structures.
%Then $({\cal K}_1,{\cal K}_2)\in\cal B$ if and only if
%  \begin{enumerate}[(i)]
%    \item $L_1(s_1)- V = L_2(s_2)- V$,
%    \item for every $(s_1,s_1')\in R_1$, there is $(s_2,s_2')\in R_2$
%    such that $({\cal K}_1',{\cal K}_2')\in \Hb$, and
%    \item for every $(s_2,s_2')\in R_2$, there is $(s_1,s_1')\in R_1$
%    such that $({\cal K}_1',{\cal K}_2')\in \Hb$,
%   \end{enumerate}
% where ${\cal K}_i'=({\cal M}_i,s_i')$ with $i\in\{1,2\}$.\\
%\begin{proof}
%$(\Rto)$
%(a) It is apparent that $L_1(s_1)- V = L_2(s_2)- V$;
%(b) %We will show that for each $(s_1, s_1') \in R_1$, there is a $(s_2, s_2')\in R_2$ such that $({\cal K}_1', {\cal K}_2') \in \Hb$.
%$({\cal K}_1, {\cal K}_2) \in \Hb$ iff $({\cal K}_1, {\cal K}_2) \in \Hb_i$ for all $i \geq 0$, then for each $(s_1, s_1') \in R_1$, there is a $(s_2, s_2')\in R_2$  such that  $({\cal K}_1', {\cal K}_2') \in \Hb_{i-1}$ for all $i > 0$ and then $L_1(s_1')- V = L_2(s_2')- V$. Therefore, $({\cal K}_1', {\cal K}_2') \in \Hb$.
%(c) %We will show that for each $(s_2, s_2') \in R_1$, there is a $(s_1, s_1')\in R_2$ such that $({\cal K}_1', {\cal K}_2') \in \Hb$.
% This is similar with (b).
%
%$(\Lto)$ (a) $L_1(s_1)- V = L_2(s_2)- V$ implies that $(s_1, s_2) \in \Hb_0$;
%(b) Condition (ii) implies that for every $(s_1,s_1')\in R_1$, there is $(s_2,s_2')\in R_2$
%    such that $({\cal K}_1',{\cal K}_2')\in \Hb_i$ for all $i \geq 0$;
%(c) Condition (iii) implies that for every $(s_2,s_2')\in R_2$, there is $(s_1,s_1')\in R_1$
%    such that $({\cal K}_1',{\cal K}_2')\in \Hb_i$ for all $i \geq 0$\\
%$\Rto$ $({\cal K}_1, {\cal K}_2) \in \Hb_i$ for all $i \geq 0$\\
%$\Rto$ $({\cal K}_1,{\cal K}_2)\in\cal B$.
%\end{proof}


\begin{lemma}\label{lem:B:relations}
  Let  $\Hb_0, \Hb_1,\ldots$ be the ones in the definition of section 3.1.
   Then,  for each $i\ge 0$,
   \begin{enumerate}[(i)]
     \item $\Hb_{i+1}\subseteq \Hb_i$;
     \item there is a (smallest) $k\ge 0$ such that $\Hb_{k+1}=\Hb_k$;
     \item $\Hb_i$ is reflexive, symmetric and transitive.
   \end{enumerate}
\end{lemma}
\begin{proof}
  (i)
  Base: it is clear for $i=0$ by the above definition.

  Step: suppose it holds for $i=n$, \ie $\Hb_{n+1}\subseteq\Hb_n$. \\
  $(s,s')\in\Hb_{n+2}$\\
  $\Rto$ (a) $(s,s')\in  \Hb_0$,
    (b) for every $(s,s_1)\in R$, there is $(s',s_1')\in R'$
     such that $(s_1,s_1')\in \Hb_{n+1}$, and
    (c)  for every $(s',s_1')\in R'$, there is $(s,s_1)\in R$
    such that $(s_1,s_1')\in \Hb_{n+1}$\\
  $\Rto$ (a) $(s,s')\in  \Hb_0$,
   (b) for every $(s,s_1)\in R$, there is $(s',s_1')\in R'$
     such that $(s_1,s_1')\in \Hb_{n}$ by inductive assumption, and
   (c)  for every $(s',s_1')\in R'$, there is $(s,s_1)\in R$
    such that $(s_1,s_1')\in \Hb_{n}$ by inductive assumption\\
  $\Rto$ $(s,s')\in \Hb_{n+1}$.

  (ii) and (iii) are evident from (i) and the definition of $\Hb_i$.
\end{proof}


\textbf{Lemma}~\ref{lem:equive}  The relation $\lrto_V$ is an equivalence relation.\\
\begin{proof}
It is clear from Lemma~\ref{lem:B:relations} (ii) such that there is a $k \geq $ 0 where $\Hb_k = \Hb_{k+1}$ which is  $\lrto_V$, and it is reflexive, symmetric and transitive by (iii).
\end{proof}


\textbf{Proposition}~\ref{div}
Let $i\in \{1,2\}$, $V_1,V_2\subseteq\cal A$, $s_i'$s be two states and
  $\pi_i'$s be two pathes,
and ${\cal K}_i=({\cal M}_i,s_i)~(i=1,2,3)$ be \MPK-structures
 such that
${\cal K}_1\lrto_{V_1}{\cal K}_2$ and ${\cal K}_2\lrto_{V_2}{\cal K}_3$.
 Then:
 \begin{enumerate}[(i)]
   \item $s_1'\lrto_{V_i}s_2'~(i=1,2)$ implies $s_1'\lrto_{V_1\cup V_2}s_2'$;
   \item $\pi_1'\lrto_{V_i}\pi_2'~(i=1,2)$ implies $\pi_1'\lrto_{V_1\cup V_2}\pi_2'$;
   \item for each path $\pi_{s_1}$ of $\Hm_1$ there is a path $\pi_{s_2}$  of $\Hm_2$ such that $\pi_{s_1} \lrto_{V_1} \pi_{s_2}$, and vice versa;
   \item ${\cal K}_1\lrto_{V_1\cup V_2}{\cal K}_3$;
   \item If $V_1 \subseteq V_2$ then ${\cal K}_1 \lrto_{V_2} {\cal K}_2$.
 \end{enumerate}
\begin{proof}
In order to distinguish the relations $\Hb_0, \Hb_1, \dots$ for different set $V \subseteq \Ha$, by $\Hb_i^V$ we mean the relation $\Hb_1, \Hb_2, \dots$ for $V \subseteq \Ha$.
Denote as $\Hb_0, \Hb_1, \dots$ when the underlying set $V$ is clear from the context. Moreover, for the ease of notation, we will refer to $\lrto_V$ by $\Hb$ (i.e., without subindex).

(i) Base: it is clear for $n = 0$.\\
Step: For $n > 0$, supposing if $({\cal K}_1, {\cal K}_2) \in \Hb_i^{V_1}$ and $({\cal K}_1, {\cal K}_2) \in \Hb_i^{V_2}$ then $({\cal K}_1, {\cal K}_2) \in \Hb_i^{V_1 \cup V_2}$ for all $0 \leq i \leq n$. We will show that if $({\cal K}_1, {\cal K}_2) \in \Hb_{n+1}^{V_1}$ and $({\cal K}_1, {\cal K}_2) \in \Hb_{n+1}^{V_2}$ then $({\cal K}_1, {\cal K}_2) \in \Hb_{n+1}^{V_1 \cup V_2}$.\\
(a) It is evident that $L_1(s_1) - (V_1 \cup V_2) = L_2(s_2) - (V_1\cup V_2)$.\\
(b) We will show that for each $(s_1, s_1^1) \in R_1$ there is a $(s_2, s_2^1) \in R_2$ such that $(s_1^1, s_2^1) \in \Hb_n^{V_1 \cup V_2}$. There is $({\cal K}_1^1, {\cal K}_2^1) \in \Hb_{n-1}^{V_1 \cup V_2}$
due to $({\cal K}_1, {\cal K}_2) \in \Hb_n^{V_1 \cup V_2}$ by inductive assumption. Then we only need to prove for each $(s_1^1, s_1^2) \in R_1$ there is a $(s_2^1, s_2^2) \in R_2$ such that $({\cal K}_1^2, {\cal K}_2^2) \in \Hb_{n-2}^{V_1 \cup V_2}$ and for each $(s_2^1, s_2^2) \in R_2$ there is a $(s_1^1, s_1^2) \in R_1$ such that $({\cal K}_1^2, {\cal K}_2^2) \in \Hb_{n-2}^{V_1 \cup V_2}$. Therefore, we only need to prove that for each $(s_1^n, s_1^{n+1}) \in R_1$ there is a $(s_2^n, s_2^{n+1}) \in R_2$ such that $({\cal K}_1^{n+1}, {\cal K}_2^{n+1}) \in \Hb_0^{V_1 \cup V_2}$ and for each $(s_2^n, s_2^{n+1}) \in R_2$ there is a $(s_1^n, s_1^{n+1}) \in R_1$ such that $({\cal K}_1^{n+1}, {\cal K}_2^{n+1}) \in \Hb_0^{V_1 \cup V_2}$. It is apparent that $L_1(s_1^{n+1}) - (V_1 \cup V_2) = L_1(s_2^{n+1}) - (V_1 \cup V_2)$ due to $({\cal K}_1, {\cal K}_2) \in \Hb_{n+1}^{V_1}$ and $({\cal K}_1, {\cal K}_2) \in \Hb_{n+1}^{V_2}$.
Where ${\cal K}_i^j = (\Hm_i, s_i^j)$ with $i \in \{1, 2\}$ and $0 < j \leq n+1$.\\
(c) It is similar with (b).

(ii) It is clear from (i).

(iii)
%It is clear from Proposition~\ref{Vbi:Equ}.
%\begin{proposition}\label{Vbi:Equ}
The following property show our result directly.
Let $V\subseteq\cal A$
%${\cal M}_i=(S_i,R_i,L_i,s_0^i)~(i=1,2)$ be model structures
and ${\cal K}_i=({\cal M}_i,s_i)~(i=1,2)$ be \MPK-structures.
Then $({\cal K}_1,{\cal K}_2)\in\cal B$ if and only if
  \begin{enumerate}[(a)]
    \item $L_1(s_1)- V = L_2(s_2)- V$,
    \item for every $(s_1,s_1')\in R_1$, there is $(s_2,s_2')\in R_2$
    such that $({\cal K}_1',{\cal K}_2')\in \Hb$, and
    \item for every $(s_2,s_2')\in R_2$, there is $(s_1,s_1')\in R_1$
    such that $({\cal K}_1',{\cal K}_2')\in \Hb$,
   \end{enumerate}
 where ${\cal K}_i'=({\cal M}_i,s_i')$ with $i\in\{1,2\}$.

 We prove it from the following two aspects:

 $(\Rto)$
(a) It is apparent that $L_1(s_1)- V = L_2(s_2)- V$;
(b) %We will show that for each $(s_1, s_1') \in R_1$, there is a $(s_2, s_2')\in R_2$ such that $({\cal K}_1', {\cal K}_2') \in \Hb$.
$({\cal K}_1, {\cal K}_2) \in \Hb$ iff $({\cal K}_1, {\cal K}_2) \in \Hb_i$ for all $i \geq 0$, then for each $(s_1, s_1') \in R_1$, there is a $(s_2, s_2')\in R_2$  such that  $({\cal K}_1', {\cal K}_2') \in \Hb_{i-1}$ for all $i > 0$ and then $L_1(s_1')- V = L_2(s_2')- V$. Therefore, $({\cal K}_1', {\cal K}_2') \in \Hb$.
(c) %We will show that for each $(s_2, s_2') \in R_1$, there is a $(s_1, s_1')\in R_2$ such that $({\cal K}_1', {\cal K}_2') \in \Hb$.
 This is similar with (b).

$(\Lto)$ (a) $L_1(s_1)- V = L_2(s_2)- V$ implies that $(s_1, s_2) \in \Hb_0$;
(b) Condition (ii) implies that for every $(s_1,s_1')\in R_1$, there is $(s_2,s_2')\in R_2$
    such that $({\cal K}_1',{\cal K}_2')\in \Hb_i$ for all $i \geq 0$;
(c) Condition (iii) implies that for every $(s_2,s_2')\in R_2$, there is $(s_1,s_1')\in R_1$
    such that $({\cal K}_1',{\cal K}_2')\in \Hb_i$ for all $i \geq 0$\\
$\Rto$ $({\cal K}_1, {\cal K}_2) \in \Hb_i$ for all $i \geq 0$\\
$\Rto$ $({\cal K}_1,{\cal K}_2)\in\cal B$.
%\end{proposition}


(iv) Let ${\cal M}_i=(S_i,R_i,L_i,s_i)~(i=1,2,3)$, $s_1 \lrto_{V_1} s_2$ via a binary relation $\Hb$, and $s_2 \lrto_{V_2} s_3$ via a binary relation $\Hb''$. Let $\Hb' = \{(w_1, w_3)| (w_1, w_2)\in \Hb$ and $(w_2, w_3)\in \Hb_2\}$. It's apparent that $(s_1, s_3) \in \Hb'$. We prove $\Hb'$ is a $V_1 \cup V_2$-bisimulation containing $(s_1, s_3)$ from the (a), (b) and (c) of the previous step (iii) of $X$-bisimulation (where $X$ is a set of atoms). For all $(w_1, w_3) \in \Hb'$:
\begin{enumerate}[(a)]
  \item there is $w_2 \in S_2$ such that $(w_1,w_2)\in \Hb$ and $(w_2, w_3)\in \Hb''$, and $\forall q \notin V_1$, $q \in L_1(w_1)$ iff $q \in L_2(w_2)$ by $w_1 \lrto_{V_1} w_2$ and $\forall q' \notin V_2$, $q'\in L_2(w_2)$ iff $q'\in L_3(w_3)$ by $w_2 \lrto_{V_2} w_3$. Then we have $\forall r\notin V_1 \cup V_2$, $r \in L_1(w_1)$ iff $r \in L_3(w_3)$.
  \item if $(w_1, u_1) \in \Hr_1$, then $\exists u_2\in S_2$ such that $(w_2, u_2) \in \Hr_2$ and $(u_1,u_2)\in \Hb$ (due to $(w_1,w_2)\in \Hb$ and $(w_2, w_3) \in \Hb''$ by the definition of $\Hb'$); and then $\exists u_3 \in S_3$ such that $(w_3, u_3) \in \Hr_3$ and $(u_2, u_3) \in \Hb''$, hence $(u_1, u_3) \in \Hb'$ by the definition of $\Hb'$.
  \item if $(w_3, u_3) \in \Hr_3$, then $\exists u_2\in S_2$ such that $(w_2, u_2) \in \Hr_2$ and $(u_2, u_3) \in \Hb_2$; and then $\exists u_1 \in S_1$ such that $(w_1, u_1) \in \Hr_1$ and $(u_1, u_2) \in \Hb$, hence $(u_1, u_3) \Hb'$ by the definition of $\Hb'$.
\end{enumerate}

(v) Let ${\cal K}_{i, j}=(\Hm_i, s_{i,j})$ and $(s_{i, k}, s_{i, k+1}) \in R_i$ mean that $s_{i, k+1}$ is the $(k+2)$-th node in the path
 $(s_i, s_{i, 1}, s_{i,2}, \dots , s_{i, k+1}, \dots)$ ($i=1,2$).
We will show that $({\cal K}_1, {\cal K}_2) \in \Hb_n^{V_2}$ for all $n \ge 0$ inductively.

Base: $L_1(s_1) - V_1 = L_2(s_2) - V_1$\\
$\Rto$ $\forall q \in {\cal A} - V_1$ there is $q \in L_1(s_1)$ iff $q \in L_2(s_2)$\\
$\Rto$ $\forall q \in {\cal A} - V_2$ there is $q \in L_1(s_1)$ iff $q \in L_2(s_2)$ due to $V_1 \subseteq V_2$\\
$\Rto$ $L_1(s_1) - V_2 = L_2(s_2) - V_2$, \ie\ $({\cal K}_1, {\cal K}_2) \in \Hb_0^{V_2}$.

Step: Supposing that $({\cal K}_1, {\cal K}_2) \in \Hb_i^{V_2}$ for all $0 \leq i \leq k$ ($k > 0)$, we will show $({\cal K}_1, {\cal K}_2) \in \Hb_{k+1}^{V_2}$.
\begin{enumerate} [(a)]
  \item It is apparent that $L_1(s_1) - V_2 = L_2(s_2) - V_2$ by base.
  \item $\forall (s_1, s_{1,1}) \in R_1$, we will show that there is a $(s_2, s_{2, 1}) \in R_2$ \st\ $({\cal K}_{1,1}, {\cal K}_{2,1})\in \Hb_k^{V_2}$. $({\cal K}_{1,1}, {\cal K}_{2,1})\in \Hb_{k-1}^{V_2}$ by inductive assumption, we need only to prove the following points:\\
      (a) $\forall (s_{1, k}, s_{1, k+1}) \in R_1$ there is a $(s_{2, k}, s_{2, k+1})\in R_2$ \st\ $({\cal K}_{1,k+1}, {\cal K}_{2,k+1})\in \Hb_0^{V_2}$ due to $({\cal K}_{1,1}, {\cal K}_{2,1})\in \Hb_{k}^{V_1}$. It is easy to see that $L_1(s_{1, k+1}) - V_1 = L_1(s_{2, k+1}) - V_1$, then there is $L_1(s_{1, k+1})- V_2 = L_1(s_{2, k+1}) - V_2$. Therefore, $({\cal K}_{1,k+1}, {\cal K}_{2,k+1})\in \Hb_0^{V_2}$.\\
      (b) $\forall (s_{2, k}, s_{2, k+1}) \in R_1$ there is a $(s_{1, k}, s_{1, k+1}) \in R_1$ \st\ $({\cal K}_{1,k+1}, {\cal K}_{2,k+1})\in \Hb_0^{V_2}$ due to $({\cal K}_{1,1}, {\cal K}_{2,1})\in \Hb_{k}^{V_1}$. This can be proved as (a).
  \item $\forall (s_2, s_{2,1}) \in R_1$, we will show that there is a $(s_1, s_{1, 1}) \in R_2$ \st\ $({\cal K}_{1,1}, {\cal K}_{2,1})\in \Hb_k^{V_2}$. This can be proved as (ii).
\end{enumerate}

\end{proof}

%\textbf{Proposition}\ref{pro:VQ}
%\begin{proof}
%Base. It is apparent that $L(s_0)- V' = L*(s_0^*)- V'$;\\
%Step. (i) For any $(s_0, s_1) \in R$ there is $s_1' \in [s_1]_{\Hb}$ such that $([s_0^*]_{\Ha}, [s_1']_{\Ha}) \in R^*$ and $s_1 \lrto_{V'} [s_1']_{\Hb}$ by the Definition~\ref{def:V-quotient};\\
%(ii) Similarly, for any $([s_0^*]_{\Ha}, [s_1']_{\Hb})\in R^*$ there is $(s_0, s_1) \in R$ such that $s_1 \in [s_1']_{\Hb}$ and $s_1 \lrto_{V'} [s_1']_{\Hb}$.
%\end{proof}



\textbf{Theorem}\ref{thm:V-bisimulation:EQ}
Let $V\subseteq\cal A$, ${\cal K}_i~(i=1,2)$ be two \MPK-structures such that
  ${\cal K}_1\lrto_V{\cal K}_2$ and $\phi$ a formula with $\IR(\phi,V)$. Then
  ${\cal K}_1\models\phi$ if and only if ${\cal K}_2\models\phi$.\\
\begin{proof}
This theorem can be proved by inducting on the formula $\phi$ and supposing $\Var(\phi) \cap V = \O$.

Here we only prove the only-if direction. The other direction can be similarly proved.

\textbf{Case} $\phi = p$ where $p \in \Ha - V$:\\
$(\Hm, s) \models \phi$ iff $p\in L(s)$  \hfill  (by the definition of satisfiability) \\
$\LRto$ $p \in L'(s')$ \hfill ($s \lrto_V s'$)\\
$\LRto$ $(\Hm', s') \models \phi$

\textbf{Case} $\phi = \neg \psi$:\\
$(\Hm, s) \models \phi$ iff $(\Hm, s) \nvDash \psi$ \\
$\LRto$ $(\Hm', s') \nvDash \psi$  \hfill   (induction hypothesis)\\
$\LRto$ $(\Hm', s') \models \phi$

\textbf{Case} $\phi = \psi_1 \vee \psi_2$:\\
$(\Hm, s) \models \phi$\\
$\LRto$ $(\Hm, s) \models \psi_1$ or $(\Hm, s) \models \psi_2$\\
$\LRto$ $(\Hm', s') \models \psi_1$ or $(\Hm', s') \models \psi_2$   \hfill  (induction hypothesis)\\
$\LRto$ $(\Hm', s') \models \phi$

\textbf{Case} $\phi = \EXIST \NEXT \psi$:\\
%By Lemma~\ref{V_path}, we assume there are two paths $\pi = s, s_1, ...$ and $\pi' = s', s_1', ...$ such that $\pi \lrto_V \pi'$.\\
$\Hm, s \models \phi$ \\
$\LRto$ There is a path $\pi = (s, s_1, ...)$ such that $\Hm, s_1 \models \psi$\\
$\LRto$ There is a path $\pi' = (s', s_1', ...)$ such that $\pi \lrto_V \pi'$ \hfill   ($s \lrto_V s'$, Proposition~\ref{div})\\
$\LRto$ $s_1 \lrto_V s_1'$  \hfill ($\pi \lrto_V \pi'$)\\
$\LRto$ $(\Hm', s_1') \models \psi$  \hfill  (induction hypothesis)\\
$\LRto$ $(\Hm', s') \models \phi$

\textbf{Case} $\phi = \EXIST \GLOBAL \psi$:\\
$\Hm, s \models \phi$ \\
$\LRto$ There is a path $\pi =(s=s_0, s_1, ...)$ such that for each $i \geq 0$ there is $(\Hm, s_i) \models \psi$\\
$\LRto$ There is a path $\pi' = (s'=s_0', s_1', ...)$ such that $\pi \lrto_V \pi'$   \hfill ($s \lrto_V s'$, Proposition~\ref{div})\\
$\LRto$ $s_i \lrto_V s_i'$ for each $i \geq 0$ \hfill ($\pi \lrto_V \pi'$)\\
$\LRto$ $(\Hm', s_i') \models \psi$ for each $i \geq 0$  \hfill  (induction hypothesis)\\
$\LRto$ $(\Hm', s') \models \phi$

\textbf{Case} $\phi = \EXIST [\psi_1 \UNTIL \psi_2]$:\\
%\textbf{Case} $\varphi = \MPE \FUTURE \psi$:
$\Hm, s \models \phi$ \\
$\LRto$ There is a path $\pi= (s=s_0, s_1, ...)$ such that there is $i \geq 0$ such that $(\Hm, s_i) \models \psi_2$, and for all $0 \leq j < i$, $(\Hm, s_j) \models \psi_1$\\
$\LRto$ There is a path $\pi' = (s=s_0', s_1', ...)$ such that $\pi \lrto_V \pi'$  \hfill  ($s \lrto_V s'$, Proposition~\ref{div})\\
$\LRto$ $(\Hm', s_i') \models \psi_2$, and for all $0 \leq j < i$ $(\Hm', s_j') \models \psi_1$   \hfill   (induction hypothesis)\\
$\LRto$ $(\Hm', s') \models \phi$
\end{proof}


\textbf{Proposition}~\ref{B_to_T}  Let $V\subseteq\cal A$ and $({\cal M}_i,s_i)~(i=1,2)$ be two \MPK-structures.
  Then
  \[(s_1,s_2)\in{\cal B}_n\mbox{ iff }
  \Tr_j(s_1)\lrto_V\Tr_j(s_2)\mbox{ for every $0\le j\le n$}.\]
\begin{proof}
We will prove this from two aspects:

$(\Rto)$ If $(s_1, s_2) \in \Hb_n$, then $Tr_j(s_1) \lrto_V Tr_j(s_2)$ for all $0 \leq j \leq n$. $(s, s') \in \Hb_n$ implies both roots of $Tr_n(s_1)$ and $Tr_n(s_2)$ have the same atoms except those atoms in $V$.
Besides, for any $s_{1,1}$ with $(s_1, s_{1,1}) \in R_1$, there is a $s_{2,1}$ with $(s_2, s_{2,1})\in R_2$ s.t. $(s_{1,1}, s_{2,1}) \in \Hb_{n-1}$ and vice versa.
Then we have $Tr_1(s_1) \lrto_V Tr_1(s_2)$.
Therefore,  $Tr_n(s_1) \lrto_V Tr_n(s_2)$ by use such method recursively, and then $Tr_j(s_1) \lrto_V Tr_j(s_2)$ for all $0 \leq j \leq n$.
%It is easy to prove this by the definition of

$(\Lto)$ If $Tr_j(s_1) \lrto_V Tr_j(s_2)$ for all $0\leq j \leq n$, then $(s_1, s_2) \in \Hb_n$.
$Tr_0(s_1) \lrto_V Tr_0(s_2)$ implies $L(s_1) - V = L'(s_2) - V$ and then $(s, s') \in \Hb_0$.
$Tr_1(s_1) \lrto_V Tr_1(s_2)$ implies $L(s_1) - V = L'(s_2)- V$ and for every successors $s$ of the root of one, it is possible to find a successor of the root of the other $s'$ such that
$(s, s')\in \Hb_0$. Therefore $(s_1, s_2) \in \Hb_1$, and then we will have $(s_1, s_2) \in \Hb_n$ by use such method recursively.
\end{proof}

\textbf{Proposition}~\ref{pro:k}   Let $V\subseteq \Ha$, $\Hm$ be a model structure and $s,s'\in S$
  such that $s\not\lrto_V s'$.
  There exists a least  $k$ such that
  $\Tr_k(s)$ and $\Tr_k(s')$ are not $V$-bisimilar.\\
\begin{proof}
If $s\not\lrto_V s'$, then there exists a least constant $k$ such that $(s_i, s_j) \notin \Hb_k$, and then there is a least constant m ($m \leq k$) such that $\Tr_m(s_i)$ and $\Tr_m(s_j)$ are not V-corresponding by Proposition~\ref{B_to_T}. Let $c=m$, the lemma is proved.
\end{proof}









\textbf{Lemma}\ref{lem:Vb:TrFormula:Equ} Let $V\subseteq \Ha$, $\Hm$ and $\Hm'$ be two model structures,
$s\in S$, $s'\in S'$ and $n\ge 0$. If $\Tr_n(s) \lrto_{\overline V} \Tr_n(s')$, then ${\cal F}_V(\Tr_n(s)) \equiv {\cal F}_V(\Tr_n(s'))$.\\
\begin{proof}
This result can be proved by inducting on $n$.

\textbf{Base.} It is apparent that for any $s_n\in S$ and $s_n' \in S'$, if $\Tr_0(s_n) \lrto_{\overline V} \Tr_0(s_n')$ then ${\cal F}_V(\Tr_0(s_n)) \equiv {\cal F}_V(\Tr_0(s_n'))$ due to $L(s_n) - \overline V = L'(s_n') - \overline V$ by known.

\textbf{Step.} Supposing that for $k=m$ $(0< m \leq n)$ there is if $\Tr_{n-k}(s_k) \lrto_{\overline V} \Tr_{n-k}(s_k')$ then ${\cal F}_V(\Tr_{n-k}(s_k)) \equiv {\cal F}_V(\Tr_{n-k}(s_k'))$, then we will show if $\Tr_{n-k+1}(s_{k-1}) \lrto_{\overline V} \Tr_{n-k+1}(s_{k-1}')$ then ${\cal F}_V(\Tr_{n-k+1}(s_{k-1})) \equiv {\cal F}_V(\Tr_{n-k+1}(s_{k-1}'))$. Apparent that:\\
 ${\cal F}_V(\Tr_{n-k+1}(s_{k-1})) =$
 $\left(\bigwedge_{(s_{k-1},s_k)\in R}
    \EXIST \NEXT {\cal F}_V(\Tr_{n-k}(s_k))\right)
    \wedge \ALL \NEXT\left(\bigvee_{(s_{k-1},s_k)\in R}
    {\cal F}_V(\Tr_{n-k}(s_k) )\right)
    \wedge {\cal F}_V(\Tr_0(s_{k-1}))$\\
 ${\cal F}_V(\Tr_{n-k+1}(s_{k-1}')) =$
 $\left(\bigwedge_{(s_{k-1}',s_k')\in R}
    \EXIST \NEXT {\cal F}_V(\Tr_{n-k}(s_k'))\right)
    \wedge \ALL \NEXT\left(\bigvee_{(s_{k-1}',s_k')\in R}
    {\cal F}_V(\Tr_{n-k}(s_k') )\right)
    \wedge {\cal F}_V(\Tr_0(s_{k-1}'))$ by the definition of characterizing formula of the computation tree.
 Then we have for any $(s_{k-1}, s_k) \in R$ there is $(s_{k-1}', s_k') \in R'$ such that $\Tr_{n-k}(s_k) \lrto_{\overline V} \Tr_{n-k}(s_k')$ by $\Tr_{n-k+1}(s_{k-1}) \lrto_{\overline V} \Tr_{n-k+1}(s_{k-1}')$. Besides, for any $(s_{k-1}', s_k') \in R'$ there is $(s_{k-1}, s_k) \in R$ such that $\Tr_{n-k}(s_k) \lrto_{\overline V} \Tr_{n-k}(s_k')$ by $\Tr_{n-k+1}(s_{k-1}) \lrto_{\overline V} \Tr_{n-k+1}(s_{k-1}')$.
 Therefore, we have ${\cal F}_V(\Tr_{n-k+1}(s_{k-1})) \equiv {\cal F}_V(\Tr_{n-k+1}(s_{k-1}'))$ by induction hypothesis.
\end{proof}



%\textbf{Lemma}~\ref{lem:models:formula} Let $\varphi$ be a formula. We have
%  \begin{equation}
%    \varphi\equiv \bigvee_{(\Hm, s_0)\in\Mod(\varphi)}{\cal F}_{\cal A}(\Hm, s_0).
%    \end{equation}
%\begin{proof}
%Let $(\Hm', s_0')$ be a model of $\varphi$. Then $(\Hm', s_0') \models \bigvee_{(\Hm, s_0)\in \Mod(\varphi)} {\cal F}_{\Ha}(\Hm, s_0)$ due to $(\Hm', s_0') \models {\cal F}_{\Ha}(\Hm', s_0')$. On the other hand, suppose that $(\Hm', s_0')$ is a model of $\bigvee_{(\Hm, s_0)\in \Mod(\varphi)} {\cal F}_{\Ha}(\Hm, s_0)$. Then there is a $(\Hm, s_0)\in \Mod(\varphi)$ such that $(\Hm', s_0') \models {\cal F}_{\Ha}(\Hm, s_0)$. And then $(\Hm, s_0) \lrto_{\O} (\Hm', s_0')$ by Theorem~\ref{CF}. Therefore, $(\Hm, s_0)$ is also a model of $\varphi$ by Theorem~\ref{thm:V-bisimulation:EQ}.
%\end{proof}


\textbf{Theorem}~\ref{CF}
Given $V\subseteq \Ha$, let $\Hm=(S,R,L,s_0)$  and $\Hm'=(S',R', L',s_0')$ be two model structures. Then,
\begin{enumerate}[(i)]
\item  $(\Hm',s_0') \models {\cal F}_V({\cal M},s_0)
\text{ iff } ({\cal M},s_0) \lrto_{\overline V} ({\cal M}',s_0')$;

\item  $s_0 \lrto_{\overline V} s_0'$ implies  ${\cal F}_V(\Hm, s_0) \equiv {\cal F}_V(\Hm', s_0')$.

\end{enumerate}

In order to prove Theorem~\ref{CF}, we prove the following two lemmas at first.

\begin{lemma}\label{Bn:to:Tn}
Let $V\subseteq \Ha$, $\Hm=(S, R, L,s_0)$ and $\Hm'=(S', R', L',s_0')$ be two model structures,
$s\in S$, $s'\in S'$ and $n\ge 0$.
\begin{enumerate}[(i)]
  \item $({\cal M},s)\models{\cal F}_V(\Tr_n(s))$.
  \item If $({\cal M},s)\models{\cal F}_V(\Tr_n(s'))$ then
  $\Tr_n(s) \lrto_{\overline V} \Tr_n(s')$.
\end{enumerate}
\end{lemma}
\begin{proof}
(i) It is apparent from the definition of ${\cal F}_V(\Tr_n(s))$.
Base. It is apparent that $({\cal M},s)\models {\cal F}_V(\Tr_0(s))$.\\
Step. For $k \geq 0$, supposing the result talked in (i) is correct in $k - 1$, we will show that $({\cal M},s)\models {\cal F}_V(\Tr_{k+1}(s))$, \ie:
\begin{equation*}
\resizebox{.91\linewidth}{!}{$
    \displaystyle
 %\[
 ({\cal M},s)\models \left(\bigwedge_{(s,s')\in R}
    \EXIST \NEXT T(s')\right)
    \wedge \ALL \NEXT\left(\bigvee_{(s,s')\in R}
    T(s')\right)
    \wedge {\cal F}_V(\Tr_0(s)).%\]
 $}
\end{equation*}
Where $T(s') ={\cal F}_V(\Tr_k(s'))$. It is apparent that $({\cal M},s)\models {\cal F}_V(\Tr_0(s))$ by Base. It is apparent that for any $(s,s') \in R$, there is $({\cal M}, s') \models {\cal F}_V(\Tr_k(s'))$ by inductive assumption. Then we have $({\cal M},s)\models \EXIST \NEXT {\cal F}_V(\Tr_k(s')$, and then $({\cal M},s)\models \left(\bigwedge_{(s,s')\in R}
    \EXIST \NEXT {\cal F}_V(\Tr_k(s'))\right)$. Similarly, we have that for any $(s,s') \in R$, there is $({\cal M}, s') \models \bigvee_{(s,s'')\in R}
    {\cal F}_V(\Tr_k(s'') )$. Therefore, $({\cal M},s)\models \ALL \NEXT\left(\bigvee_{(s,s'')\in R}
    {\cal F}_V(\Tr_k(s'') )\right)$.

(ii)  \textbf{Base}. If $n=0$, then $(\Hm, s)  \models {\cal F}_V(\Tr_0(s'))$ implies $L(s) - \overline V = L'(s') - \overline V$. Hence, $\Tr_0(s) \lrto_{\overline V} \Tr_0(s')$.\\
    \textbf{Step}. Supposing $n>0$ and the result talked in (ii) is correct in $n-1$.\\
   (a) It is easy to see that $L(s) - \overline V = L'(s') - \overline V$.\\
   (b) We will show that for each $(s, s_1) \in R$, there is a $(s', s_1') \in R'$ such that $\Tr_{n-1}(s_1) \lrto_{\overline V} \Tr_{n-1}(s_1')$.
      Since $(\Hm, s) \models {\cal F}_V(\Tr_n(s'))$, then $(\Hm, s) \models \ALL \NEXT\left(\bigvee_{(s',s_1')\in R}{\cal F}_V(\Tr_{n-1}(s_1') )\right)$.
      Therefore, for each $(s, s_1) \in R$ there is a $(s', s_1') \in R'$ such that $(\Hm, s_1) \models {\cal F}_V(\Tr_{n-1}(s_1') )$. Hence, $\Tr_{n-1}(s_1) \lrto_{\overline V} \Tr_{n-1}(s_1')$ by inductive hypothesis.\\
   (c) We will show that for each $(s',s_1')\in R'$ there is a $(s,s_1)\in R$ such that $\Tr_{n-1}(s_1') \lrto_{\overline V} \Tr_{n-1}(s_1)$.
      Since $(\Hm, s) \models {\cal F}_V(\Tr_n(s'))$, then $(\Hm, s) \models  \bigwedge_{(s',s_1')\in R'} \EXIST \NEXT {\cal F}_V(\Tr_{n-1}(s_1'))$.
      Therefore, for each $(s',s_1')\in R'$ there is a $(s,s_1)\in R$ such that $(\Hm, s_1) \models {\cal F}_V(\Tr_{n-1}(s_1')$.
      Hence, $\Tr_{n-1}(s_1) \lrto_{\overline V} \Tr_{n-1}(s_1')$ by inductive hypothesis.
\end{proof}


A consequence of the previous lemma is:

\begin{lemma}\label{div_s}
Let $V\subseteq \Ha$, $\Hm=(S,R,L,s_0)$ a model structure, $k={ch({\cal M},V)}$ and $s\in S$.
%There is a formula $\phi$ such that
\begin{itemize}
  \item $(\Hm, s)\models {\cal F}_V(\Tr_k(s))$, and
  \item for each $s'\in S$, $({\cal M},s) \lrto_{\overline V} ({\cal M},s')$
  if and only if $({\cal M},s')\models{\cal F}_V(\Tr_k(s))$.
\end{itemize}
\end{lemma}
\begin{proof}
Let $\phi = {\cal F}_V(\Tr_k(s))$, where $k$ is the V-characteristic number of $\Hm$. $(\Hm, s) \models \phi$ by the definition of ${\cal F}$, and then $\forall s' \in S$, if $s \lrto_{\overline V} s'$ there is $(\Hm, s') \models \phi$ by Theorem~\ref{thm:V-bisimulation:EQ} due to $\IR(\phi, \Ha - V)$. Supposing $(\Hm, s')\models \phi$, if $s \nleftrightarrow_{\overline V} s'$, then $\Tr_k(s) \not \lrto_{\overline V} \Tr_k(s')$, and then $(\Hm, s')\nvDash \phi$ by Lemma~\ref{Bn:to:Tn}, a contradiction.
\end{proof}




Now we are in the position of proving Theorem~\ref{CF}.\\
\begin{proof}
(i) Let ${\cal F}_V(\Hm, s_0)$ be the characterizing formula of $(\Hm, s_0)$ on $V$.
It is apparent that $\IR({\cal F}_V(\Hm, s_0), \overline V)$. We will show that $(\Hm, s_0) \models {\cal F}_V(\Hm, s_0)$ at first.

It is apparent that $(\Hm, s_0) \models {\cal F}_V(\Tr_c(s_0))$ by Lemma~\ref{Bn:to:Tn}.
We must show that $(\Hm, s_0) \models \bigwedge_{s\in S} G(\Hm, s)$.
Let ${\cal X} = {\cal F}_V(\Tr_c(s)) \rto \left(\bigwedge_{(s,s_1) \in R} \EXIST \NEXT {\cal F}_V(\Tr_c(s_1))\right)$ $\wedge \ALL \NEXT \left(\bigvee_{(s,s_1) \in R} {\cal F}_V(\Tr_c(s_1))\right)$, we will show $\forall s\in S$, $(\Hm, s_0) \models G(\Hm, s)$. Where $G(\Hm, s)=\ALL\GLOBAL \cal X$.
%Let $s_1, s_2, ..., s_m$ be the successors of $s$.
There are two cases we should consider:
\begin{itemize}
  \item  If $(\Hm, s_0) \nvDash {\cal F}_V(\Tr_c(s))$, it is apparent that $(\Hm, s_0) \models {\cal X}$;
  \item  If $(\Hm, s_0) \models {\cal F}_V(\Tr_c(s))$:\\
         $(\Hm, s_0) \models {\cal F}_V(\Tr_c(s))$\\
        $\Rto$  $s_0 \lrto_{\overline V} s$ by the definition of characteristic number and Lemma~\ref{div_s}.

        For each $(s, s_1)\in R$ there is:\\
         $(\Hm, s_1) \models {\cal F}_V(\Tr_c(s_1))$  \hfill  ($s_1 \lrto_{\overline V} s_1$)\\
        $\Rto$ $(\Hm, s) \models \bigwedge_{(s,s_1)\in R}\EXIST \NEXT {\cal F}_V(\Tr_c(s_1))$\\
        $\Rto$ $(\Hm, s_0) \models$ $\bigwedge_{(s,s_1)\in R}\EXIST \NEXT {\cal F}_V(\Tr_c(s_1))$    \qquad  (by $\IR(\bigwedge_{(s,s_1)\in R}\EXIST \NEXT {\cal F}_V(\Tr_c(s_1)), \overline V)$, $s_0 \lrto_{\overline V} s$).

         For each $(s, s_1)$ there is:\\
          $\Hm, s_1 \models \bigvee_{(s, s_2)\in R}{\cal F}_V(\Tr_c(s_2))$\\
        $\Rto$ $(\Hm, s) \models \ALL \NEXT \left( \bigvee_{(s, s_2)\in R} {\cal F}_V(\Tr_c(s_2)) \right)$ \\
        $\Rto$ $(\Hm, s_0) \models$  $\ALL \NEXT \left( \bigvee_{(s, s_2)\in R} {\cal F}_V(\Tr_c(s_2)) \right)$   \qquad  (by $\IR(\ALL \NEXT \left( \bigvee_{(s, s_2)\in R} {\cal F}_V(\Tr_c(s_2)) \right), \overline V)$, $s_0 \lrto_{\overline V} s$)\\
        $\Rto$ $(\Hm, s_0) \models {\cal X}$.\\
       % where $s_i$ and $s_j$ are the successors of $s$.
\end{itemize}
For any other states $s'$ which can reach from $s_0$ can be proved similarly, \ie, $(\Hm,s')\models \cal X$.
Therefore, $\forall s\in S$, $(\Hm, s_0) \models G(\Hm, s)$, and then $(\Hm, s_0) \models {\cal F}_V(\Hm, s_0)$.


We will prove this theorem from the following two aspects:

$(\Lto)$ If $s_0 \lrto_{\overline V} s_0'$, then $(\Hm',s_0') \models {\cal F}_V(M,s_0)$. Since $(\Hm, s_0) \models {\cal F}_V(\Hm, s_0)$ and $\IR({\cal F}_V(\Hm, s_0), \overline V)$, hence
$(\Hm',s_0') \models {\cal F}_V(M,s_0)$ by Theorem~\ref{thm:V-bisimulation:EQ}.

$(\Rto)$ If $(\Hm',s_0') \models {\cal F}_V(M,s_0)$, then $s_0 \lrto_{\overline V} s_0'$. We will prove this by showing that $\forall n \geq 0$, $Tr_n(s_0) \lrto_{\overline V} Tr_n(s_0')$.


\textbf{Base}. It is apparent that $Tr_0(s_0) \equiv Tr_0(s_0')$.

\textbf{Step}. Supposing $\Tr_k(s_0) \lrto_{\overline V} \Tr_k(s_0')$ ($k > 0$), we will prove $\Tr_{k+1}(s_0) \lrto_{\overline V} \Tr_{k+1}(s_0')$. We should only show that $\Tr_1(s_k) \lrto_{\overline V} \Tr_1(s_k')$. Where $(s_0, s_1), (s_1, s_2)$, $\dots$, $(s_{k-1}, s_k) \in R$ and $(s_0', s_1'), (s_1', s_2'), \dots, (s_{k-1}', s_k') \in R'$, \ie $s_{i+1}$ ($s_{i+1}'$) is an immediate successor of $s_i$ ($s_i'$) for all $0 \leq i \leq k-1$.

      (a) It is apparent that $L(s_k) - \overline V = L'(s_k') - \overline V$ by inductive assumption.

      Before talking about the other points, note the following fact that:\\
      $(\Hm',s_0') \models {\cal F}_V(\Hm,s_0)$\\
      $\Rto$ $\forall s'\in S'$, $(\Hm', s')\models {\cal F}_V(\Tr_c(s)) \rto$ $\left(\bigwedge_{(s,s_1)\in R} \EXIST \NEXT {\cal F}_V(\Tr_c(s_1))\right)$ $\wedge$ $\ALL \NEXT \left( \bigvee_{(s,s_1)\in R} {\cal F}_V(\Tr_c(s_1))\right)$  for any $s\in S$.   \hfill  \textbf{(fact)}\\
      (I) $(\Hm', s_0') \models {\cal F}_V(\Tr_c(s_0)) \rto \left(\bigwedge_{(s_0, s_1) \in R} \EXIST \NEXT {\cal F}_V(\Tr_c(s_1))\right)$ $\wedge$ $\ALL \NEXT \left(\bigvee_{(s_0, s_1) \in R} {\cal F}_V(\Tr_c(s_1)) \right)$     \hfill  \textbf{(fact)}\\
        (II) $(\Hm', s_0') \models {\cal F}_V(\Tr_c(s_0)))$  \hfill  (known)\\
        (III) $(\Hm', s_0') \models \left(\bigwedge_{(s_0, s_1) \in R} \EXIST \NEXT {\cal F}_V(\Tr_c(s_1))\right)$ $\wedge$ $\ALL \NEXT \left(\bigvee_{(s_0, s_1) \in R} {\cal F}_V(\Tr_c(s_1)) \right)$  \hfill  ((I),(II))\\

      % It is apparent that $L'(s_0') - \overline V = L(s_0) - \overline V$;\\
        (b) We will show that for each $(s_k, s_{k+1}) \in R$ there is a $(s_k', s_{k+1}') \in R'$ such that $L(s_{k+1}) - \overline V = L'(s_{k+1}') - \overline V$.\\
        (1) $(\Hm', s_0') \models \bigwedge_{(s_0, s_1) \in R} \EXIST \NEXT {\cal F}_V(\Tr_c(s_1))$  \hfill  (III)\\
        (2) $\forall (s_0, s_1) \in R$, $\exists (s_0', s_1') \in R'$ \st\ $(\Hm', s_1') \models {\cal F}_V(\Tr_c(s_1))$  \hfill  (2)\\
        (3) $\Tr_c(s_1) \lrto_{\overline V} \Tr_c(s_1')$  \hfill  ((2), Lemma~\ref{Bn:to:Tn}) \\
        (4) $L(s_1) - \overline V = L'(s_1') - \overline V$  \hfill   ((3), $c \geq 0)$\\
        (5) $(\Hm', s_1') \models {\cal F}_V(\Tr_c(s_1)) \rto \left(\bigwedge_{(s_1,s_2)\in R} \EXIST \NEXT {\cal F}_V(\Tr_c(s_2))\right) \wedge \ALL \NEXT \left(\bigvee_{(s_1,s_2)\in R} {\cal F}_V(\Tr_c(s_2))\right)$     \hfill  \textbf{(fact)}\\
        (6) $(\Hm', s_1') \models \left(\bigwedge_{(s_1,s_2)\in R} \EXIST \NEXT {\cal F}_V(\Tr_c(s_2))\right) \wedge \ALL \NEXT \left(\bigvee_{(s_1,s_2)\in R} {\cal F}_V(\Tr_c(s_2))\right)$ \hfill ((2), (5))\\
        (7) $\dots \dots$ \\
        (8) $(\Hm', s_k') \models \left(\bigwedge_{(s_k,s_{k+1})\in R} \EXIST \NEXT {\cal F}_V(\Tr_c(s_{k+1}))\right) \wedge \ALL \NEXT \left(\bigvee_{(s_k,s_{k+1})\in R} {\cal F}_V(\Tr_c(s_{k+1}))\right)$       \hfill (similar with (6))\\
        (9) $\forall (s_k, s_{k+1}) \in R$, $\exists (s_k', s_{k+1}') \in R'$ \st\ $(\Hm', s_{k+1}') \models {\cal F}_V(\Tr_c(s_{k+1}))$  \hfill  (8)\\
        (10) $\Tr_c(s_{k+1}) \lrto_{\overline V} \Tr_c(s_{k+1}')$    \hfill ((9), Lemma~\ref{Bn:to:Tn}) \\
        (11) $L(s_{k+1}) - \overline V = L'(s_{k+1}') - \overline V$  \hfill   ((10), $c \geq 0)$\\

        (c) We will show that for each $(s_k', s_{k+1}') \in R'$ there is a $(s_k, s_{k+1})\in R$ such that $L(s_{k+1}) - \overline V = L'(s_{k+1}') - \overline V$.\\
        (1) $(\Hm', s_k') \models \ALL \NEXT \left(\bigvee_{(s_k,s_{k+1})\in R} {\cal F}_V(\Tr_c(s_{k+1}))\right)$  \hfill (by (8) talked above)\\
        (2) $\forall (s_k', s_{k+1}') \in R'$, $\exists (s_k, s_{k+1}) \in R$ \st\ $(\Hm', s_{k+1}') \models {\cal F}_V(\Tr_c(s_{k+1}'))$  \hfill (1) \\
        (3) $\Tr_c(s_{k+1}) \lrto_{\overline V} \Tr_c(s_{k+1}')$    \hfill ((2), Lemma~\ref{Bn:to:Tn}) \\
        (4) $L(s_{k+1}) - \overline V = L'(s_{k+1}') - \overline V$  \hfill   ((3), $c \geq 0)$\\

(ii) This is following Lemma~\ref{lem:Vb:TrFormula:Equ} and the definition of the characterizing formula of initial \MPK-structure ${\cal K}$ on $V$.

\end{proof}


\textbf{Lemma}~\ref{lem:models:formula} Let $\varphi$ be a formula. We have
  \begin{equation}
    \varphi\equiv \bigvee_{(\Hm, s_0)\in\Mod(\varphi)}{\cal F}_{\cal A}(\Hm, s_0).
    \end{equation}
\begin{proof}
Let $(\Hm', s_0')$ be a model of $\varphi$. Then $(\Hm', s_0') \models \bigvee_{(\Hm, s_0)\in \Mod(\varphi)} {\cal F}_{\Ha}(\Hm, s_0)$ due to $(\Hm', s_0') \models {\cal F}_{\Ha}(\Hm', s_0')$. On the other hand, suppose that $(\Hm', s_0')$ is a model of $\bigvee_{(\Hm, s_0)\in \Mod(\varphi)} {\cal F}_{\Ha}(\Hm, s_0)$. Then there is a $(\Hm, s_0)\in \Mod(\varphi)$ such that $(\Hm', s_0') \models {\cal F}_{\Ha}(\Hm, s_0)$. And then $(\Hm, s_0) \lrto_{\O} (\Hm', s_0')$ by Theorem~\ref{CF}. Therefore, $(\Hm, s_0)$ is also a model of $\varphi$ by Theorem~\ref{thm:V-bisimulation:EQ}.
\end{proof}

%\textbf{Theorem}\ref{thm:VBChFEQ} Let $V\subseteq \Ha$, $\Hm=(S,R,L,s_0)$ a model structure with initial state $s_0$
%and $\Hm'=(S',R', L',s_0')$ a model structure with initial state $s_0'$.
%If $({\cal M},s_0) \lrto_{\overline V} ({\cal M}',s_0')$ then ${\cal F}_V(\Hm, s_0) \equiv {\cal F}_V(\Hm', s_0')$.\\



\textbf{Theorem}~\ref{thm:close}
(Representation theorem).
Let $\varphi$, $\varphi'$ and $\phi$ be \CTL\ formulas and $V \subseteq \Ha$.
Then the following statements are equivalent:
\begin{enumerate}[(i)]
  \item $\varphi' \equiv \CTLforget(\varphi, V)$,
  \item $\varphi'\equiv \{\phi | \varphi \models \phi \text{ and } \IR(\phi, V)\}$,
  \item Postulates (\W), (\PP), (\NgP) and (\textbf{IR}) hold.
\end{enumerate}
\begin{proof}
$(i) \LRto (ii)$. To prove this, we will show that:
\begin{align*}
 & \Mod(\CTLforget(\varphi, V)) = \Mod(\{\phi | \varphi \models \phi, \IR(\phi, V)\})\\
 & = \Mod(\bigvee_{\Hm, s_0\in \Mod(\varphi)} {\cal F}_{\Ha- V}(\Hm, s_0)).
\end{align*}
Firstly, suppose that $(\Hm', s_0')$ is a model of $\CTLforget(\varphi, V)$. Then there exists an  an initial \MPK-structure $(\Hm, s_0)$ such that $(\Hm, s_0)$ is a model of $\varphi$ and $(\Hm, s_0) \lrto_V (\Hm', s_0')$. By Theorem~\ref{thm:V-bisimulation:EQ}, we have $(\Hm', s_0') \models \phi$ for all $\phi$ that $\varphi\models \phi$ and $\IR(\phi, V)$. Thus, $(\Hm', s_0')$ is a model of $\{\phi | \varphi \models \phi, \IR(\phi, V)\}$.

Secondly, suppose that $(\Hm', s_0')$ is a models of $\{\phi | \varphi \models \phi, \IR(\phi, V)\}$. Thus, $(\Hm', s_0')$ $\models$ $\bigvee_{(\Hm, s_0)\in \Mod(\varphi)} {\cal F}_{\Ha- V}(\Hm, s_0)$ due to $\bigvee_{(\Hm, s_0)\in \Mod(\varphi)} {\cal F}_{\Ha- V}(\Hm, s_0)$ is irrelevant to $V$.

Finally, suppose that $(\Hm', s_0')$ is a model of $\bigvee_{\Hm, s_0\in \Mod(\varphi)} {\cal F}_{\Ha- V}(\Hm, s_0)$. Then there exists $(\Hm, s_0) \in \Mod(\varphi)$ such that $(\Hm', s_0') \models {\cal F}_{\Ha- V}(\Hm, s_0)$. Hence, $(\Hm, s_0)$ $\lrto_V$ $(\Hm', s_0')$ by Theorem~\ref{CF}. Thus $(\Hm', s_0')$ is also a model of $\CTLforget(\varphi,V)$.


$(ii)\Rto (iii)$. It is not difficult to prove it.

$(iii)\Rto (ii)$. Suppose that all postulates hold. By Positive Persistence, we have $\varphi' \models \{\phi | \varphi \models \phi, \IR(\phi, V)\}$. Now
we show that $\{\phi | \varphi \models \phi, \IR(\phi, V)\} \models \varphi'$. Otherwise, there exists formula $\phi'$ such that $\varphi' \models \phi'$ but $\{\phi | \varphi \models \phi, \IR(\phi, V)\} \nvDash \phi'$. There are three cases:
\begin{itemize}
  \item $\phi'$ is relevant to $V$. Thus, $\varphi'$ is also relevant to $V$, a contradiction to Irrelevance.
  \item $\phi'$ is irrelevant to $V$ and $\varphi \models \phi'$. This contradicts to our assumption.
  \item $\phi'$ is irrelevant to $V$ and $\varphi \nvDash \phi'$. By Negative Persistence, $\varphi' \nvDash \phi'$, a contradiction.
\end{itemize}
Thus, $\varphi'$ is equivalent to $\{\phi | \varphi \models \phi, \IR(\phi, V)\}$.
\end{proof}


%\begin{lemma}\label{lem:KF:eq}
%	Let $\varphi$ and $\alpha$ be two \CTL\ formulae and $q\in
%		\overline{\Var(\varphi\cup\{\alpha\})}$. Then
%	$\forget(\varphi \cup\{q\lrto\alpha\}, q)\equiv \varphi$.
%\end{lemma}
  \textbf{Lemma}~\ref{lem:KF:eq} Let $\varphi$ and $\alpha$ be two \CTL\ formulae and $q\in
		\overline{\Var(\varphi\cup\{\alpha\})}$. Then
	$\forget(\varphi \cup\{q\lrto\alpha\}, q)\equiv \varphi$.\\
    \begin{proof}
	Let $\varphi' =\varphi \cup\{q\lrto\alpha\}$. For any model $({\cal M},s)$ of $\CTLforget(\Gamma', q)$ there is an initial \MPK-structure $({\cal M}',s')$ s.t.\ $({\cal M},s)\lrto_{\{q\}}({\cal M}',s')$ and $({\cal M}',s') \models \varphi'$. It's apparent that $({\cal M}',s') \models \varphi$, and then $({\cal M},s) \models \varphi$ since $\IR(\varphi,\{q\})$ and $({\cal M},s)\lrto_{\{q\}}({\cal M}',s')$
	by Theorem~\ref{thm:V-bisimulation:EQ}.

	Let $(\Hm,s) \in \Mod(\varphi)$ with ${\cal M}=(S, R, L,s)$. We construct $(\Hm', s)$ with $\Hm' = (S, R, L',s)$ as follows:
    \begin{align*}
       & L':S \rto \Ha\ and\ \forall s^*\in S, L'(s^*) = L(s^*)\ if\ (\Hm, s^*) \nvDash \alpha,\\
       & else\ L'(s^*) = L(s^*)\cup\{q\}, \\
       & L'(s) = L(s) \cup\{q\}\ if\ (\Hm, s) \models \alpha,\ and\ L'(s) = L(s)\\
       & otherwise.
    \end{align*}
	It is clear that $({\cal M}',s) \models \varphi$, $({\cal M}',s) \models q\lrto \alpha$ and
	$({\cal M}', s) \lrto_{\{q\}} ({\cal M}, s)$. Therefore $({\cal M}', s) \models \varphi \cup\{q\lrto \alpha\}$, and then $({\cal M}, s) \models \CTLforget (\varphi \cup\{q\lrto\alpha\}, q)$ by
	$({\cal M}', s) \lrto_{\{q\}} ({\cal M}, s)$.
\end{proof}


\textbf{Proposition}~\ref{disTF} Let $\varphi$ be a formula, $V$ a set of atoms and $p$ an atom such that $p \notin V$. Then:
\[
\CTLforget(\varphi, \{p\} \cup V) \equiv \CTLforget(\CTLforget(\varphi, p), V).
\]
\begin{proof}
Let $(\Hm_1, s_1) $ with ${\cal M}_1=(S_1, R_1, L_1,s_1)$ be a model of $\CTLforget(\varphi, \{p\} \cup V)$. By the definition, there exists a model $(\Hm,s)$ with ${\cal M} = (S, R,L,s)$ of $\varphi$, such that $(\Hm_1, s_1)$ $\lrto_{\{p\} \cup V}$ $(\Hm, s)$. We construct an initial \MPK-structure $(\Hm_2, s_2)$ with ${\cal M}_2 = (S_2, R_2, L_2,s_2)$ as follows:
\begin{enumerate}[(1)]
  \item for $s_2$: let $s_2$ be the state such that:
  \begin{itemize}
    \item $p \in L_2(s_2)$ iff $p \in L_1(s_1)$,
    \item for all $q \in V$, $q \in L_2(s_2)$ iff $q\in L(s)$,
    \item for all other atoms $q'$, $q' \in L_2(s_2)$ iff $q' \in L_1(s_1)$ iff $q'\in L(s)$.
  \end{itemize}
  \item for another:
  \begin{enumerate}[(i)]
    \item for all pairs  $w \in S$ and $w_1 \in S_1$ such that $w \lrto_{\{p\} \cup V} w_1$, let $w_2 \in S_2$ and
        \begin{itemize}
          \item $p \in L_2(w_2)$ iff $p \in L_1(w_1)$,
          \item for all $q \in V$, $q \in L_2(w_2)$ iff $q\in L(w)$,
          \item for all other atoms $q'$, $q' \in L_2(w_2)$ iff $q' \in L_1(w_1)$ iff $q'\in L(w)$.
        \end{itemize}
    \item if $(w_1', w_1)\in R_1$, $w_2$ is constructed based on $w_1$ and $w_2'\in S_2$ is constructed based on $w_1'$, then $(w_2', w_2)\in R_2$.
     %And if $w' \Hr^i w$, $w_2$ is constructed based on $w$ and $w_2'\in \Hw_2$ is constructed based on $w'$, then $w_2' \Hr_2^i w_2$
    %\item if $\exists w_1'\in \Hw_1$ such that $w_1' \Hr_1 w_1$, then let $w_2' \in \Hw_2$, $w_2' \Hr_2 w_2$, and if $w_1' \neq s_1$ then do (i) for $w_2'$, else let$w_2' = s_2$.
  \end{enumerate}
  \item delete duplicated states in $S_2$ and pairs in $R_2$.
\end{enumerate}
Then we have $(\Hm, s) \lrto_{\{p\}} (\Hm_2, s_2)$ and $(\Hm_2, s_2) \lrto_V (\Hm_1, s_1)$. Thus, $(\Hm_2, s_2) \models \CTLforget(\varphi, p)$. And therefore $(\Hm_1, s_1) \models \CTLforget(\CTLforget(\varphi, p), V)$.

On the other hand, suppose that $(\Hm_1, s_1)$ be a model of $\CTLforget(\CTLforget(\varphi, p), V)$, then there exists an initial \MPK-structure $(\Hm_2, s_2)$ such that $(\Hm_2, s_2) \models \CTLforget(\varphi, p)$ and $(\Hm_2, s_2) \lrto_V (\Hm_1, s_1)$, and there exists $(\Hm, s)$ such that $(\Hm, s) \models \varphi$ and $(\Hm, s) \lrto_{\{p\}} (\Hm_2, s_2)$. Therefore, $(\Hm, s) \lrto_{\{p\} \cup V} (\Hm_1, s_1)$ by Proposition~\ref{div}, and consequently, $(\Hm_1, s_1) \models \CTLforget(\varphi, \{p\} \cup V)$.
\end{proof}



\textbf{Proposition}~\ref{pro:ctl:forget:1}
Let $\varphi$, $\varphi_i$, $\psi_i$ ($i=1,2$) be formulas and $V\subseteq \Ha$. We have
\begin{enumerate}[(i)]
  \item $\CTLforget(\varphi, V)$ is satisfiable iff $\varphi$ is;
  \item If $\varphi_1 \equiv \varphi_2$, then $\CTLforget(\varphi_1, V) \equiv \CTLforget(\varphi_2, V)$;
  \item If $\varphi_1 \models \varphi_2$, then $\CTLforget(\varphi_1, V) \models \CTLforget(\varphi_2, V)$;
  \item $\CTLforget(\psi_1 \vee \psi_2, V) \equiv \CTLforget(\psi_1, V) \vee \CTLforget(\psi_2, V)$;
  \item $\CTLforget(\psi_1 \wedge \psi_2, V) \models \CTLforget(\psi_1, V) \wedge \CTLforget(\psi_2, V)$;
 % \item If $\IR(\psi_1, V)$, then $\CTLforget(\varphi \wedge \psi_1, V) \equiv \CTLforget(\varphi, V) \wedge \psi_1$.
\end{enumerate}
\begin{proof}
(i) ($\Rto$) Supposing $(\Hm, s)$ is a model of $\CTLforget(\varphi, V)$, then there is a model $(\Hm',s')$ of $\varphi$ s.t. $(\Hm,s) \lrto_V (\Hm',s')$ by the definition of $\CTLforget$.

($\Lto$) Supposing $(\Hm, s)$ is a model of $\varphi$, then there is an initial Kripke structure $(\Hm',s')$ s.t. $(\Hm,s) \lrto_V (\Hm',s')$, and then $(\Hm',s') \models \CTLforget(\varphi, V)$ by the definition of $\CTLforget$.

The (ii) and (iii) can be proved similarly.

(iv) ($\Rto$) $\forall (\Hm,s) \in \Mod(\CTLforget(\psi_1 \vee \psi_2, V))$, $\exists (\Hm',s') \in \Mod(\psi_1\vee \psi_2)$ s.t. $(\Hm,s) \lrto_V (\Hm',s')$ and $(\Hm',s') \models \psi_1$ or $(\Hm',s') \models \psi_2$ \\
$\Rto$ $\exists (\Hm_1,s_1) \in \Mod(\CTLforget(\psi_1, V))$ s.t. $(\Hm',s') \lrto_V (\Hm_1,s_1)$ or $\exists (\Hm_2,s_2) \in \Mod(\CTLforget(\psi_2, V))$ s.t. $(\Hm',s') \lrto_V (\Hm_2,s_2)$ \\
%$\Rto$ $(\Hm,s) \lrto_V (\Hm_1,s_1)$ or $(\Hm,s) \lrto_V (\Hm_2,s_2)$\\
$\Rto$ $(\Hm,s) \models \CTLforget(\psi_1, V) \vee \CTLforget(\psi_2, V)$ by Theorem~\ref{thm:V-bisimulation:EQ}.

($\Lto$) $\forall (\Hm,s) \in \Mod(\CTLforget(\psi_1, V) \vee \CTLforget(\psi_2, V))$\\
$\Rto$ $(\Hm,s) \models \CTLforget(\psi_1,V)$ or $(\Hm,s) \models \CTLforget(\psi_2,V)$\\
$\Rto$ there is an initial \MPK-structure $(\Hm_1,s_1)$ s.t. $(\Hm,s) \lrto_V (\Hm_1,s_1)$ and $(\Hm_1,s_1) \models \psi_1$ or  $(\Hm_1,s_1) \models \psi_2$\\
$\Rto$ $(\Hm_1,s_1) \models \psi_1 \vee \psi_2$\\
$\Rto$ there is an initial \MPK-structure $(\Hm_2,s_2)$ s.t. $(\Hm_1,s_1) \lrto_V (\Hm_2,s_2)$ and $(\Hm_2,s_2) \models \CTLforget(\psi_1 \vee \psi_2, V)$\\
$\Rto$ $(\Hm,s) \lrto_V (\Hm_2,s_2)$ and $(\Hm,s) \models \CTLforget(\psi_1 \vee \psi_2, V)$.

The (v) can be proved as (iv).
\end{proof}



\textbf{Proposition}~\ref{pro:ctl:forget:2}
Let $V\subseteq\cal A$ and $\phi$ a formula.% and $Q\in \{\EXIST, \ALL\}$.
  \begin{enumerate}[(i)]
    \item $\CTLforget(\ALL\NEXT\phi,V)\equiv \ALL\NEXT \CTLforget(\phi,V)$.
    \item $\CTLforget(\EXIST\NEXT\phi,V)\equiv\EXIST\NEXT \CTLforget(\phi,V)$.
    \item $\CTLforget(\ALL \FUTURE\phi,V)\equiv \ALL \FUTURE \CTLforget(\phi,V)$.
    \item $\CTLforget(\EXIST\FUTURE\phi,V)\equiv\EXIST\FUTURE \CTLforget(\phi,V)$.
  \end{enumerate}
\begin{proof}
Let $\Hm=(S, R, L,s_0)$ with initial state $s_0$ and $\Hm'=(S', R', L',s_0')$ with initial state $s_0'$, then we call $\Hm', s_0'$ be a sub-structure of $\Hm,s_0$ if:
\begin{itemize}
  \item $S'=\{s' | s'$ is reachable from $s_0'\}$ and $S' \subseteq S$,
  \item $R' =\{(s_1, s_2)| s_1, s_2 \in S'$ and $(s_1, s_2) \in R\}$,
  \item $L': S' \rto \Ha$ and $\forall s_1 \in S'$ there is $L'(s_1) = L(s_1)$, and
  \item there is a stete $s\in S$ reachable from $s_0$ such that $(\Hm, s) \lrto_{{\O}} (\Hm', s_0')$.
\end{itemize}

(i) In order to prove $\CTLforget(\ALL \NEXT \phi, V) \equiv \ALL \NEXT(\CTLforget(\phi, V))$, we only need to prove $\Mod(\CTLforget(\ALL \NEXT \phi, V)) = \Mod( \ALL\NEXT\CTLforget(\phi, V))$:

$(\Rto)$ $\forall (\Hm', s') \in \Mod(\CTLforget(\ALL \NEXT \phi, V))$ there exists an initial \MPK-structure $(\Hm, s)$ s.t. $(\Hm, s)\models \ALL \NEXT \phi$ and $(\Hm, s) \lrto_V (\Hm',s')$\\
$\Rto$ for any sub-structure $(\Hm_1, s_1)$ of $(\Hm, s)$ there is $(\Hm_1, s_1) \models \phi$, where $s_1$ is a directed successor of $s$ \\
$\Rto$ there is an initial \MPK-structure $(\Hm_2, s_2)$ s.t. $(\Hm_2, s_2) \models \CTLforget(\phi,V)$ and $(\Hm_2, s_2) \lrto_V (\Hm_1,s_1)$\\
$\Rto$ it is easy to construct an initial \MPK-structure $(\Hm_3, s_3)$ by $(\Hm_2, s_2)$ s.t. $(\Hm_2, s_2)$ is a sub-structure of $(\Hm_3, s_3)$ that $s_2$ is a direct successor of $s_3$ and $(\Hm_3, s_3) \lrto_V (\Hm,s)$\\
$\Rto$ $\Hm_3, s_3 \models \ALL \NEXT (\CTLforget(\phi,V))$, especially, let $\Hm_3, s_3 = \Hm', s'$, we have $\Hm', s' \models \ALL \NEXT (\CTLforget(\phi,V))$.

$(\Lto)$ $\forall$ $(\Hm_3, s_3) \in \Mod(\ALL \NEXT (\CTLforget(\phi,V)))$, then for any sub-structure $(\Hm_2, s_2)$ whit $s_2$ is a directed successor $s_3$ of $(\Hm_3, s_3)$ s.t. $(\Hm_2, s_2) \models \CTLforget(\phi,V)$\\
$\Rto$ there is an initial \MPK-structure $(\Hm_1, s_1)$ s.t. $(\Hm_1, s_1) \models \phi$ and $(\Hm_1, s_1) \lrto_V (\Hm_2, s_2)$\\
$\Rto$ it is easy to construct an initial structure $(\Hm,s)$ by $(\Hm_1, s_1)$ s.t. $(\Hm_1, s_1)$ is a sub-structure of $(\Hm, s)$ that $s_1$ is a direct successor of $s$ and $(\Hm, s)\lrto_V (\Hm_3, s_3)$\\
$\Rto$ $(\Hm, s) \models \ALL \NEXT \phi$ and then $(\Hm_3, s_3) \models \CTLforget(\ALL \NEXT \phi, V)$.


(ii) In order to prove $\CTLforget(\EXIST \NEXT \phi, V) \equiv \EXIST\NEXT\CTLforget(\phi, V)$, we only need to prove $\Mod$ $(\CTLforget(\EXIST \NEXT \phi$, $V)) = \Mod( \EXIST\NEXT\CTLforget(\phi, V))$:

$(\Rto)$ $\forall \Hm', s' \in \Mod(\CTLforget(\EXIST \NEXT \phi, V))$ there exists an initial \MPK-structure $(\Hm, s)$ s.t. $(\Hm, s) \models \EXIST \NEXT \phi$ and $(\Hm, s) \lrto_V (\Hm',s')$\\
$\Rto$ there is a sub-structure $(\Hm_1, s_1)$ of $(\Hm, s)$ s.t. $(\Hm_1, s_1) \models \phi$, where $s_1$ is a directed successor of $s$\\
$\Rto$ there is an initial \MPK-structure $(\Hm_2, s_2)$ s.t. $(\Hm_2, s_2) \models \CTLforget(\phi,V)$ and $(\Hm_2, s_2) \lrto_V (\Hm_1,s_1)$\\
$\Rto$ it is easy to construct an initial \MPK-structure $(\Hm_3, s_3)$ by $(\Hm_2, s_2)$ s.t. $(\Hm_2, s_2)$ is a sub-structure of $(\Hm_3, s_3)$ that $s_2$ is a direct successor of $s_3$ and $(\Hm_3, s_3) \lrto_V (\Hm,s)$\\
$\Rto$ $(\Hm_3, s_3) \models \EXIST \NEXT (\CTLforget(\phi,V))$, especially, let $(\Hm_3, s_3) = (\Hm', s')$, we have $(\Hm', s') \models \EXIST \NEXT (\CTLforget(\phi,V))$.

$(\Lto)$ $\forall$ $(\Hm_3, s_3) \in \Mod(\EXIST \NEXT (\CTLforget(\phi,V)))$, then there exists a sub-structure $(\Hm_2, s_2)$ of $(\Hm_3, s_3)$ s.t. $(\Hm_2, s_2) \models \CTLforget(\phi,V)$\\
$\Rto$ there is an initial \MPK-structure $(\Hm_1, s_1)$ s.t. $(\Hm_1, s_1) \models \phi$ and $(\Hm_1, s_1) \lrto_V (\Hm_2, s_2)$\\
$\Rto$ it is easy to construct an initial \MPK-structure $(\Hm,s)$ by $(\Hm_1, s_1)$ s.t. $(\Hm_1, s_1)$ is a sub-structure of $(\Hm, s)$ that $s_1$ is a direct successor of $s$ and $(\Hm, s)\lrto_V (\Hm_3, s_3)$\\
$\Rto$ $(\Hm, s) \models \EXIST \NEXT \phi$ and then $(\Hm_3, s_3) \models \CTLforget(\EXIST \NEXT \phi, V)$.



(iii) and (iV) can be proved as (i) and (ii) respectively.
\end{proof}



\textbf{Proposition}\ref{modelChecking} Let $(\Hm,s_0)$ be an initial \MPK-structure, $\varphi$ be a \CTL\ formula and $V$ a set of atoms. Deciding whether $(\Hm,s_0)$ is a model of $\forget(\varphi, V )$ is NP-complete.\\
\begin{proof}
The problem can be determined by the following two things: (1) guessing
an initial \MPK-structure $(\Hm',s_0')$ satisfying $\varphi$; and
(2) checking if  $(\Hm, s_0) \leftrightarrow_V (\Hm', s_0')$. Both two steps can be done in polynomial time.
 Hence, the problem is in NP.
The hardness follows that the model checking for propositional variable
forgetting is NP-hard~\cite{Zhang2008Properties}.
\end{proof}

\textbf{Theorem}~\ref{thm:comF}
Let $\varphi$ and $\psi$ be two $\CTL_{\ALL \FUTURE}$ (a fragment of \CTL, in which each formula contains only $\ALL \FUTURE$ temporal connective) formulas and $V$ a set of atoms. Then we have the
results:
\begin{enumerate}[(i)]
  \item deciding  $\forget(\varphi, V ) \models^? \psi$ is co-NP-complete,
  \item deciding  $\psi \models^? \forget(\varphi, V)$ is $\Pi_2^P$-complete,
  \item deciding $\forget(\varphi, V) \models^? \forget(\psi, V)$ is $\Pi_2^P$-complete.
\end{enumerate}
\begin{proof}
(1) It is proved that deciding whether $\psi$ is satisfiable is NP-Complete~\cite{DBLP:journals/ijfcs/MeierTVM15}. The hardness is easy to see by setting $\forget(\varphi, \Var(\varphi))\equiv \top$, \ie deciding whether $\psi$ is valid.
For membership, from Theorem
3, we have $\forget(\varphi, V ) \models \psi$ iff $\varphi \models \psi$ and $IR(\psi, V )$.
Clearly, in $\CTL_{\ALL \FUTURE}$, deciding $\varphi\models \psi$ is in co-NP. We show that deciding whether $IR(\psi, V )$ is also
in co-NP. Without loss of generality, we assume that $\psi$ is satisfiable.
 %Then $\psi$ has a model in the polynomial size of $\psi$.
 We consider the complement of the problem: deciding whether $\psi$ is not irrelevant to $V$. It is easy to see that $\psi$ is
not irrelevant to $V$ iff there exist a model $(\Hm, s_0)$ of $\psi$ and an
initial \MPK-structure $(\Hm',s_0')$  such that
$(\Hm, s_0) \leftrightarrow_V (\Hm',s_0')$ and $(\Hm',s_0')\nvDash \psi$. So checking whether $\psi$ is not irrelevant to $V$ can be achieved in the following steps: (1) guess two initial \MPK-structures $(\Hm,s_0)$ and $(\Hm',s_0')$, (2) check if $(\Hm,s_0) \models \psi$ and $(\Hm',s_0')\nvDash \psi$, and (3) check
$(\Hm, s_0) \leftrightarrow_V (\Hm',s_0')$. Obviously (1) can be done in polynomial time and also (2) and (3) can be done in polynomial time.

(2) Membership. We consider the complement of the
problem. We may guess an initial \MPK-structure $(\Hm, s_0)$ and check whether $(\Hm,s_0) \models \psi$ and $(\Hm,s_0)$ $\nvDash \forget($ $\varphi$, $V)$. From Proposition~\ref{modelChecking}, we know that this is in $\Sigma_2^P$. So the original problem is in $\Pi_2^P$. Hardness. Let $\psi \equiv \top$. Then the problem is reduced to decide $\forget(\varphi, V )$'s validity. Since a propositional variable forgetting is a special case temporal forgetting, the hardness is directly followed from the proof of Proposition 24 in~\cite{DBLP:journals/jair/LangLM03}.

(3) Membership. If $\forget(\varphi, V) \nvDash \forget(\psi, V)$ then there exist an initial \MPK-structure $(\Hm, s)$ such that $(\Hm, s)\models \forget(\varphi, V)$ but $(\Hm, s) \nvDash \forget(\psi, V)$, \ie, there is $(\Hm_1, s_1) \lrto_V (\Hm, s)$ with $(\Hm_1, s_1) \models \varphi$ but $(\Hm_2, s_2) \nvDash \psi$ for every $(\Hm_2, s_2)$ with $(\Hm, s) \lrto_V (\Hm_2, s_2)$. It is evident that guessing such $(\Hm, s)$, $(\Hm_1, s_1)$ with $(\Hm_1, s_1) \lrto_V (\Hm, s)$ and checking $(\Hm_1, s_1)\models \varphi$ are feasible while checking $(\Hm_2, s_2) \nvDash \psi$ for every $(\Hm, s) \lrto_V (\Hm_2, s_2)$ can be done in polynomial time. Thus the problem is in $\Pi_2^P$.

Hardness. It follows from (2) due to the fact that $\forget(\varphi, V) \models \forget(\psi, V)$ iff $\varphi \models \forget(\psi, V)$ thanks to $IR(\forget(\psi, V), V)$.

\end{proof}




\textbf{Proposition}~\ref{dual} Let $V,q,\varphi$ and $\psi$ are the ones in Definition~\ref{def:NC:SC}.
 The $\psi$ is a SNC (WSC) of $q$ on $V$ under $\varphi$ iff $\neg \psi$ is a WSC (SNC)
    of $\neg q$ on $V$ under $\varphi$.\\
\begin{proof}
     (i) Suppose $\psi$ is the SNC of $q$. Then $\varphi \models q \rto \psi$. Thus $\varphi \models \neg \psi \rto \neg q$. So $\neg \psi$ is a
SC of $\neg q$. Suppose $\psi'$ is any other SC of $\neg q$: $\varphi \models \psi' \rto \neg q$. Then $\varphi \models q \rto \neg \psi'$, this means $\neg \psi'$ is a NC of $q$ on $P$ under $\varphi$.
Thus $\varphi \models \psi \rto \neg \psi'$ by assumption. So $\varphi \models \psi' \rto \neg \psi$. This proves that $\neg \psi$ is the WSC of $\neg q$.
The proof of the other part of the proposition is similar.

(ii) The WSC case can be proved similarly with SNC case.
    \end{proof}




\textbf{Proposition}~\ref{formulaNS_to_p}   Let $\Gamma$ and $\alpha$ be two formulas, $V \subseteq \Var(\alpha) \cup \Var(\phi)$  and $q$ is a new proposition not in $\Gamma$ and $\alpha$. Then, a formula $\varphi$ of $V$ is the SNC (WSC) of $\alpha$ on $V$ under  $\Gamma$ iff it is the SNC (WSC) of $q$ on $V$ under $\Gamma' = \Gamma \cup \{q \equiv \alpha\}$.

\begin{proof}
    We prove this for SNC. The case for WSC is similar.
    Let $\emph{SNC}(\varphi,\alpha,V,\Gamma)$ denote that $\varphi$ is the SNC of $\alpha$ on $V$ under $\Gamma$, and  $\emph{NC}(\varphi,\alpha,V,\Gamma)$ denote that $\varphi$ is the NC of $\alpha$ on $V$ under $\Gamma$.

    ($\Rto$) if $\emph{SNC}(\varphi,\alpha,V,\Gamma)$ holds, then $\emph{SNC}(\varphi,q,V,\Gamma')$ will be true. According to $\emph{SNC}(\varphi,\alpha,V,\Gamma)$ and $\alpha\equiv q$, we have $\Gamma' \models q\rto \varphi$, which means $\varphi$ is a NC of $q$ on $V$ under $\Gamma'$. Suppose $\varphi'$ is any NC of $q$ on $V$ under $\Gamma'$, then $\CTLforget(\Gamma',q)\models \alpha \rto \varphi'$ due to $\alpha\equiv q$, $\emph{IR}(\alpha \rto \varphi', \{q\})$ and $(\PP)$, \ie $\Gamma \models \alpha \rto \varphi'$ by Lemma \ref{lem:KF:eq}, this means $\emph{NC}(\varphi',\alpha,V,\Gamma)$. Therefore, $\Gamma \models \varphi \rto \varphi'$ by the definition of SNC and $\Gamma' \models \varphi \rto \varphi'$. Hence, $\emph{SNC}(\varphi,q,V,\Gamma')$ holds.

    ($\Lto$) if $\emph{SNC}(\varphi,q,V,\Gamma')$ holds, then $\emph{SNC}(\varphi,\alpha,V,\Gamma)$ will be true. According to $\emph{SNC}(\varphi,q,V,\Gamma')$, it's not difficult to know that $\CTLforget(\Gamma', \{q\})\models \alpha \rto \varphi$ due to $\alpha\equiv q$, $\emph{IR}(\alpha \rto \varphi, \{q\})$ and $(\PP)$, \ie $\Gamma \models \alpha \rto \varphi$ by Lemma \ref{lem:KF:eq}, this means $\emph{NC}(\varphi,\alpha,V,\Gamma)$. Suppose $\varphi'$ is any NC of $\alpha$ on $V$ under $\Gamma$. Then $\Gamma' \models q \rto \varphi'$ since $\alpha\equiv q$ and $\Gamma'=\Gamma \cup \{q\equiv \alpha\}$, which means $\emph{NC}(\varphi',q,V,\Gamma')$. According to $\emph{SNC}(\varphi,q,V,\Gamma')$, $\emph{IR}(\varphi \rto \varphi', \{q\})$ and $(\PP)$, we have
    $\CTLforget(\Gamma', \{q\})\models \varphi \rto \varphi'$, and $\Gamma \models \varphi \rto \varphi'$ by Lemma \ref{lem:KF:eq}. Hence, $\emph{SNC}(\varphi,\alpha,V, \Gamma)$ holds.
    \end{proof}


\textbf{Theorem}~\ref{thm:SNC:WSC:forget} Let $\varphi$ be a formula, $V\subseteq\Var(\varphi)$ and $q\in\Var(\varphi)- V$.
 \begin{enumerate}[(i)]
   \item $\CTLforget (\varphi \land q$, $(\Var(\varphi) \cup \{q\})- V)$
   is a SNC of $q$ on $V$ under $\varphi$.
   \item  $\neg\CTLforget (\varphi \land \neg q$, $(\Var(\varphi) \cup \{q\})- V)$
   is a WSC of $q$ on $V$ under $\varphi$.
 \end{enumerate}
\begin{proof}
 We will prove the SNC part, while it is not difficult to prove the WSC part according to Proposition \ref{dual}.
 Let ${\cal F}=\CTLforget(\varphi \wedge q, (\Var(\varphi) \cup \{q\})- V)$.

   The ``NC" part: It's easy to see that $\varphi \wedge q \models {\cal F}$ by {\bfseries (W)}. Hence, $\varphi\models q \rto {\cal F}$, this means
   ${\cal F}$ is a NC of $q$ on $P$ under $\varphi$.

  %  $\Gamma \models q \rto F$\\
%    $\Gamma \wedge q \models F$    \quad \quad \quad $(W)$\\
%    $\Rto$ $\Gamma\models q \rto F$   \quad \quad \quad $(\rightarrow +)$\\
%    \\
    The ``SNC" part: for all $\psi'$, $\psi'$ is the NC of $q$ on $V$ under $\varphi$, s.t. $\varphi \models {\cal F} \rto \psi'$.
    Suppose that there is a NC $\psi$ of $q$ on $V$ under $\varphi$ and $\psi$ is not logic equivalence with ${\cal F}$ under $\varphi$, s.t. $\varphi \models \psi \rto {\cal F}$.
    We know that $\varphi \wedge q \models \psi$ iff ${\cal F} \models \psi$ by {\bfseries (PP)}, since $\emph{IR}(\psi, (\Var(\varphi) \cup \{q\})- V)$. Hence, $\varphi \wedge {\cal F} \models \psi$ by $\varphi \wedge q \models \psi$ (by suppose).
    We can see that $\varphi \wedge \psi \models {\cal F}$ by suppose. Therefore, $\varphi \models \psi \lrto {\cal F}$, which means $\psi$ is logic equivalence with ${\cal F}$ under $\varphi$.
    This is contradict with the suppose. Then ${\cal F}$ is the SNC of $q$ on $P$ under $\varphi$.
 \end{proof}








\textbf{Theorem}~\ref{thm:inK:SNC} Let ${\cal K}= (\Hm, s)$ be an initial \MPK-structure with $\Hm=(S,R,L,s_0)$ on the finite set $\Ha$ of atoms, $V \subseteq \Ha$ and $q\in V'$ ($V' = \Ha - V$). Then:
 \begin{enumerate}[(i)]
   \item the SNC of $q$ on $V$ under ${\cal K}$ is $\CTLforget({\cal F}_{\Ha}({\cal K}) \wedge q, V')$.
   \item the WSC of $q$ on $V$ under ${\cal K}$ is $\neg \CTLforget({\cal F}_{\Ha}({\cal K}) \wedge \neg q, V')$.
 \end{enumerate}
\begin{proof}
(i)
As we know that any initial \MPK-structure ${\cal K}$ can be described as a characterizing formula ${\cal F}_{\Ha}({\cal K})$, then the SNC of $q$ on $V$ under ${\cal F}_{\Ha}({\cal K})$ is $\CTLforget({\cal F}_{\Ha}({\cal K}) \wedge q, \Ha - V)$.
 %We will prove that $\CTLforget({\cal F}_{V \cup \{q\}}({\cal K}_{|V \cup \{q\}}) \wedge q, q)  \equiv  \CTLforget({\cal F}_{\Ha}({\cal K}) \wedge q, \Ha - V)$.

%($\Rto$) $\forall {\cal K}_1 \in \Mod(\CTLforget({\cal F}_{V \cup \{q\}}({\cal K}_{|V \cup \{q\}}) \wedge q, q))$\\
%$\Rto$ there is an initial \MPK-structure ${\cal K}'$ such that ${\cal K}' \models {\cal F}_{V \cup \{q\}}({\cal K}_{|V \cup \{q\}}) \wedge q$ and ${\cal K}_1 \lrto_{\{q\}} {\cal K}'$\\
%$\Rto$ ${\cal K}' \lrto_{\Ha-(V\cup \{q\})} {\cal K}_{|V \cup \{q\}}$  \hfill (Theorem~\ref{CF})\\
%$\Rto$ ${\cal K}_1 \lrto_{\Ha-V} {\cal K}_{|V \cup \{q\}}$   \hfill (Proposition~\ref{div})\\
%$\Rto$ ${\cal K}_{|V \cup \{q\}} \lrto_{\Ha-(V \cup \{q\})} {\cal K}$   \hfill  (Proposition~\ref{pro:VQ})\\
%$\Rto$ ${\cal K}' \lrto_{\Ha-(V\cup \{q\})} {\cal K}$  \hfill (Proposition~\ref{div})\\
%$\Rto$ ${\cal K} \models {\cal F}_{\Ha}({\cal K}) \wedge q$\\
%$\Rto$ ${\cal K}_1 \lrto_{\Ha -V} {\cal K}$\\
%$\Rto$ ${\cal K}_1 \models \CTLforget({\cal F}_{\Ha}({\cal K}) \wedge q, \Ha - V)$
%
%$(\Lto)$ $\forall {\cal K}_1 \in \Mod(\CTLforget({\cal F}_{\Ha}({\cal K}) \wedge q, \Ha - V))$ \\
%$\Rto$ there an initial \MPK-structure ${\cal K}_2$ s.t.\ ${\cal K}_2 \models {\cal F}_{\Ha}({\cal K}) \wedge q$ and ${\cal K}_1 \lrto_{\Ha - V} {\cal K}_2$\\
%$\Rto$ ${\cal K}_2 \lrto_{{\O}} {\cal K}$   \hfill (Theorem~\ref{CF})\\
%$\Rto$ ${\cal K}_2 \lrto_{\Ha-(V\cup \{q\})} {\cal K}_{|V \cup \{q\}}$ due to ${\cal K}_{|V \cup \{q\}} \lrto_{\Ha-(V \cup \{q\})} {\cal K}$   \hfill  (Proposition~\ref{pro:VQ})\\
%$\Rto$ ${\cal K}_2 \models {\cal F}_{V \cup \{q\}}({\cal K}_{|V \cup \{q\}}) \wedge q$  \\
%$\Rto$ ${\cal K}_1 \models \CTLforget({\cal F}_{V \cup \{q\}}({\cal K}_{|V \cup \{q\}}) \wedge q, q)$.

(ii) This is proved by the dual property.
\end{proof}


%\textbf{Theorem}\ref{thm:state:bound}
%\begin{proof}
%If $|S| \leq 2^m$, this result clearly holds. If $|S| > 2^m$, let ${\cal K}' = {\cal K}_{|V}$, then it is apparent that ${\cal K} \lrto_{\Ha - V} {\cal K}'$ by Proposition~\ref{pro:VQ} and $(\Hm', s_0') \models \varphi$. In the worst case, the number of states in $S'$ is $2^m$ by the definition of ${\cal K}_{|V}$. Then the theorem is proved.
%\end{proof}

\textbf{Proposition}\ref{pro:time:alg1} Let $\varphi$ be a CTL formula and $V\subseteq \Ha$. The time and space complexity of
Algorithm~\ref{alg:compute:forgetting:by:VB} are $O(2^{m*2^m})$.\\
\begin{proof}
The time and space spent by Algorithm~\ref{alg:compute:forgetting:by:VB} is mainly the \textbf{for} cycles between lines 4 and 16.
Under a given number $i$ of states, there are $i^i$ number of relations, $i^{2^m}$ number of label functions and $i$ number of possible initial states. In this case, we need the memory for the initial \MPK-model in each time is $(i+i^i+i^{2^m}+1)$.
%Therefore, in the worst case is $i=2^m$, that is we need $(2^m + 2^{2m}+m*2^m+1)$ memory to store the initial \MPK-model.

For each $1\leq i \leq 2^m$, there is at most $i*i^i*i^{2^m}*i=i^2*i^{(i+2^m)}$ possible initial \MPK-models.
Suppose that we can obtain an initial \MPK-models in unit time (at each step), then we require $(2^m)^2*(2^m)^{(2^m+2^m)} = (2^m)^{2+2*2^m}$ steps in the worst case. Therefore, the time and space complexity are $O(2^{m*2^m})$.
\end{proof}


\bibliographystyle{named}
\bibliography{ijcai20}

\end{document}
