% This is samplepaper.tex, a sample chapter demonstrating the
% LLNCS macro package for Springer Computer Science proceedings;
% Version 2.20 of 2017/10/04
%
\documentclass[runningheads]{llncs}
%
\usepackage{graphicx}
% Used for displaying a sample figure. If possible, figure files should
% be included in EPS format.
%
% If you use the hyperref package, please uncomment the following line
% to display URLs in blue roman font according to Springer's eBook style:
% \renewcommand\UrlFont{\color{blue}\rmfamily}

\usepackage{soul}
\usepackage{url}
\usepackage[hidelinks]{hyperref}
\usepackage[utf8]{inputenc}
\usepackage[small]{caption}
\usepackage{graphicx}
\usepackage{amsmath}
\usepackage{booktabs}
\urlstyle{same}



\usepackage{setspace}
\usepackage{times}  %Required
\usepackage{helvet}  %Required
\usepackage{courier}  %Required
\usepackage{url}  %Required
%\usepackage{graphicx}  %Required

\usepackage{enumerate}


%\usepackage{algorithm}
%\usepackage{algorithmic}

\usepackage{amssymb}
\usepackage{enumerate}

\usepackage{subfigure}

\usepackage[linesnumbered,boxed,ruled,commentsnumbered]{algorithm2e}



\newcommand{\tuple}[1]{{\langle{#1}\rangle}}
\newcommand{\Mod}{\textit{Mod}}
\newcommand\ie{{\it i.e. }}
\newcommand\eg{{\it e.g.}}
%\newcommand\st{{\it s.t. }}
%\newtheorem{definition}{Definition}
%\newtheorem{examp}{Example}
%\newenvironment{example}{\begin{examp}\rm}{\end{examp}}
%\newtheorem{lemma}{Lemma}
%\newtheorem{proposition}{Proposition}
%\newtheorem{theorem}{Theorem}
%\newtheorem{corollary}[theorem]{Corollary}
%\newenvironment{proof}{{\bf Proof:}}{\hfill\rule{2mm}{2mm}\\ }
\newcommand{\rto}{\rightarrow}
\newcommand{\lto}{\leftarrow}
\newcommand{\lrto}{\leftrightarrow}
\newcommand{\Rto}{\Rightarrow}
\newcommand{\Lto}{\Leftarrow}
\newcommand{\LRto}{\Leftrightarrow}
\newcommand{\Var}{\textit{Var}}
\newcommand{\Forget}{\textit{Forget}}
\newcommand{\KForget}{\textit{KForget}}
\newcommand{\TForget}{\textit{TForget}}
%\newcommand{\forget}{\textit{forget}}
\newcommand{\Fst}{\textit{Fst}}
\newcommand{\dep}{\textit{dep}}
\newcommand{\term}{\textit{term}}
\newcommand{\literal}{\textit{literal}}

\newcommand{\Atom}{\mathcal{A}}
\newcommand{\SFive}{\textbf{S5}}
\newcommand{\MPK}{\textsc{k}}
\newcommand{\MPB}{\textsc{b}}
\newcommand{\MPT}{\textsc{t}}
\newcommand{\MPA}{\forall}
\newcommand{\MPE}{\exists}

\newcommand{\DNF}{\textit{DNF}}
\newcommand{\CNF}{\textit{CNF}}

\newcommand{\degree}{\textit{degree}}
\newcommand{\sunfold}{\textit{sunfold}}

\newcommand{\Pos}{\textit{Pos}}
\newcommand{\Neg}{\textit{Neg}}
\newcommand\wrt{{\it w.r.t.}}
\newcommand{\Hm} {{\cal M}}
\newcommand{\Hw} {{\cal W}}
\newcommand{\Hr} {{\cal R}}
\newcommand{\Hb} {{\cal B}}
\newcommand{\Ha} {{\cal A}}

\newcommand{\Dsj}{\triangledown}

\newcommand{\wnext}{\widetilde{\bigcirc}}
\newcommand{\nex}{\bigcirc}
\newcommand{\ness}{\square}
\newcommand{\qness}{\boxminus}
\newcommand{\wqnext}{\widetilde{\circleddash}}
\newcommand{\qnext}{\circleddash}
\newcommand{\may}{\lozenge}
\newcommand{\qmay}{\blacklozenge}
\newcommand{\unt} {{\cal U}}
\newcommand{\since} {{\cal S}}
\newcommand{\SNF} {\textit{SNF$_C$}}
\newcommand{\start}{\textbf{start}}
\newcommand{\Elm}{\textit{Elm}}
\newcommand{\simp}{\textbf{simp}}
\newcommand{\nnf}{\textbf{nnf}}

\newcommand{\CTL}{\textrm{CTL}}
\newcommand{\Ind}{\textrm{Ind}}
\newcommand{\Tran}{\textrm{Tran}}
\newcommand{\Sub}{\textrm{Sub}}
\newcommand{\NI}{\textrm{NI}}
\newcommand{\forget}{{\textsc{f}_\CTL}}
\newcommand{\ALL}{\textsc{a}}
\newcommand{\EXIST}{\textsc{e}}
\newcommand{\NEXT}{\textsc{x}}
\newcommand{\FUTURE}{\textsc{f}}
\newcommand{\UNTIL}{\textsc{u}}
\newcommand{\GLOBAL}{\textsc{g}}
\newcommand{\UNLESS}{\textsc{w}}
\newcommand{\Def}{\textrm{def}}
\newcommand{\IR}{\textrm{IR}}
\newcommand{\Tr}{\textrm{Tr}}
\newcommand{\dis}{\textrm{dis}}
\def\PP{\ensuremath{\textbf{PP}}}
\def\NgP{\ensuremath{\textbf{NP}}}
\def\W{\ensuremath{\textbf{W}}}
\newcommand{\Pre}{\textrm{Pre}}
\newcommand{\Post}{\textrm{Post}}


\newcommand{\CTLsnf}{{\textsc{SNF}_{\textsc{ctl}}^g}}
\newcommand{\ResC}{{\textsc{R}_{\textsc{ctl}}^{\succ, S}}}
\newcommand{\CTLforget}{{\textsc{F}_{\textsc{ctl}}}}
\newcommand{\Refine}{\textsc{Refine}}
\newcommand{\cf}{\textrm{cf.}}
\newcommand{\NEXP}{\textmd{\rm NEXP}}
\newcommand{\EXP}{\textmd{\rm EXP}}
\newcommand{\coNEXP}{\textmd{\rm co-NEXP}}
\newcommand{\NP}{\textmd{\rm NP}}
\newcommand{\coNP}{\textmd{\rm co-NP}}
\newcommand{\Pol}{\textmd{\rm P}}
\newcommand{\BH}[1]{\textmd{\rm BH}_{#1}}
\newcommand{\coBH}[1]{\textmd{\rm co-BH}_{#1}}
\newcommand{\Empty}{\varnothing}
\newcommand{\NLOG}{\textmd{\rm NLOG}}
\newcommand{\DeltaP}[1]{\Delta_{#1}^{p}}
\newcommand{\PIP}[1]{\Pi_{#1}^{p}}
\newcommand{\SigmaP}[1]{\Sigma_{#1}^{p}}

\begin{document}



%
\title{Contribution Title\thanks{Supported by organization x.}}
%
%\titlerunning{Abbreviated paper title}
% If the paper title is too long for the running head, you can set
% an abbreviated paper title here
%
\author{Renyan Feng\inst{1}\orcidID{0000-1111-2222-3333} \and
Erman Acar\inst{3}\orcidID{2222--3333-4444-5555} \and
Stefan Schlobach\inst{3}\orcidID{2222--3333-4444-5555} \and
Yisong Wang\inst{2,3}\orcidID{1111-2222-3333-4444}}
%
\authorrunning{F. Author et al.}
% First names are abbreviated in the running head.
% If there are more than two authors, 'et al.' is used.
%
\institute{Princeton University, Princeton NJ 08544, USA \and
Springer Heidelberg, Tiergartenstr. 17, 69121 Heidelberg, Germany
\email{lncs@springer.com}\\
\url{http://www.springer.com/gp/computer-science/lncs} \and
ABC Institute, Rupert-Karls-University Heidelberg, Heidelberg, Germany\\
\email{\{abc,lncs\}@uni-heidelberg.de}}
%


\maketitle              % typeset the header of the contribution
%
\begin{abstract}
This paper proved a method to computing the forgetting in \CTL\, which has been submitted to IJCAI, from the resolution proposed by Zhang at all by extending the resolution rules.

\keywords{Forgetting  \and CTL \and Model checking.}
\end{abstract}

\section{Introduction}
As a logical notion, \emph{forgetting} was first formally defined
in propostional and first order logics by Lin and Reiter~\cite{lin1994forget}.
Over the last twenty years, researchers have developed forgetting notions and theories not only in classical logic but also in other non-classical logic systems~\cite{eiter2019brief}, such as forgetting in logic programs under answer set/stable model semantics~\cite{DBLP:Zhang:AIJ2006,Eiter2008Semantic,Wong:PhD:Thesis,Yisong:KR:2012,Yisong:IJCAI:2013}, forgetting in description logic~\cite{Wang:AMAI:2010,Lutz:IJCAI:2011,zhao2017role} and knowledge forgetting in modal logic~\cite{Yan:AIJ:2009,Kaile:JAIR:2009,Yongmei:IJCAI:2011,fang2019forgetting}. In application, forgetting has been used in planning~\cite{lin2003compiling},  conflict solving \cite{Lang2010Reasoning,Zhang2005Solving},
%knowledge compilation \cite{Zhang2009Knowledge,Bienvenu2010Knowledge},
createing restricted views of ontologies~\cite{zhao2017role},
%{ZhaoSchmidt18a},
strongest and weakest definitions \cite{Lang2008On}, SNC (WSC) \cite{DBLP:journals/ai/Lin01} and so on.


Though forgetting has been extensively investigated from various aspects of different logical systems.
However, the existing forgetting method in propositional
logic, answer set programming, description logic and modal logic are not directly applicable in \CTL.
Similar with that in~\cite{Yan:AIJ:2009}, we research forgetting in \CTL\ from the semantic forgetting point of view.
And it is shown that our definition of forgetting satisfies those four postulates of forgetting.
%
%
\section{Preliminaries}
We start with some technical and notational preliminaries. Throughout this paper, we fix a finite set $\Ha$ of propositional variables (or atoms), and use $V$, $V'$ for subsets of $\Ha$. In the following several parts, we will introduce the structure we use for \CTL, syntactic and semantic of \CTL\ and the normal form $\CTLsnf$ (Separated Normal Form with Global Clauses for \CTL) of \CTL~\cite{zhang2009refined}.
\subsection{Model structure in \CTL}
 In general, a transition system~\footnote{According to \cite{Baier:PMC:2008},
a {\em transition system} TS is a tuple $(S, Act,\rto,I, AP, L)$ where
(1) $S$ is a set of states,
(2) $\textrm{Act}$ is a set of actions,
(3) $\rto\subseteq S\times \textrm{Act}\times S$ is a transition relation,
(4) $I\subseteq S$ is a set of initial states,
(5) $\textrm{AP}$ is a set of atomic propositions, and
(6) $L:S\rto 2^{\textrm{AP}}$ is a labeling function.} is described as a \emph{model\ structure} (or \emph{Kripke \ structure})(in this article, we treat transition system and model structure as the same thing), and a model structure is a triple $\Hm=(S,R,L)$~\cite{emerson1990temporal}, where
\begin{itemize}
  \item $S$ is a set of states,
  \item $R\subseteq S\times S$ is a total binary relation over $S$, \ie, for each state $s\in S$ there is a state $s'\in S$ such that $(s,s')\in R$, and
  \item $L$ is an interpretation function $S\rto 2^{\cal A}$ mapping every state to the set of atoms true at that state.
\end{itemize}
In this article, the same as~\cite{DBLP:journals/tcs/BrowneCG88}, all of our results apply only to finite Kripke structures.
Besides, we restrict ourselves to model structure $\Hm=(S,R,L,s_0)$ (similar with that in~\cite{zhang2009refined}) such that
\begin{itemize}
  \item there exists a state $s_0$, called the \emph{initial\ state}, such that for every state $s\in S$ there is a path $\pi_{s_0}$ s.t. $s\in \pi_{s_0}$.
\end{itemize}
We call a model structure $\Hm$ on a set $V$ of atoms if $L: S \rto 2^V$, \ie, the labeling function $L$ map every state to $V$ (not the $\Ha$).  A \emph{path} $\pi_{s_i}$ start from $s_i$ of $\Hm$ is a infinite sequence of states $\pi_{s_i}=(s_i, s_{i+1} s_{i+2},\dots)$, where for each $j$ ($i\leq j$), $(s_j, s_{j+1}) \in R$. By $s'\in \pi_{s_i}$ we mean that $s'$ is a state in the path $\pi_{s_i}$.

For a given model structure $(S,R,L,s_0)$ and $s\in S$,
the {\em computation tree}
$\Tr_n^{\cal M}(s)$ of $\cal M$(or simply $\Tr_n(s)$), that has depth $n$ and is rooted at $s$, is recursively defined as~\cite{DBLP:journals/tcs/BrowneCG88}, for $n\ge 0$,
\begin{itemize}
  \item $\Tr_0(s)$ consists of a single node $s$ with label $s$.
  \item $\Tr_{n+1}(s)$ has as its root a node $m$ with label  $s$, and
  if $(s,s')\in R$ then the node $m$ has a subtree $\Tr_n(s')$\footnote{Though
  some nodes of the tree may have the same label, they are different nodes in the tree.}.
\end{itemize}
By $s_n$ we mean the node at the $n$th level in tree $\Tr_m(s)$ $(m \geq n)$.

A {\em \MPK-structure} (or {\em \MPK-interpretation}) is a model structure
${\cal M}=(S, R, L, s_0)$ associating
with a state $s\in S$, which is written as $({\cal M},s)$ for convenience in the following.
In the case $s$ is an initial state of $\cal M$, the \MPK-structure is {\em initial}.


\subsection{Syntactic and semantic of \CTL}
In the following we briefly review the basic syntax and semantics
of the {\em Computation Tree Logic}
(\CTL\ in short)~\cite{DBLP:journals/toplas/ClarkeES86}. %Huth:BOOK:1999}.
%In  $\cal L$, the model of time is a tree-like structure in which the future is not
%determined; there are different paths in the future, any one of which might
%be the `actual' path that is realized.
%
The {\em signature} of $\cal L$ includes:
\begin{itemize}
  \item a finite set of Boolean variables, called {\em atoms} of $\cal L$: $\cal A$;
  \item the classical connectives: $\bot,\lor$ and $\neg$;
  \item the path quantifiers: $\ALL$ and $\EXIST$;
  \item the temporal operators: \NEXT, \FUTURE, \GLOBAL\, \UNTIL\ and \UNLESS, that
  means `neXt state', `some Future state', `all future states (Globally)', `Until' and `Unless', respectively;
  \item parentheses: ( and ).
\end{itemize}

The {\em (existential normal form or ENF in short) formulas} of
$\cal L$ are inductively defined via a Backus Naur form:
\begin{equation}\label{def:CTL:formulas}
  \phi ::= \bot\mid p \mid\neg\phi \mid \phi\lor\phi \mid
    \EXIST \NEXT \phi \mid
    %\EXIST \FUTURE \phi \mid
    \EXIST \GLOBAL \phi \mid
    \EXIST [\phi\ \UNTIL\ \phi]%.% \mid
    %\ALL \NEXT \phi \mid
%    \ALL \FUTURE \phi \mid
%    \ALL \GLOBAL \phi \mid
%    \ALL [\phi\ \UNTIL\ \phi]
\end{equation}
where $p\in\cal A$. The formulas $\phi\land\psi$ and $\phi\rto\psi$
are defined in a standard manner of propositional logic.
%Intuitively,
%the formula $\EXIST\NEXT\phi$ means that $\phi$ holds in some immediate successor
%of the current program state; the formula $\EXIST\GLOBAL\phi$ means
%that for some computation path $\phi$ holds at every state along the path; and the
%formula $\EXIST[\phi\UNTIL\psi]$ means that
%for some computation path there is an initial prefix of the path such
%that $\psi$ holds at the last state of the prefix  and $\phi$ holds at all other
%states along the prefix.
The other form formulas of $\cal L$ are abbreviated
using the forms of (\ref{def:CTL:formulas}).
%\begin{align*}
%  & \top =_{\Def} \neg\bot,\\
%  & \ALL [\phi\ \UNTIL\ \psi] =_{\Def}
%    \neg(\EXIST[\neg\psi\UNTIL \left(\neg \phi\land\neg\psi)]\lor\EXIST\GLOBAL\neg\psi\right),\\
%  & \ALL\FUTURE\phi=_{\Def}   \ALL [\top \UNTIL\ \phi],\\
%  & \EXIST\FUTURE\phi=_{\Def}\EXIST[\top\UNTIL\phi],\\
%  & \ALL\GLOBAL\phi =_{\Def} \neg\EXIST\FUTURE\neg\phi,\\
%  & \ALL\NEXT\phi =_{\Def} \neg\EXIST\NEXT\neg\phi,\\
%  & \ALL(\varphi \UNLESS \psi) =_{\Def} \neg \EXIST(\neg \psi \UNTIL (\neg \varphi \wedge \neg \psi)),\\
%  & \EXIST (\varphi \UNLESS \psi) =_{\Def} \neg \ALL(\neg \psi \UNTIL (\neg \varphi \wedge \neg \psi)).
%\end{align*}
Notice that, according to the
above definition for formulas of \CTL,
each of the \CTL\ {\em temporal connectives} has the form $XY$
where $X\in \{\ALL,\EXIST\}$ and  $Y\in\{\NEXT, \FUTURE, \GLOBAL, \UNTIL, \UNLESS\}$.
 %
The priorities for the \CTL\ connectives are assumed to be (from the highest to the lowest):
\begin{equation*}
  \neg, \EXIST\NEXT, \EXIST\FUTURE, \EXIST\GLOBAL, \ALL\NEXT, \ALL\FUTURE, \ALL\GLOBAL
  \prec \land \prec \lor \prec \EXIST\UNTIL, \ALL\UNTIL, \EXIST \UNLESS, \ALL \UNLESS, \rto.
\end{equation*}

We are now in the position to define the semantics of $\cal L$.
Let ${\cal M}=(S,R,L,s_0)$ be an model structure, $s\in S$ and $\phi$ a formula of $\cal L$.
The {\em satisfiability} relationship between ${\cal M},s$ and $\phi$,
written $({\cal M},s)\models\phi$, is inductively defined on the structure of $\phi$ as follows:
\begin{itemize}
  \item $({\cal M},s)\not\models\bot$;
  \item $({\cal M},s)\models p$ iff $p\in L(s)$;
  \item $({\cal M},s)\models \phi_1\lor\phi_2$ iff
    $({\cal M},s)\models \phi_1$ or $({\cal M},s)\models \phi_2$;
  \item $({\cal M},s)\models \neg\phi$ iff  $({\cal M},s)\not\models\phi$;
  \item $({\cal M},s)\models \EXIST\NEXT\phi$ iff
    $({\cal M},s_1)\models\phi$ for some $s_1\in S$ and $(s,s_1)\in R$;
  \item $({\cal M},s)\models \EXIST\GLOBAL\phi$ iff
    $\cal M$ has a path $(s_1=s,s_2,\ldots)$ such that
    $({\cal M},s_i)\models\phi$ for each $i\ge 1$;
  \item $({\cal M},s)\models \EXIST[\phi_1\UNTIL\phi_2]$ iff
    $\cal M$ has a path $(s_1=s,s_2,\ldots)$ such that, for some $i\ge 1$,
    $({\cal M},s_i)\models\phi_2$ and
    $({\cal M},s_j)\models\phi_1$ for each $j<i$.
\end{itemize}

Similar to the work in \cite{DBLP:journals/tcs/BrowneCG88,Bolotov:1999:JETAI},
only initial \MPK-structures are considered to be candidate models
in the following, unless explicitly stated. Formally,
an initial \MPK-structure $\cal K$ is a {\em model} of a formula $\phi$
whenever ${\cal K}\models\phi$.
Let $\Pi$ be a set of formulae, ${\cal K} \models \Pi$ if for each $\phi\in \Pi$ there is $\cal K \models \phi$.
We denote $\Mod(\phi)$  ($\Mod(\Pi)$) the set of models of $\phi$ ($\Pi$).
The formula $\phi$ (set $\Pi$ of formulae) is {\em satisfiable}
if $\Mod(\phi)\neq\emptyset$ ($\Mod(\Pi)\neq\emptyset$).
Since both the underlying states in model structure and signatures are finite, $\Mod(\phi)$ ($\Mod(\Pi)$)
is finite for any formula $\phi$ (set $\Pi$ of formulae).

Let $\phi_1$ and $\phi_2$ be two formulas or set of formulas.
By $\phi_1\models\phi_2$ we denote $\Mod(\phi_1)\subseteq\Mod(\phi_2)$.
By $\phi_1\equiv\phi_2$ we mean $\phi_1\models\phi_2$ and $\phi_2\models\phi_1$.
In this case $\phi_1$ is {\em equivalent} to $\phi_2$.

Let $\phi$ be a formula or set of formulas. By $\Var(\phi)$ we mean the set of atoms occurring in $\phi$.
Let $V\subseteq\cal A$.
The formula $\phi$ is $V$-{\em irrelevant}, written $\IR(\phi,V)$,
if there is a formula $\psi$ with
$\Var(\psi)\cap V=\emptyset$ such that $\phi\equiv\psi$.


\subsection{The normal form of \CTL}
It has proved that any \CTL\ formula $\varphi$ can be transformed into a set $T_\varphi$ of $\CTLsnf$ (Separated Normal Form with Global Clauses for \CTL) clauses in polynomial time such that $\varphi$ is satisfiable iff $T_\varphi$ is satisfiable~\cite{zhang2008first}.
An important difference between \CTL\ formulae and $\CTLsnf$ is that $\CTLsnf$ is an extension of the syntax of \CTL\ to use indices. These indices can be used to preserve a particular path context. The language of $\CTLsnf$ clauses is defined over an extension of \CTL. That is the language is based on: (1) the language of CTL; (2) a propositional constant $\start$; (3) a countably infinite index set $\Ind$; and (4) temporal operators: $\EXIST_{\tuple{ind}} \NEXT$, $\EXIST_{\tuple{ind}} \FUTURE$, $\EXIST_{\tuple{ind}} \GLOBAL$,$\EXIST_{\tuple{ind}} \UNTIL$ and $\EXIST_{\tuple{ind}} \UNLESS$.

The priorities for the $\CTLsnf$\ connectives are assumed to be (from the highest to the lowest):
\begin{align*}
  &\neg, (\EXIST\NEXT,\EXIST_{\tuple{ind}}\NEXT), (\EXIST\FUTURE ,\EXIST_{\tuple{ind}}\FUTURE), (\EXIST\GLOBAL,\EXIST_{\tuple{ind}} \GLOBAL), \ALL\NEXT, \ALL\FUTURE, \ALL\GLOBAL \\
  &\prec \land \prec \lor \prec (\EXIST\UNTIL,\EXIST_{\tuple{ind}} \UNTIL), \ALL\UNTIL, (\EXIST \UNLESS, ,\EXIST_{\tuple{ind}}\UNLESS), \ALL \UNLESS, \rto.
\end{align*}
Where the operators in the same brackets have the same priority.

%The $\CTLsnf$ clauses consists of formulae of the following forms: $\ALL \GLOBAL(\start \supset \bigvee_{j=1}^k m_j)$ (initial clause), $\ALL \GLOBAL(true \supset \bigvee_{j=1}^k m_j)$ (global clause), $\ALL \GLOBAL(\bigwedge_{i=1}^n l_i \supset \ALL \NEXT \bigvee_{j=1}^k m_j)$ (\ALL-step clause), $\ALL \GLOBAL(\bigwedge_{i=1}^n l_i \supset \EXIST_\tuple{ind} \NEXT \bigvee_{j=1}^k m_j)$ (\EXIST-step clause), $\ALL \GLOBAL(\bigwedge_{i=1}^n l_i \supset \ALL \FUTURE l)$ (\ALL-sometime clause) and $\ALL \GLOBAL(\bigwedge_{i=1}^n l_i \supset \EXIST_{\tuple{ind}} \FUTURE l)$ (\EXIST-sometime clause),
Before talked about the sematic of this language, we introduce the $\CTLsnf$ clauses at first. The $\CTLsnf$ clauses consists of formulae of the following forms.
\begin{align*}
& \ALL \GLOBAL(\start \supset \bigvee_{j=1}^k m_j) && (initial\ clause) \\
& \ALL \GLOBAL(true \supset \bigvee_{j=1}^k m_j) && (global\ clause) \\
& \ALL \GLOBAL(\bigwedge_{i=1}^n l_i \supset \ALL \NEXT \bigvee_{j=1}^k m_j) && (\ALL-step\ clause)\\
& \ALL \GLOBAL(\bigwedge_{i=1}^n l_i \supset \EXIST_\tuple{ind} \NEXT \bigvee_{j=1}^k m_j) && (\EXIST-step\ clause)\\
& \ALL \GLOBAL(\bigwedge_{i=1}^n l_i \supset \ALL \FUTURE l) && (\ALL-sometime\ clause)\\
& \ALL \GLOBAL(\bigwedge_{i=1}^n l_i \supset \EXIST_{\tuple{ind}} \FUTURE l) && (\EXIST-sometime\ clause).
\end{align*}
where $k \ge 0$, $n > 0$, $\start$ is a propositional constant, $l_i$ ($1 \le i \le n$), $m_j$ ($1 \le j \le k$) and $l$ are literals, that is atomic propositions or their negation and ind is an element of Ind (Ind is a countably infinite index set). By clause we mean the classical clause or the $\CTLsnf$ clause unless explicitly stated.
 %A set $T$ of $\CTLsnf$ clauses is satisfiable if there is a model $\Hm=(S, R, L, [\_], s_0)$ \st\ for all clause $C\in T$, $(\Hm, s_0) \models C$.

Formulae of $\CTLsnf$ over $\Ha$ are interpreted in \Ind-model structure $\Hm=(S,R,L, [\_], s_0)$, where $S$, $R$, $L$ and $s_0$ is the same as our model structure talked in 2.1 and $[\_]: \Ind \rto 2^{(S*S)}$ maps every index $ind \in \Ind$ to a successor function $[ind]$ which is a functional relation on $S$ and a subset of the binary accessibility relation $R$, such that for every $s\in S$ there exists exactly a state $s'\in S$ such that $(s,s')\in [ind]$ and $(s,s')\in R$.
%In this paper we do not need a strict tree model structure as in~\cite{zhang2009refined}, that is we do not those restrictions on $s_0$ due to that only for simplifying the proof but do not impact the satisfiability of a formula~\cite{zhang2009refined}.
An infinite path $\pi_{s_i}^{\tuple{ind}}$ is an infinite sequence of states $s_i, s_{i+1}, s_{i+2},\dots$ such that for every $j\geq i$, $(s_j, s_{j+1})\in [ind]$.
%Let $\pi$ be a path in \Ind-model structure $\Hm$, by $s\in \pi$ we mean that $s$ is a state in the path $\pi$.

Similarly, an {\em \Ind-structure} (or {\em \Ind-interpretation}) is a \Ind-model structure
${\cal M}=(S, R, L, [\_], s_0)$ associating
with a state $s\in S$, which is written as $({\cal M},s)$ for convenience in the following.
In the case $s$ is an initial state of $\cal M$, the \Ind-structure is {\em initial}.

The semantics of $\CTLsnf$ is an extension of the semantics of \CTL\ defined in Section 2.2 except using the \Ind-model structure $\Hm=(S,R,L,[\_],s_0)$ replace model structure, $({\cal M},s_i) \models \start$ iff $s_i=s_0$ and for all $\EXIST_{\tuple{ind}} \Gamma$ are explained in the path $\pi_{s_i}^{\tuple{ind}}$, where $\Gamma\in \{\NEXT, \GLOBAL, \UNTIL,\UNLESS\}$.
The semantics of $\CTLsnf$ is then
defined as shown next as an extension of the semantics of CTL defined in Section 2.2. Let $\varphi$ and $\psi$ be two $\CTLsnf$ formulae and $\Hm=(S,R,L,[\_],s_0)$ be an \Ind-model structure, the relation ``$\models$" between $\CTLsnf$ formulae and $\Hm$ is defined recursively as follows:
\begin{itemize}
  \item $({\cal M},s_i) \models \start$ iff $s_i=s_0$;
  \item $({\cal M},s_i)\models \EXIST_{\tuple{ind}} \NEXT \psi$ iff for the path $\pi_{s_i}^{\tuple{ind}}$, $(\Hm, s_{i+1})\models \psi$;
  \item $({\cal M},s_i)\models \EXIST_{\tuple{ind}}\GLOBAL\psi$ iff
    for every $s_j \in \pi_{s_i}^{\tuple{ind}}$,
    $(\Hm,s_j) \models \psi$;
  \item $({\cal M},s_i)\models \EXIST_{\tuple{ind}}[\varphi\UNTIL\psi]$ iff
      there exists $s_j\in \pi_{s_i}^{\tuple{ind}}$ such that $(\Hm,s_j) \models \psi$ and for every $s_k \in \pi_{s_i}^{\tuple{ind}}$, if $i\leq k < j$, then $(\Hm,s_k) \models \varphi$;
  \item $(\Hm,s_i) \models \EXIST_{\tuple{ind}} \FUTURE \psi$ iff $(\Hm,s_i) \models \EXIST_{\tuple{ind}}[\top \UNTIL\psi]$;
  \item $({\cal M},s_i)\models \EXIST_{\tuple{ind}}[\varphi\UNLESS\psi]$ iff $(\Hm,s_i) \models \EXIST_{\tuple{ind}}\GLOBAL \varphi$ or $({\cal M},s_i)\models \EXIST_{\tuple{ind}}[\varphi\UNTIL\psi]$.
\end{itemize}
The semantics of the remaining operators is analogous to that given previously but in the
extended \Ind-model structure ${\cal M}=(S, R, L, [\_],s_0)$.
A $\CTLsnf$ formula $\varphi$ is satisfiable, iff for some \Ind-model structure $\Hm=(S,R,L,[\_],s_0)$, $(\Hm,s_0)\models \varphi$, and unsatisfiable otherwise. And if $(\Hm,s_0)\models \varphi$ then $(\Hm,s_0)$ is called a \Ind-model of $\varphi$, and we say that $(\Hm,s_0)$ satisfies $\varphi$.
By $T \wedge \varphi$ we mean $\bigwedge_{\psi\in T} \psi \wedge \varphi$, where $T$ is a set of formulae.
Other terminologies are similar with those in section 2.2.




\section{Problem Definition}
In order to define our problem, \ie forgetting in \CTL, we review our definition of $V$-bisimulation (read ?? for more detials).
\begin{definition}\label{def:Vbi}
Let $V\subseteq\cal A$
%${\cal M}_i=(S_i,R_i,L_i,s_0^i)~(i=1,2)$ be model structures
and ${\cal K}_i=({\cal M}_i,s_i)~(i=1,2)$ be \MPK-structures (Ind-structures).
Then $({\cal K}_1,{\cal K}_2)\in\cal B$ if and only if
  \begin{enumerate}[(i)]
    \item $L_1(s_1)- V = L_2(s_2)-V$,
    \item for every $(s_1,s_1')\in R_1$, there is $(s_2,s_2')\in R_2$
    such that $({\cal K}_1',{\cal K}_2')\in \Hb$, and
    \item for every $(s_2,s_2')\in R_2$, there is $(s_1,s_1')\in R_1$
%    such that $({\cal K}_1',{\cal K}_2')\in \Hb$,
   \end{enumerate}
 where ${\cal K}_i'=({\cal M}_i,s_i')$ with $i\in\{1,2\}$.
\end{definition}

\begin{proposition}\label{div}
Let $i\in \{1,2\}$, $V_1,V_2\subseteq\cal A$, $s_i'$s be two states and
  $\pi_i'$s be two pathes,
and ${\cal K}_i=({\cal M}_i,s_i)~(i=1,2,3)$ be \MPK-structures (Ind-structures)
 such that
${\cal K}_1\lrto_{V_1}{\cal K}_2$ and ${\cal K}_2\lrto_{V_2}{\cal K}_3$.
 Then:
 \begin{enumerate}[(i)]
   \item $s_1'\lrto_{V_i}s_2'~(i=1,2)$ implies $s_1'\lrto_{V_1\cup V_2}s_2'$;
   \item $\pi_1'\lrto_{V_i}\pi_2'~(i=1,2)$ implies $\pi_1'\lrto_{V_1\cup V_2}\pi_2'$;
   \item for each path $\pi_{s_1}$ of $\Hm_1$ there is a path $\pi_{s_2}$  of $\Hm_2$ such that $\pi_{s_1} \lrto_{V_1} \pi_{s_2}$, and vice versa;
   \item ${\cal K}_1\lrto_{V_1\cup V_2}{\cal K}_3$;
   \item If $V_1 \subseteq V_2$ then ${\cal K}_1 \lrto_{V_2} {\cal K}_2$.
 \end{enumerate}
\end{proposition}

Now we give the formal definition of forgetting in \CTL\ from the semantic forgetting point view.
\begin{definition}[Forgetting]\label{def:V:forgetting}
  Let $V\subseteq\cal A$ and $\phi$ a \CTL\ formula.
A \CTL\ formula $\psi$ with $\Var(\psi)\cap V=\emptyset$
is a {\em result of forgetting $V$ from} $\phi$, if
\begin{equation}
  \Mod(\psi)=\{{\cal K}\mbox{ is initial}\mid \exists {\cal K}'\in\Mod(\phi)\ \&\ {\cal K}'\lrto_V{\cal K}\}.
\end{equation}
Where $\cal K$ and ${\cal K}'$ are $\MPK$-structures.
\end{definition}
Note that if both $\psi$ and $\psi'$ are results of forgetting $V$ from $\phi$ then
$\Mod(\psi)=\Mod(\psi')$, \ie, $\psi$ and $\psi'$ have the same models. In the sense
of equivalence the forgetting result is unique (up to equivalence).


Similar with the $V$-bisimulation between \MPK-structures, we define the $\tuple{V,I}$-bisimulation between \Ind-structures as follows:
\begin{definition}\label{def:VInd:bisimulation}
\textbf{($\tuple{V,I}$-bisimulation)}
Let $\Hm_i=(S_i, R_i, L_i, [\_]_i, s_0^i)$ with $i\in \{1, 2\}$ be two \Ind-structures, $V$ be a set of atoms and $I \subseteq Ind$. The $\tuple{V,I}$-bisimulation $\beta_{\tuple{V,I}}$ between initial \Ind-structures is a set that satisfy $((\Hm_1, s_0^1), (\Hm_2, s_0^2)) \in \beta_{\tuple{V,I}}$  if and only if $(\Hm_1, s_0^1) \lrto_V (\Hm_2, s_0^2)$ and $\forall j \notin I$ there is
\begin{enumerate}[(i)]
  \item $\forall (s, s_1)\in [j]_1$ there is $(s',s_1')\in [j]_2$ such that $s\lrto_V s'$ and $s_1 \lrto_V s_1'$, and
  \item $\forall (s', s_1')\in [j]_2$ there is $(s,s_1)\in [j]_1$ such that $s\lrto_V s'$ and $s_1 \lrto_V s_1'$.
\end{enumerate}
%$\forall j \notin I$ there is $[j]_1 = [j]_2$.
\end{definition}
Apparently, this definition is similar with our concept $V$-bisimulation except that this $\tuple{V,I}$-bisimulation has introduced the index.
%Besides, it is not difficult to prove $\tuple{V,I}$-bisimulation possess those properties (talked-above) possessed by $V$-bisimulation.

\begin{proposition}\label{pro:VI:div}
Let $i\in \{1,2\}$, $V_1,V_2\subseteq\cal A$, $I_1, I_2 \subseteq Ind$
and ${\cal K}_i=({\cal M}_i,s_0^i)~(i=1,2,3)$ be Ind-structures
 such that
${\cal K}_1\lrto_{\tuple{V_1, I_1}}{\cal K}_2$ and ${\cal K}_2\lrto_{\tuple{V_2,I_2}}{\cal K}_3$.
 Then:
 \begin{enumerate}[(i)]
  % \item $s_1'\lrto_{V_i}s_2'~(i=1,2)$ implies $s_1'\lrto_{V_1\cup V_2}s_2'$;
%   \item $\pi_1'\lrto_{V_i}\pi_2'~(i=1,2)$ implies $\pi_1'\lrto_{V_1\cup V_2}\pi_2'$;
%   \item for each path $\pi_{s_1}$ of $\Hm_1$ there is a path $\pi_{s_2}$  of $\Hm_2$ such that $\pi_{s_1} \lrto_{V_1} \pi_{s_2}$, and vice versa;
   \item ${\cal K}_1\lrto_{\tuple{V_1\cup V_2, I_1 \cup I_2}}{\cal K}_3$;
   \item If $V_1 \subseteq V_2$ and $I_1 \subseteq I_2$ then ${\cal K}_1 \lrto_{\tuple{V_2, I_2}} {\cal K}_2$.
 \end{enumerate}
\end{proposition}
\begin{proof}
%This can be proved similarly with Proposition~\ref{div}.
(i) By Proposition~\ref{div} we have ${\cal K}_1\lrto_{V_1\cup V_2}{\cal K}_3$. For (i) of Definition~\ref{def:VInd:bisimulation} we can prove it as follows:
$\forall (s,s_1) \in [j]_1$ there is a $(s', s_1') \in [j]_2$ such that $s\lrto_{V_1} s'$ and $s_1 \lrto_{V_1} s_1'$ and there is a $(s'', s_1'') \in [j]_3$ such that $s'\lrto_{V_2} s''$ and $s_1' \lrto_{V_2} s_1''$, and then we have $\forall (s,s_1) \in [j]_1$ there is a $(s'', s_1'') \in [j]_3$ such that $s  \lrto_{V_1\cup V_2} s''$ and $s_1 \lrto_{V_1\cup V_2} s_1''$. The (ii) of Definition~\ref{def:VInd:bisimulation} can be proved similarly.

(ii) This can be proved from (i).
\end{proof}



\section{The Calculus}
\emph{Resolution} in \CTL\ is a method to decide the satisfiability of a \CTL\ formula.
In this paper, we will explore a resolution-based method to compute forgetting in \CTL.
In this part we use the transformation rules Trans(1) to Trans(12) and resolution rules (SRES1), \dots, (SRES8), RW1, RW2, (ERES1), (ERES2) in~\cite{zhang2009refined}.

The key problems of this method include (1) How to fill the gap between \CTL\ and $\CTLsnf$; and (2) How to eliminate the irrelevant atoms in the formula.
We will resolve these two problems by $\tuple{V,I}$-bisimulation and \emph{substitution} operator.
For convenient, we use $V\subseteq \Ha$ denote the set we want to forget, $V' \subseteq \Ha$ with $V \cap V'={\O}$ the set of atoms ($I$ be the set of index) introduced in the transformation process, $\varphi$  the \CTL\ formula, $T_{\varphi}$ be the set of $\CTLsnf$ clause obtained from $\varphi$ by using transformation rules  and $\Hm=(S,R,L,[\_], s_0)$ unless explicitly stated.
 Let $T$, $T'$ be two set of formulae, $I$ a set of indexes and $V''\subseteq \Ha$, by $T\equiv_{\tuple{V'', I}} T'$ we mean that $\forall (\Hm, s_0) \in \Mod(T)$ there is a $(\Hm', s_0')$ such that $(\Hm,s_0) \lrto_{\tuple{V'', I}} (\Hm',s_0')$ and $(\Hm', s_0') \models T'$ and vice versa.


\begin{proposition}
Let $P$, $P_i$ and $\varphi_i$ be \CTL\ formulas, then
\begin{enumerate}[(i)]
  \item $P\supset \EXIST_{\tuple{ind}} \NEXT \varphi_1 \wedge \dots \wedge P\supset \EXIST_{\tuple{ind}}\NEXT \varphi_n  \equiv_{\tuple{\emptyset, \{ind\}}} P\supset \EXIST \NEXT \bigwedge_{i\in \{0,\dots, n\}}\varphi_i$,
  \item $P_1\supset \EXIST_{\tuple{ind}} \NEXT \varphi_1 \wedge \dots \wedge P_n\supset \EXIST_{\tuple{ind}}\NEXT \varphi_n \in T \equiv_{\tuple{\emptyset, \{ind\}}} \bigwedge_{e \in 2^{\{0,\dots, n\}} \setminus \{\emptyset\}}(\bigwedge_{i\in e}P_i\supset \EXIST \NEXT (\bigwedge_{i\in e}\varphi_i))$,
  \item $P\supset \EXIST_{\tuple{ind}} \FUTURE \varphi_1 \wedge \dots \wedge P\supset \EXIST_{\tuple{ind}} \FUTURE \varphi_n \in T \equiv_{\tuple{\emptyset, \{ind\}}} P\supset \bigvee\EXIST\FUTURE (\varphi_{j_1} \wedge \EXIST\FUTURE(\varphi_{j_2} \wedge \EXIST\FUTURE(\dots \wedge \EXIST\FUTURE \varphi_{j_n})))$, where $(j_1, \dots, j_n)$ are sequences of all elements in $\{0, \dots, n\}$,
  \item $P\supset (C \vee \EXIST_{\tuple{ind}} \NEXT \varphi_1) \wedge P \supset \EXIST_{\tuple{ind}} \NEXT \varphi_2 \equiv_{\tuple{\emptyset, \{ind\}}} P \supset ((C \wedge \EXIST \NEXT \varphi_2) \vee \EXIST \NEXT (\varphi_1 \wedge \varphi_2))$
\end{enumerate}
\end{proposition}
\begin{proof}
It is easy to check.
\end{proof}

The algorithm of computing the forgetting in \CTL\ is as Algorithm~\ref{alg:compute:forgetting:by:Resolution}.
\begin{algorithm}[!h]
\caption{Computing forgetting - A resolution-based method}% ??????
\label{alg:compute:forgetting:by:Resolution}
%\LinesNumbered %?????????
\KwIn{A CTL formula $\varphi$ and a set $V$ of atoms}% ????????
\KwOut{$\CTLforget(\varphi, V)$}% ????
$T={\O}$ // the initial set of $\CTLsnf$ clauses of $\varphi$ \;
$T' = {\O}$ // the set of $\CTLsnf$ clauses without index\;
$V'={\O}$ // the set of atoms introduced in the process of transforming $\varphi$ into $\CTLsnf$ clauses\;


$T, V' \lto Tran(\varphi, V)$\;

$Res \lto Res(T, V')$\;

$\Gamma = \Sub(Res, V')$\;
%$\Gamma_1 = \Elm(\Gamma, \Gamma)$\;
%Using distribution law by viewing each $\CTLsnf$ as an atom to change $\Gamma_1$ into a set $L$ of sets of $\CTLsnf$; // from $(a \vee b) \wedge c$ to $(a \wedge c) \vee (b \wedge c)$\;
%$L'=R(\NI(L)_{\CTL})$\;
%\Return $\bigwedge_{\psi \in L'} \psi$.
%Let $\Omega$, which obtained from $\Elm(\EXIST\FUTURE(\Gamma), \Gamma)$ by using distribution law by viewing each $\CTLsnf$ clause as an atom, be a set of sets of $\CTLsnf$ clauses.\;

$\Omega\lto \Elm(\EXIST\FUTURE(\Gamma), \Gamma)$\;
$\Gamma_1\lto \NI(\Omega)$\;
%Transform $\Gamma_1$ into a formula~\footnote{$X$ is a set of formulas, in this paper set $X$ and $\bigwedge X$ are expressing the conjunction of elements in $X$.}:
% \[\Gamma_2=\bigwedge X \wedge \bigwedge_{p\in (V'\setminus \Gamma)\cup V^F} (p\supset \varphi_1 \vee \dots \vee p\supset \varphi_n)\]

\Return $\bigwedge_{\psi \in R(\Gamma_1)_{\CTL}} \psi$.
\end{algorithm}


\begin{algorithm}[!h]
\caption{$Tran(\varphi, V)$}% ??????
\label{alg:compute:transformation}
%\LinesNumbered %?????????
\KwIn{A CTL formula $\varphi$ and a set $V$ of atoms}% ????????
\KwOut{A set $T$ of $\CTLsnf$ clauses and a set $V'$ of atoms}% ????
$T={\O}$ // the initial set of $\CTLsnf$ clauses of $\varphi$ \;
$OldT=\{\start \supset z, z \supset \varphi\}$\;
$V'=\{z\}$\;
\While {$OldT\neq T$} {
    $OldT=T$\;
    $R={\O}$\;
    $X={\O}$\;
    \If {Chose a formula $\psi\in OldT$ that dose not a $\CTLsnf$ clause}{
    Using a match rule $Rl$ to transform $\psi$ into a set $R$ of $\CTLsnf$ clauses\;
    $X$ is the set of atoms introduced by using $Rl$\;
    $V' =V' \cup X$\;
    $T=OldT\setminus \{\psi\} \cup R$\;
    }
}

\end{algorithm}


\begin{algorithm}[!h]
\caption{$Res(T, V')$}% ??????
\label{alg:compute:Res}
%\LinesNumbered %?????????
\KwIn{A set $T$ of $\CTLsnf$ clauses and a set $V'$ of atoms}% ????????
\KwOut{A set $Res$ of $\CTLsnf$ clauses}% ????

$S=\{C | C\in T$ and $\Var(C) \cap V= {\O}\}$\;
$\Pi=T\setminus S$ \;
\For {($p\in V\cup V')$} {
    $\Pi'=\{C \in \Pi| p\in \Var(C)\}$ \;
    $\Sigma = \Pi \setminus \Pi'$\;
    \For {($C\in \Pi'$ s.t. $p$ appearing in  $C$ positively)} {
        \For {($C'\in\Pi'$ s.t. $p$ appearing in  $C'$ negatively and $C$, $C'$ are resolvable)}{
            $\Sigma = \Sigma \cup \{res(C,C')\}$\;
            $\Pi' = \Pi' \cup \{C''=res(C,C') | p\in \Var(C'')\}$\;
        }
    }
    $\Pi= \Sigma$\;
}

$Res=\Pi \cup S$\;

\end{algorithm}


%
% ---- Bibliography ----
%
% BibTeX users should specify bibliography style 'splncs04'.
% References will then be sorted and formatted in the correct style.
%
 \bibliographystyle{splncs04}
 \bibliography{ijcai20}
%
\begin{thebibliography}{8}
\bibitem{ref_article1}
Author, F.: Article title. Journal \textbf{2}(5), 99--110 (2016)

\bibitem{ref_lncs1}
Author, F., Author, S.: Title of a proceedings paper. In: Editor,
F., Editor, S. (eds.) CONFERENCE 2016, LNCS, vol. 9999, pp. 1--13.
Springer, Heidelberg (2016). \doi{10.10007/1234567890}

\bibitem{ref_book1}
Author, F., Author, S., Author, T.: Book title. 2nd edn. Publisher,
Location (1999)

\bibitem{ref_proc1}
Author, A.-B.: Contribution title. In: 9th International Proceedings
on Proceedings, pp. 1--2. Publisher, Location (2010)

\bibitem{ref_url1}
LNCS Homepage, \url{http://www.springer.com/lncs}. Last accessed 4
Oct 2017
\end{thebibliography}
\end{document}
